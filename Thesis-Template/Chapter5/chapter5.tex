\chapter{Kết luận và hướng phát triển}
\label{Chapter5}
\section{Kết luận}
Thuật toán gom nhóm phân cấp tích tụ khi sử dụng văn bản với thể hiện là doc2vec cho ra kết quả tôt hơn so với thuật toán sử dụng TFIDF (trường hợp ARI).
Như vậy, không những giúp cho việc giảm dung lượng bộ nhớ lúc chạy thuật toán, doc2vec còn cho ra kết quả tốt hơn.
Áp dụng doc2vec trong việc thể hiện văn bản còn có thể giúp cho việc tăng số lượng văn bản lúc chạy.
Trong khi đó, TFIDF do có số chiều quá lớn nên khi chạy với ngữ liệu lớn sẽ có thể bị chết may do hết bộ nhớ.
Vì vậy, doc2vec là thể hiện văn bản thích hợp cho ngữ liệu lớn.

\section{Hướng phát triển}
Hiện tại, vấn đề về dung lượng bộ nhớ lúc chạy thuật toán doc2vec tạm thời được khắc phục.
Tuy nhiên, ta vẫn còn tồn tại vấn đề thời gian thực thi thuật toán.
Vì vậy, trong tương lai, ta có thể đào sâu vào hướng tiếp cận này để cải thiện thời gian thực thi.
Như vậy, ta có thể sẽ có thể chạy được nhanh hơn nếu cải tiến được thời gian thực thi.

%kết quả đạt được (1-3 đoạn)

%đóng góp mới cho bài toán
	%sử dụng Doc2vec
	%sử dụng biến liên tục

%Kiến nghị, đề xuất cho hướng nghiên cứu tiếp theo

%%(4-5)