% Template KLTN cho SV trường ĐHKHTN
% Liên hệ: nqminh@fit.hcmus.edu.vn
% Last update: 08/06/2016

% Chú ý: đọc các phần chú ý đóng khung của file này và chỉnh lại cho phù hợp.
% Trước khi build, xóa hết các file được tạo ra trong quá trình build trước đó, và build theo thứ tự: BIB > PDF > PDF.
% Nếu cập nhật tài liệu tham khảo, cũng cần build lại theo cách trên.

\documentclass[oneside,a4paper,14pt]{extreport}

% Packages đồ họa
\usepackage[svgnames,pdf]{pstricks}
%\usepackage[usenames,dvipsnames]{pstricks}
%\usepackage{pstricks-add}
%\usepackage{auto-pst-pdf}
%\usepackage{pst-pdf}
 
\usepackage{epsfig}
\usepackage{pst-grad} % For gradients
\usepackage{pst-plot} % For axes
\usepackage[space]{grffile} % For spaces in paths
\usepackage{etoolbox} % For spaces in paths

% Font tiếng Việt
\usepackage[T5]{fontenc}
\usepackage[utf8]{inputenc}
\DeclareTextSymbolDefault{\DH}{T1}

% Packages được thêm vào
\usepackage{makecell}
\usepackage{tabularx}
%\usepackage{tabulary}
\usepackage{floatrow}
\floatsetup[table]{capposition=top}

% Tài liệu tham khảo
\usepackage[
	sorting=nty,
	backend=bibtex,
	defernumbers=true]{biblatex}
\usepackage[unicode]{hyperref} % Bookmark tiếng Việt
\addbibresource{References/references.bib}

\makeatletter
\def\blx@maxline{77}
\makeatother


% Chèn hình, các hình trong luận văn được để trong thư mục Images/
\usepackage{graphicx}
\graphicspath{ {Images/} }

% Chèn và định dạng mã nguồn
\usepackage{listings}
\usepackage{color}
\definecolor{codegreen}{rgb}{0,0.6,0}
\definecolor{codegray}{rgb}{0.5,0.5,0.5}
\definecolor{codepurple}{rgb}{0.58,0,0.82}
\definecolor{backcolour}{rgb}{0.95,0.95,0.92}
\lstdefinestyle{mystyle}{
    backgroundcolor=\color{backcolour},   
    commentstyle=\color{codegreen},
    keywordstyle=\color{magenta},
    numberstyle=\tiny\color{codegray},
    stringstyle=\color{codepurple},
    basicstyle=\footnotesize,
    breakatwhitespace=false,         
    breaklines=true,                 
    captionpos=b,                    
    keepspaces=true,                 
    numbers=left,                    
    numbersep=5pt,                  
    showspaces=false,                
    showstringspaces=false,
    showtabs=false,                  
    tabsize=2
}
\lstset{style=mystyle}

% Chèn và định dạng mã giả
\usepackage{amsmath}
\usepackage{algorithm}
\usepackage[noend]{algpseudocode}
\makeatletter
\def\BState{\State\hskip-\ALG@thistlm}
\makeatother

% Bảng biểu
\usepackage{multirow}
\usepackage{array}
\newcolumntype{L}[1]{>{\raggedright\let\newline\\\arraybackslash\hspace{0pt}}m{#1}}
\newcolumntype{C}[1]{>{\centering\let\newline\\\arraybackslash\hspace{0pt}}m{#1}}
\newcolumntype{R}[1]{>{\raggedleft\let\newline\\\arraybackslash\hspace{0pt}}m{#1}}
\newcolumntype{Y}{>{\centering\arraybackslash}X}
%\newcolumntype{Y}[0]{>{\centering\let\newline\\\arraybackslash\hspace{0pt}}X}
%\newcolumntype{Y}[0]{>{\centering\arraybackslash}X}%

%\renewcommand\tabularxcolumn[1]{>{\centering\arraybackslash}m{#1}}

% Đổi tên mặc định
\renewcommand{\chaptername}{Chương}
\renewcommand{\figurename}{Hình}
\renewcommand{\tablename}{Bảng}
\renewcommand{\contentsname}{Mục lục}
\renewcommand{\listfigurename}{Danh sách hình}
\renewcommand{\listtablename}{Danh sách bảng}
\renewcommand{\appendixname}{Phụ lục}

% Dãn dòng 1.5
\usepackage{setspace}
\onehalfspacing

% Thụt vào đầu dòng
\usepackage{indentfirst}

% Canh lề
\usepackage[
  top=35mm,
  bottom=30mm,
  left=35mm,
  right=20mm,
  includefoot]{geometry}
  
 %Tạo độ sâu trong TOC
\setcounter{tocdepth}{3}
\setcounter{secnumdepth}{3}

% Trang bìa
\usepackage{tikz}
\usetikzlibrary{calc}
\newcommand\HRule{\rule{\textwidth}{1pt}}

% ========================================================================================= %
% CHÚ Ý: Thông tin chung về KLTN - sinh viên điền vào đây để tự động update các trang khác  %
% ========================================================================================= %
\newcommand{\tenSV}{Tên~Sinh~Viên} % Dấu ~ là khoảng trắng không được tách (các chữ nối với nhau bằng dấu ~ sẽ nằm cùng 1 dòng
\newcommand{\mssv}{1234567}
\newcommand{\tenKL}{Sử~dụng~LaTeX trong Khoá~luận~tốt~nghiệp} % Chú ý dấu ~ trong tên khóa luận
\newcommand{\tenGVHD}{Tên~Giáo~Viên}
\newcommand{\tenBM}{Công nghệ tri thức}

\begin{document}

\begin{titlepage}

\begin{center}
ĐẠI HỌC QUỐC GIA THÀNH PHỐ HỒ CHÍ MINH\\
TRƯỜNG ĐẠI HỌC KHOA HỌC TỰ NHIÊN\\[2cm]

{ \LARGE \bfseries Họ và tên HVCH\\[1cm] } 

{ \huge \bfseries TÊN ĐỀ TÀI LUẬN VĂN\\[2cm] } 

\begin{flushleft} \large
Chuyên ngành: Khoa học máy tính\\
Mã số chuyên ngành: 12345\\[2cm]
\end{flushleft}

\large LUẬN VĂN THẠC SĨ: KHOA HỌC MÁY TÍNH\\[2cm]


\begin{flushright} \large
NGƯỜI HƯỚNG DẪN KHOA HỌC:\\
1. Tên người hướng dẫn
\end{flushright}

\begin{tikzpicture}[remember picture, overlay]
  \draw[line width = 2pt] ($(current page.north west) + (2cm,-2cm)$) rectangle ($(current page.south east) + (-1.5cm,2cm)$);
\end{tikzpicture}

\vfill
Tp. Hồ Chí Minh, Năm 2015

\end{center}

\end{titlepage}
% Sasu trang Title, các bạn chèn nhận xét gủa GVHD và GVPB. Nhận xét sẽ được giáo vụ phát sau buổi bảo vệ để các bạn đóng quyển.

\pagenumbering{roman} % Đánh số i, ii, iii, ...

%\addcontentsline{toc}{chapter}{Lời cam đoan}
%\chapter*{Lời cam đoan}
\label{reassurances}

Tôi cam đoan ...

\addcontentsline{toc}{chapter}{Lời cảm ơn}
\chapter*{Lời cảm ơn}
\label{thanks}

Tôi xin chân thành cảm ơn ...

\addcontentsline{toc}{chapter}{Đề cương chi tiết}
\include{Appendix/decuong}

% Mục lục, danh sách hình, danh sách bảng
\addcontentsline{toc}{chapter}{Mục lục}
\tableofcontents
\listoffigures
\listoftables

\addcontentsline{toc}{chapter}{Tóm tắt}
\chapter*{Tóm tắt}

\clearpage

\pagenumbering{arabic} % Đánh số 1, 2, 3, ...

% Các chương nội dung
\chapter{Giới thiệu}
\label{Chapter1}

\section{Giới thiệu về gom nhóm văn bản}

%Gom nhóm là gì?
Gom nhóm là vấn đề được nghiên cứu nhiều trong khai thác dữ liệu ~\cite{Jain-Dubes, Jardin-Rijsbergen, Ji-Xu, Jolliffee}.
The như Kumar định nghĩa ~\cite{Vipin-Kumar} ``\textit{Gom nhóm là công việc tìm kiếm và chia những đối tượng gần giống nhau vào thành những nhóm trong dữ liệu sao cho nhũng nhóm này có ý nghĩa, hữu dụng hoặc cả hai}''. %\cite{1984-TeX-Knuth}
Nếu như mục đích của gom nhóm là tìm kiếm ý nghĩa của dữ liệu thì các nhóm được gom sẽ thể hiện cấu trúc của dữ liệu.
Trong trường hợp ngược lại, nếu sử dụng gom nhóm với mục đích hữu dụng thì gom nhóm được xem là cầu nối, quá trình trung gian để thực hiện cho những mục đích khác như tóm tắt dữ liệu.
Bất kể việc sử dụng gom nhóm cho mục đích ý nghĩa hay là hữu dụng thì gom nhóm đóng vai trò quan trọng trong nhiều lĩnh vực khác nhau như: tâm lý học và các ngành khoa học xã hội, sinh học, thống kê, nhận diện mô hình, truy vấn thông tin, máy học và khai thác dữ liệu.

Gom nhóm văn bản là một ứng dụng của gom nhóm.
Trong đó, các văn bản gần tương đồng sẽ được gom chung với nhau thành nhóm riêng biệt theo cách thức không giám sát.
Các văn bản gần liên quan với nhau có độ tương đồng lớn hơn so với những văn bản dị biệt.
Như vậy, những văn bản tương đồng này có thể tạo thành nhóm có cùng một chủ để.
Các nhóm này được hình thành từ quá trình gom nhóm không giám sát nên giúp cho chúng ta thấy được cấu trúc của ngữ liệu.

Khi áp dụng gom nhóm vào văn bản, ta cần nghiên cứu tính khả dụng của vấn đề này.
Thông thường, các phương pháp áp dụng cho gom nhóm tập trung vào dữ liệu lớn có thuộc tính là số.
Trong khi đó, văn bản là thông tin có thể hiện là từ ngữ nên ta cần chuyển đổi thể hiện của văn bản sang dạng số để có thể áp dụng gom nhóm.
Gom nhóm văn bản cần có được tính khả dụng trong các nhiệm vụ sau:
\begin{enumerate}
\item[•]Duyệt và tổ chức văn bản: tổ chức phân cấp của văn bản vào trong các hạng mục hợp lý có thể giúp ích cho việc duyệt hệ thống của tập hợp văn bản. Ví dụ kinh kiển cho phương pháp này là phân tán hoặc tập hợp. Phương pháp này cung cấp kỹ thuật duyệt hệ thống với sử dụng gom nhóm tổ chức của tập hợp văn bản.
\item[•]Tóm tắt ngữ liệu: kỹ thuật gom nhóm cung cấp tóm tắt mạch lạc của tập hợp trong dạng nhóm tài tiệu hoặc nhóm từ. Điều này được sử dụng đề cung cấp tóm tắt trong phần nội dung tổng kết của ngữ liệu căn bản. Lĩnh vực này có nhiều phương pháp, đặc biệt là gom nhóm câu dùng để tóm tắt văn bản.
\item[•]Phân loại văn bản : Gom nhóm là phương pháp học không giám sát. Nó thể được đòn bẩy hóa để cải thiện chất lượng kết quả trong giám sát. Cụ thể, các nhóm từ và phương thức đồng huấn luyện có thể được sử dụng để cải thiện độ chính xác phân loại của ứng dụng giám sát với tác dụng của kỹ thuật gom nhóm.
\end{enumerate}

\section{Bài toán gom nhóm văn bản}
Trước khi tiến hành gom nhóm, ta cần phải hiểu rõ định nghĩa của nhóm là gì.
Vì trong nhiều ứng dụng, khái niệm về nhóm được định nghĩa không rõ ràng.
Cho nên, ta cần phải biết được một nhóm được tạo thành như thế nào trong quá trình gom nhóm.
Nhìn vào hình \ref{fig:pic11}, ta thấy có 20 điểm nhưng lại có đến 3 cách khác nhau để tạo thành nhóm.
Ta có thể dựa vào hình dạng của dữ liệu để thấy được mối quan hệ giữa các nhóm để tiến hành gom nhóm.
Dựa vào hình \ref{fig:pic11}(a) và hình \ref{fig:pic11}(b), ta có thể chia dữ liệu thành 2 nhóm khác nhau với mỗi nhóm gồm 6 điểm.
Tuy nhiên, khi quan sát kỹ hơn, ta có thể chia mỗi nhóm lớn này thành 2 nhóm nhỏ hơn như hình \ref{fig:pic11}(c).
Điều này mang tính chất chủ quan dựa vào cách nhìn của từng người mà ta có thể chia thành những nhóm con khác nhau.
Thậm chí, ta có thể tạo thành 6 nhóm khác nhau như hình \ref{fig:pic11}(d).
Vì vậy, định nghĩa về nhóm chỉ mang tính tương đối và cách tiếp cận tốt nhất để gom nhóm là ta phải dựa vào bản chất của dữ liệu và kết quả mong muốn.
Đối với gom nhóm văn bản, một nhóm được xem là gồm những văn bản có độ tương đồng gần giống nhau.

\begin{figure}[htp]
\makeatletter % For spaces in paths
\patchcmd\Gread@eps{\@inputcheck#1 }{\@inputcheck"#1"\relax}{}{}
\makeatother
\psscalebox{1.0 1.0} % Change this value to rescale the drawing.
{
\begin{pspicture}(0,-3.465664)(13.02909,3.465664)
\definecolor{colour0}{rgb}{0.0,0.0,0.4}
\definecolor{colour1}{rgb}{0.0,0.4,0.4}
\definecolor{colour2}{rgb}{0.4,0.0,0.4}
\definecolor{colour3}{rgb}{0.4,0.4,0.0}
\psdots[linecolor=black, dotsize=0.2](0.51454985,2.934336)
\psdots[linecolor=black, dotsize=0.2](0.9145499,3.334336)
\psdots[linecolor=black, dotsize=0.2](0.9145499,2.934336)
\psdots[linecolor=black, dotsize=0.2](0.9145499,2.934336)
\psdots[linecolor=black, dotsize=0.2](1.3145499,3.334336)
\psdots[linecolor=black, dotsize=0.2](1.7145499,2.934336)
\psdots[linecolor=black, dotsize=0.2](1.7145499,2.534336)
\psdots[linecolor=black, dotsize=0.2](2.1145499,2.934336)
\psdots[linecolor=black, dotsize=0.2](0.51454985,1.734336)
\psdots[linecolor=black, dotsize=0.2](0.51454985,2.134336)
\psdots[linecolor=black, dotsize=0.2](0.11454987,1.734336)
\psdots[linecolor=black, dotsize=0.2](4.5145497,3.334336)
\psdots[linecolor=black, dotsize=0.2](4.1145496,2.934336)
\psdots[linecolor=black, dotsize=0.2](4.5145497,2.934336)
\psdots[linecolor=black, dotsize=0.2](4.5145497,1.734336)
\psdots[linecolor=black, dotsize=0.2](4.91455,1.734336)
\psdots[linecolor=black, dotsize=0.2](4.1145496,1.3343359)
\psdots[linecolor=black, dotsize=0.2](4.5145497,1.3343359)
\psdots[linecolor=black, dotsize=0.2](5.31455,2.534336)
\psdots[linecolor=black, dotsize=0.2](5.31455,2.134336)
\psdots[linecolor=black, dotsize=0.2](5.71455,2.134336)
\rput[bl](1.3145499,0.534336){(a) Các điểm gốc}
\psdots[linecolor=colour0, dotstyle=square*, dotsize=0.2](7.71455,2.934336)
\psdots[linecolor=colour0, dotstyle=square*, dotsize=0.2](8.11455,3.334336)
\psdots[linecolor=colour0, dotstyle=square*, dotsize=0.2](8.11455,2.934336)
\psdots[linecolor=colour0, dotstyle=square*, dotsize=0.2](8.11455,2.934336)
\psdots[linecolor=colour0, dotstyle=square*, dotsize=0.2](8.51455,3.334336)
\psdots[linecolor=colour0, dotstyle=square*, dotsize=0.2](8.91455,2.934336)
\psdots[linecolor=colour0, dotstyle=square*, dotsize=0.2](8.91455,2.534336)
\psdots[linecolor=colour0, dotstyle=square*, dotsize=0.2](9.314549,2.934336)
\psdots[linecolor=colour0, dotstyle=square*, dotsize=0.2](7.71455,1.734336)
\psdots[linecolor=colour0, dotstyle=square*, dotsize=0.2](7.71455,2.134336)
\psdots[linecolor=colour0, dotstyle=square*, dotsize=0.2](7.31455,1.734336)
\psdots[linecolor=colour1, dotstyle=triangle*, dotsize=0.2](11.71455,3.334336)
\psdots[linecolor=colour1, dotstyle=triangle*, dotsize=0.2](11.314549,2.934336)
\psdots[linecolor=colour1, dotstyle=triangle*, dotsize=0.2](11.71455,2.934336)
\psdots[linecolor=colour1, dotstyle=triangle*, dotsize=0.2](11.71455,1.734336)
\psdots[linecolor=colour1, dotstyle=triangle*, dotsize=0.2](12.11455,1.734336)
\psdots[linecolor=colour1, dotstyle=triangle*, dotsize=0.2](11.314549,1.3343359)
\psdots[linecolor=colour1, dotstyle=triangle*, dotsize=0.2](11.71455,1.3343359)
\psdots[linecolor=colour1, dotstyle=triangle*, dotsize=0.2](12.51455,2.534336)
\psdots[linecolor=colour1, dotstyle=triangle*, dotsize=0.2](12.51455,2.134336)
\psdots[linecolor=colour1, dotstyle=triangle*, dotsize=0.2](12.91455,2.134336)
\rput[bl](8.51455,0.534336){(b) 2 nhóm}
\psdots[linecolor=colour0, dotstyle=+, dotsize=0.2](0.51454985,-1.065664)
\psdots[linecolor=colour0, dotstyle=+, dotsize=0.2](0.9145499,-0.665664)
\psdots[linecolor=colour0, dotstyle=+, dotsize=0.2](0.9145499,-1.065664)
\psdots[linecolor=colour0, dotstyle=+, dotsize=0.2](0.9145499,-1.065664)
\psdots[linecolor=colour0, dotstyle=+, dotsize=0.2](1.3145499,-0.665664)
\psdots[linecolor=colour0, dotstyle=+, dotsize=0.2](1.7145499,-1.065664)
\psdots[linecolor=colour0, dotstyle=+, dotsize=0.2](1.7145499,-1.465664)
\psdots[linecolor=colour0, dotstyle=+, dotsize=0.2](2.1145499,-1.065664)
\psdots[linecolor=colour1, dotstyle=triangle*, dotsize=0.2](0.51454985,-2.265664)
\psdots[linecolor=colour1, dotstyle=triangle*, dotsize=0.2](0.51454985,-1.865664)
\psdots[linecolor=colour1, dotstyle=triangle*, dotsize=0.2](0.11454987,-2.265664)
\psdots[linecolor=colour2, dotstyle=asterisk, dotsize=0.2](4.5145497,-0.665664)
\psdots[linecolor=colour2, dotstyle=asterisk, dotsize=0.2](4.1145496,-1.065664)
\psdots[linecolor=colour2, dotstyle=asterisk, dotsize=0.2](4.5145497,-1.065664)
\psdots[linecolor=colour3, dotstyle=diamond*, dotsize=0.2](4.5145497,-2.265664)
\psdots[linecolor=colour3, dotstyle=diamond*, dotsize=0.2](4.91455,-2.265664)
\psdots[linecolor=colour3, dotstyle=diamond*, dotsize=0.2](4.1145496,-2.665664)
\psdots[linecolor=colour3, dotstyle=diamond*, dotsize=0.2](4.5145497,-2.665664)
\psdots[linecolor=colour3, dotstyle=diamond*, dotsize=0.2](5.31455,-1.465664)
\psdots[linecolor=colour3, dotstyle=diamond*, dotsize=0.2](5.31455,-1.865664)
\psdots[linecolor=colour3, dotstyle=diamond*, dotsize=0.2](5.71455,-1.865664)
\rput[bl](1.3145499,-3.465664){(c) 4 nhóm}
\psdots[linecolor=colour0, dotstyle=oplus, dotsize=0.2](7.71455,-1.065664)
\psdots[linecolor=colour0, dotstyle=oplus, dotsize=0.2](8.11455,-0.665664)
\psdots[linecolor=colour0, dotstyle=oplus, dotsize=0.2](8.11455,-1.065664)
\psdots[linecolor=colour0, dotstyle=oplus, dotsize=0.2](8.11455,-1.065664)
\psdots[linecolor=colour0, dotstyle=oplus, dotsize=0.2](8.51455,-0.665664)
\psdots[linecolor=black, dotsize=0.2](8.91455,-1.065664)
\psdots[linecolor=black, dotsize=0.2](8.91455,-1.465664)
\psdots[linecolor=black, dotsize=0.2](9.314549,-1.065664)
\psdots[linecolor=brown, dotstyle=triangle*, dotsize=0.2](7.71455,-2.265664)
\psdots[linecolor=brown, dotstyle=triangle*, dotsize=0.2](7.71455,-1.865664)
\psdots[linecolor=brown, dotstyle=triangle*, dotsize=0.2](7.31455,-2.265664)
\psdots[linecolor=colour3, dotstyle=asterisk, dotsize=0.2](11.71455,-0.665664)
\psdots[linecolor=colour3, dotstyle=asterisk, dotsize=0.2](11.314549,-1.065664)
\psdots[linecolor=colour3, dotstyle=asterisk, dotsize=0.2](11.71455,-1.065664)
\psdots[linecolor=colour2, dotstyle=diamond*, dotsize=0.2](11.71455,-2.265664)
\psdots[linecolor=colour2, dotstyle=diamond*, dotsize=0.2](12.11455,-2.265664)
\psdots[linecolor=colour2, dotstyle=diamond*, dotsize=0.2](11.314549,-2.665664)
\psdots[linecolor=colour2, dotstyle=diamond*, dotsize=0.2](11.71455,-2.665664)
\psdots[linecolor=colour1, dotstyle=square*, dotsize=0.2](12.51455,-1.465664)
\psdots[linecolor=colour1, dotstyle=square*, dotsize=0.2](12.51455,-1.865664)
\psdots[linecolor=colour1, dotstyle=square*, dotsize=0.2](12.91455,-1.865664)
\rput[bl](8.51455,-3.465664){(d) 6 nhóm}
\end{pspicture}
}
\caption{Những cách khác nhau về gom nhóm trên cùng một tập điểm}
\label{fig:pic11}
\end{figure}

Gom nhóm văn bản có thể liên quan đến các kỹ thuật để phân chia văn bản vào các nhóm khác nhau.
Gom nhóm văn bản có thể tự động tạo ra nhãn đại diện cho nhóm trong quá trình gom nhóm.
Các nhãn này được trích xuất từ văn bản trong quá trình tạo thành nhóm.
Trong khi đó, phân lớp văn bản lại là phương pháp có giám sát, những văn bản chưa có nhãn sẽ được gán với nhãn đã được cho trước.
Những văn bản có cùng nhãn sẽ được phân vào thành một lớp.
Điều này khác hoàn toàn với gom nhóm văn bản khi mà quá trình gom nhóm có thể trích xuất ra nhãn cho nhóm.
Vì vậy, gom nhóm văn bản còn được gọi là phân lớp văn bản 
không có giám sát.

Có thể nói, bài toán gom nhóm văn bản là bài toán dùng để giải quyết việc tìm kiếm số lượng nhóm có thể có trong ngữ liệu.
Quá trình gom nhóm văn bản có thể được thực thi từ các phương pháp khác nhau nhưng đều có đặc điểm chung là phương pháp không có giám sát.
Động lực để thực hiện gom nhóm văn bản là quá trình tự động gom các văn bản thành các nhóm khác nhau dựa vào thông tin được tìm thấy trong dữ liệu mô tả của văn bản và quan hệ giữa các văn bản.
Mục tiêu của gom nhóm văn bản là những văn bản gần giống nhau (có độ tương đồng gần nhau) sẽ ở trong cùng một nhóm so với những văn bản dị biệt được gom trong nhóm khác.
Độ tương đồng giữa các văn bản trong cùng một nhóm quá lớn và độ khác biệt giữa các nhóm cũng quá lớn có thể dẫn đến khoảng cách để kết nối giữa các nhóm là rất lớn.
Nếu ta gom nhóm văn bản một cách hiệu quả thì có thể giúp ích cho việc tìm kiếm kết quả trong truy vấn thông tin.

\section{Tầm quan trọng gom nhóm văn bản}
Ngày nay, việc tìm kiếm thông tin đã trở thành kỹ năng cần thiết cho bất kì ai.
Cho nên, truy vấn thông tin trở thành nhu cầu quan trọng của người dùng.
Tuy nhiên, khi truy vấn thông tin, nếu dữ liệu được truy vấn không được tổ chức khoa học thì ta có thể mất khá nhiều thời gian để có kết quả của câu truy vấn hoặc thông tin ta nhận được lại có thể không đầy đủ.
Vì vậy, ta cần tổ chức lại dữ liệu thành cấu trúc phù hợp để hỗ trợ truy vấn thông tin.
Và gom nhóm văn bản là một phương pháp để giúp cho việc sắp xếp thông tin từ các văn bản thành những chủ để phù hợp.
Từ đó, việc truy vấn thông tin trở nên dễ dàng hơn.

Gom nhóm văn bản và truy vấn thông tin có quan hệ mật thiết với nhau.
Truy vấn thông tin cần tìm kiếm những văn bản một cách nhanh chóng, chính xác và đầy đủ.
Gom nhóm văn bản tự động tạo ra những nhóm có văn bản liên quan thành cùng một chủ đề.
Từ đó, khi ta truy vấn sử dụng từ khóa thì sẽ những văn bản liên quan đến từ khóa cần truy vấn sẽ trở thành kết quả trả về.
Ngoài ra, gom nhóm văn bản không cần phải tạo trước tập huấn luyện hay nguyên tắc phân loại.
Có thể nói ~\cite{text-clustering}, gom nhóm văn bản được dùng để cải thiện hiệu quả tìm kiếm trong truy vấn thông tin.
\begin{enumerate}
\item[•]Cải thiện hồi quy trong tìm kiếm: khi câu truy vấn trùng với một văn bản thì kết quả trả về có thể bao gồm toàn bộ nhóm chứa văn bản được tìm kiếm.
\item[•]Cải thiện độ chính xác trong tìm kiếm: gom nhóm những văn bản vào thành những nhóm nhỏ hơn trong nhóm chứa các văn bản liên quan.
\item[•]Phân tán hoặc tập hợp: khi một câu truy vấn không thể được công thức hóa thì ta có thể cho phép người dùng duyệt văn bản theo nhóm.
\item[•]Gom nhóm theo truy vấn đặc tả: những văn bản gần liên quan đến câu truy vấn nhất sẽ nằm trong nhóm con (có độ tương đồng cao) bên trong nhóm lớn (độ tương đồng thấp hơn).
\end{enumerate}

%Nghiên cứu của vấn đề gom nhóm đứng trước tính khả dụng khi ứng dụng vào văn bản. Các phương pháp truyền thống cho gom nhóm thường tập trung vào dữ liệu lớn, khi thuộc tính của dữ liệu là số. Vấn đề này cũng được nghiên cứu trong phân loại dữ liệu, khi mà thuộc tính có giá trị nặc danh. Vấn đề của gom nhóm là tìm được tính khả dụng trong các nhiệm vụ sau:
%\begin{enumerate}
%\item[•]Duyệt và tổ chức văn bản: tổ chức phân cấp của văn bản vào trong các hạng mục mạch lạc. Điều này có thể giúp ích cho việc duyệt hệ thống của tập hợp văn bản. Ví dụ kinh kiển cho phương pháp này là phân tán hoặc tập hợp. Phương pháp này cung cấp kỹ thuật duyệt hệ thống với sử dụng gom nhóm tổ chức của tập hợp văn bản.
%\item[•]Tóm tắt ngữ liệu: kỹ thuật gom nhóm cung cấp tóm tắt mạch lạc của tập hợp trong dạng nhóm tài tiệu hoặc nhóm từ. Điều này được sử dụng đề cung cấp tóm tắt trong phần nội dung tổng kết của ngữ liệu căn bản. Lĩnh vực này có nhiều phương pháp, đặc biệt là gom nhóm câu dùng để tóm tắt văn bản. Vấn đề của gom nhóm liên quan đến việc giảm số chiều và mô hình hóa chủ đề. 
%\item[•]Phân loại văn bản : Gom nhóm là phương pháp học không giám sát. Nó thể được đòn bẩy hóa để cải thiện chất lượng kết quả trong giám sát. Cụ thể, các nhóm từ và phương thức đồng huấn luyện có thể được sử dụng để cải thiện độ chính xác phân loại của ứng dụng giám sát với tác dụng của kỹ thuật gom nhóm.
%\end{enumerate}

\section{Hình dạng gom nhóm}
\subsection{Phân cấp và phân chia}
Vấn đề thường được thảo luận về điểm khác biệt về hình dạng của dữ liệu là các nhóm sau khi được gom có lồng vào nhau hay là tách biệt thành từng nhóm riêng biệt.
Hình dạng các nhóm lồng vào nhau còn có tên gọi phân cấp và những nhóm tách biệt thì có tên gọi là phân chia.
Các nhóm được gom theo phân chia khi tập dữ liệu sẽ chia các đối tượng vào thành từng nhóm riêng biệt.
Nhìn vào hình \ref{fig:pic11} từ (b) đến (d) cho chúng ta thấy được hình dạng của dữ liệu bao gồm các nhóm riêng biệt sau khi phân chia các đối tượng vào những nhóm này.

Nếu ta có thể chia tách những nhóm hiện hữu thành các nhóm con thì ta sẽ có được hình dạng phân cấp, gồm các nhóm lồng vào nhau hình thành nên cấu trúc cây phân cấp.
Mỗi một nốt (nhóm) trong cây có thể là tập hợp của những nốt con (nhóm con) và nốt gốc trong cây là nhóm chứa toàn bộ các nhóm còn lại.
Ngoài ra, nốt trong cây cũng có thể là nốt lá, đơn vị chứa dữ liệu.
Như vậy, hình dạng phân cấp của các nhóm cho ta thấy được mối quan hệ giữa các nhóm trong dữ liệu.

Áp dụng ý tưởng trên vào hình \ref{fig:pic11}(a), ta thấy được đây là nốt gốc có chứa 2 nhóm con.
Tiếp đến hình \ref{fig:pic11}(b), ứng với mỗi nhóm, ta có thể lần lượt chia thành 3 nhóm con (kết quả như hình \ref{fig:pic11}(d)).
Các nhóm trong hình \ref{fig:pic11} từ (a) đến (d), khi thực thi theo thứ tự có thể tạo thành nhóm phân cấp tương ứng 1, 2, 4, và 6 nhóm ứng với mỗi mức độ.
Cuối cùng, ta có thể xem hình dạng phân cấp của nhóm là là chuỗi liên tiếp của hình dạng phân chia của nhóm.
Nghĩa là ta có thể lấy được kết quả của hình dạng phân chia của nhóm bằng việc cắt tại một mức độ nhất định trong hình dạng cây phân cấp của nhóm.

\subsection{Duy nhất, chồng đè và mờ}
Hình dạng các nhóm trong hình \ref{fig:pic11} đều được xem là duy nhất khi mà mỗi đối tượng được gán vào một nhóm duy nhất.
Thực tế, ta có rất nhiều trường hợp mà một đối tượng có thể được gắn với nhiều nhóm khác nhau.
Với những trường hợp mà một đối tượng có thể nằm trong nhóm này hoặc nhóm khác thì ta sẽ có các nhóm không duy nhất hay còn gọi là chồng đè.
%Trong trường hợp tổng quát, hướng tiếp cận chồng đè hay không duy nhất dùng để phản ánh một đối tượng có thể được gán đồng thời ở nhóm này cũng như là nhóm khác.
%Hướng tiếp cận duy nhất cho ta thấy được các nhóm độc lập, riêng lẻ với nhau còn hướng tiếp cận không duy nhất hay còn gọi là chồng đè cho ta thấy được quan hệ giữa các nhóm.
%Ví dụ, một người trong công ty có thể kiêm nhiều chức vụ khác nhau.
%Hướng tiếp cận không duy nhất thường được sử dụng khi mà một đối tượng ở giữa một hoặc nhiều nhóm khác nhau.
%Đồng thời, đối tượng này có thể được gán vào một trong những nhóm này mà vẫn đảm bảo được tính hợp lý.
Ngoài ra, ta còn có thêm hình dạng của các nhóm mờ với mỗi đối tượng thuộc về từng nhóm ứng với trọng số có giá trị trong khoảng từ 0 (không thuộc về) đến 1 (hoàn toàn thuộc về).
Nói cách khác, các nhóm này được xem như là tập hợp mờ.
Tập hợp mờ là tập hợp mà các phần tử ở trong bất kì tập nào đều có trọng số ứng với tập đó và có giá trị nằm trong khoảng từ 0 đến 1.
Tuy nhiên, tổng của tất cả trọng số của mỗi phần tử đều phải bằng 1.

Gom nhóm mờ rất giống với gom nhóm xác suất khi trọng số của mỗi đối tượng ứng với mỗi nhóm xem như xác suất của đối tượng thuộc về nhóm đó và tổng của xác suất này phải bằng 1.
Vì tổng của trọng số có giá trị là 1 và ứng với mỗi nhóm thì đối tượng có trọng số khác nhau nên gom nhóm mờ không thể chỉ ra được tình huống mà một đối tượng có thể thuộc về nhiều nhóm.
Thay vào đó, gom nhóm mờ thường thích hợp cho việc tránh ngẫu nhiên gán một đối tượng vào một nhóm duy nhất khi mà nó có thể thích hợp cho nhiều nhóm khác nhau.
Trong thực tế, gom nhóm mờ thường được chuyển đổi thành gom nhóm duy nhất bằng việc gán đối tượng vào nhóm mà có trọng số cao nhất.

\subsection{Toàn phần và cục bộ}
Gom nhóm toàn phần là gán mỗi đối tượng vào một nhóm cụ thể, trong khi gom nhóm cục bộ thì lại khác.
Trong gom nhóm cục bộ, một vài đối tượng có thể không được gán vào bất kì nhóm nào.
Điều này xảy ra vì độ tương đồng giữa các đối tượng này so với các nhóm hiện tại là không cao nên các đối tượng này sẽ được loại bỏ.
Và sự tồn tại của các đối tượng này có thể gây ra độ nhiễu trong hình dạng của gom nhóm.

\section{Các loại gom nhóm}
\label{sec:clgn}
Gom nhóm dùng để tìm được các nhóm đối tượng hữu dụng, thế nào được xem là hữu dụng tùy vào mục đích gom nhóm.
Ta có rất nhiều cách cũng như là ý tưởng khác nhau đê thực hiện gom nhóm.
Để có cái nhìn tổng quát về các loại gom nhóm khác nhau, ta sẽ sử dụng dữ liệu là điểm có hai chiều cho các ví dụ được thảo luận sau đây.
Tuy nhiên, những phương pháp được thảo luận dưới đây đều có thể áp dụng cho bất kì dạng dữ liệu dạng nào.

\subsection{Phân tách nhiều}
Một nhóm là tập hợp của các đối tượng mà mỗi đối tượng có độ tương đồng gần nhau thì ở trong cùng nhóm.
Đôi khi ta sẽ đặt ngưỡng để đảm bảo rằng các đối tượng ở trong cùng nhóm phải có độ tương đồng bằng hoặc trên ngưỡng đó để nhóm được hình thành ít nhiễu nhất có thể.
Tuy nhiên, ý tưởng này chỉ khả thi khi mà hình dạng của các nhóm có khoảng cách xa nhau như thể hiện trong hình \ref{fig:pic12}.
Dựa vào hình \ref{fig:pic12}, ta có thể thấy gom nhóm phân tách nhiều đã chia các điểm nằm ở hai nhóm tách biệt nhau.
Khoảng cách giữa hai điểm bất kì của hai nhóm khác nhau luôn lớn hơn so với khoảng cách giữa hai điểm trong cùng một nhóm.
Gom nhóm phân tách nhiều không nhất thiết phải là hình cầu mà có thể là bất kì hình dạng nào.

\begin{figure}[htp]
\makeatletter % For spaces in paths
\patchcmd\Gread@eps{\@inputcheck#1 }{\@inputcheck"#1"\relax}{}{}
\makeatother
\psscalebox{1.0 1.0} % Change this value to rescale the drawing.
{
\begin{pspicture}(0,-1.3333334)(11.466666,1.3333334)
\pscircle[linecolor=black, linewidth=0.04, dimen=outer](1.3333334,0.0){1.3333334}
\pscircle[linecolor=black, linewidth=0.04, dimen=outer](10.133333,0.0){1.3333334}
\end{pspicture}
}
\caption{Các nhóm được gom theo phân tách nhiều}
\label{fig:pic12}
\end{figure}

\subsection{Dựa vào mẫu}
Một nhóm là một tập hợp các đối tượng mà mỗi đối tượng trong nhóm gần tương đồng với mẫu.
Điều này có nghĩa là mỗi nhóm sẽ được đại diện bởi một mẫu, các đối tượng gần với mẫu nào sẽ nằm trong nhóm có mẫu đó.
Đối với dữ liệu có thuộc tính liên tục, mẫu của một nhóm thường là trung điểm (nghĩa là điểm trung bình của tất cả các điểm trong nhóm đó).
Tuy nhiên, khi dữ liệu tồn tại thuộc tính phân lớp thì sử dụng trung điểm sẽ không hiệu quả.
Thay vào đó, ta sẽ sử dụng mẫu là điểm tiêu biểu (điểm đại diện cho toàn bộ nhóm).
Đối với nhiều loại dữ liệu, mẫu thường được xem như là trung điểm và trong những trường hợp như vậy, ta thường gọi gom nhóm dựa vào mẫu là gom nhóm dựa vào trung điểm.
Hình dạng thường thấy của dạng gom nhóm này là hình cầu, hình \ref{fig:pic13} cho ta thấy điều đó.

\begin{figure}[htp]
\makeatletter % For spaces in paths
\patchcmd\Gread@eps{\@inputcheck#1 }{\@inputcheck"#1"\relax}{}{}
\makeatother
\psscalebox{1.0 1.0} % Change this value to rescale the drawing.
{
\begin{pspicture}(0,-1.2235354)(4.819711,1.2235354)
\pscircle[linecolor=black, linewidth=0.04, dimen=outer](1.2197113,0.023535462){1.2}
\psdots[linecolor=black, dotsize=0.04](0.4197113,0.82353544)
\psdots[linecolor=black, dotsize=0.04](0.4197113,0.42353547)
\psdots[linecolor=black, dotsize=0.04](0.4197113,0.023535462)
\psdots[linecolor=black, dotsize=0.04](0.4197113,-0.37646455)
\psdots[linecolor=black, dotsize=0.04](0.4197113,-0.37646455)
\psdots[linecolor=black, dotsize=0.04](0.4197113,0.023535462)
\psdots[linecolor=black, dotsize=0.04](0.4197113,0.42353547)
\psdots[linecolor=black, dotsize=0.04](0.4197113,0.42353547)
\psdots[linecolor=black, dotsize=0.04](0.4197113,0.023535462)
\psdots[linecolor=black, dotsize=0.04](0.019711304,0.023535462)
\psdots[linecolor=black, dotsize=0.04](0.4197113,0.42353547)
\psdots[linecolor=black, dotsize=0.04](0.4197113,0.023535462)
\psdots[linecolor=black, dotsize=0.04](0.019711304,0.023535462)
\psdots[linecolor=black, dotsize=0.04](0.4197113,0.42353547)
\psdots[linecolor=black, dotsize=0.04](0.4197113,0.023535462)
\psdots[linecolor=black, dotsize=0.04](1.2197113,0.42353547)
\psdots[linecolor=black, dotsize=0.04](1.2197113,0.42353547)
\psdots[linecolor=black, dotsize=0.04](0.4197113,-0.37646455)
\psdots[linecolor=black, dotsize=0.04](0.8197113,0.023535462)
\psdots[linecolor=black, dotsize=0.04](0.8197113,0.42353547)
\psdots[linecolor=black, dotsize=0.04](1.2197113,0.42353547)
\psdots[linecolor=black, dotsize=0.04](1.6197113,0.023535462)
\psdots[linecolor=black, dotsize=0.04](1.2197113,-0.7764645)
\psdots[linecolor=black, dotsize=0.04](0.8197113,-0.37646455)
\psdots[linecolor=black, dotsize=0.04](0.8197113,-0.37646455)
\psdots[linecolor=black, dotsize=0.04](1.6197113,0.023535462)
\psdots[linecolor=black, dotsize=0.04](0.8197113,-0.37646455)
\psdots[linecolor=black, dotsize=0.04](0.8197113,-0.7764645)
\psdots[linecolor=black, dotsize=0.04](1.6197113,-0.7764645)
\psdots[linecolor=black, dotsize=0.04](1.2197113,-0.7764645)
\psdots[linecolor=black, dotsize=0.04](1.2197113,0.023535462)
\psdots[linecolor=black, dotsize=0.04](1.2197113,-0.37646455)
\psdots[linecolor=black, dotsize=0.04](2.0197113,-0.37646455)
\psdots[linecolor=black, dotsize=0.04](2.0197113,-0.37646455)
\psdots[linecolor=black, dotsize=0.04](2.0197113,0.023535462)
\psdots[linecolor=black, dotsize=0.04](2.0197113,0.42353547)
\psdots[linecolor=black, dotsize=0.04](2.0197113,0.42353547)
\psdots[linecolor=black, dotsize=0.04](1.6197113,0.42353547)
\psdots[linecolor=black, dotsize=0.04](1.2197113,-0.37646455)
\psdots[linecolor=black, dotsize=0.04](1.6197113,0.42353547)
\psdots[linecolor=black, dotsize=0.04](1.2197113,0.82353544)
\psdots[linecolor=black, dotsize=0.04](0.8197113,0.82353544)
\psdots[linecolor=black, dotsize=0.04](0.8197113,0.82353544)
\psdots[linecolor=black, dotsize=0.04](0.4197113,0.82353544)
\psdots[linecolor=black, dotsize=0.04](1.2197113,0.82353544)
\psdots[linecolor=black, dotsize=0.04](1.6197113,0.82353544)
\psdots[linecolor=black, dotsize=0.04](2.0197113,0.82353544)
\psdots[linecolor=black, dotsize=0.04](2.0197113,-0.37646455)
\psdots[linecolor=black, dotsize=0.04](2.0197113,-0.7764645)
\psdots[linecolor=black, dotsize=0.04](1.2197113,-0.37646455)
\psdots[linecolor=black, dotsize=0.04](1.6197113,-0.37646455)
\psdots[linecolor=black, dotsize=0.04](1.2197113,-0.7764645)
\psdots[linecolor=black, dotsize=0.04](1.2197113,-1.1764646)
\psdots[linecolor=black, dotsize=0.04](1.2197113,-1.1764646)
\psdots[linecolor=black, dotsize=0.04](1.2197113,0.023535462)
\psdots[linecolor=black, dotsize=0.04](0.8197113,0.023535462)
\psdots[linecolor=black, dotsize=0.04](0.8197113,-0.7764645)
\psdots[linecolor=black, dotsize=0.04](0.4197113,-0.7764645)
\psdots[linecolor=black, dotsize=0.04](0.8197113,0.023535462)
\psdots[linecolor=black, dotsize=0.04](0.8197113,0.82353544)
\psdots[linecolor=black, dotsize=0.04](1.6197113,0.82353544)
\pscircle[linecolor=black, linewidth=0.04, dimen=outer](3.6197114,0.023535462){1.2}
\psdots[linecolor=black, dotstyle=x, dotsize=0.1](2.8197112,0.82353544)
\psdots[linecolor=black, dotstyle=x, dotsize=0.1](2.8197112,0.42353547)
\psdots[linecolor=black, dotstyle=x, dotsize=0.1](2.8197112,0.023535462)
\psdots[linecolor=black, dotstyle=x, dotsize=0.1](2.8197112,-0.37646455)
\psdots[linecolor=black, dotstyle=x, dotsize=0.1](2.8197112,-0.37646455)
\psdots[linecolor=black, dotstyle=x, dotsize=0.1](2.8197112,0.023535462)
\psdots[linecolor=black, dotstyle=x, dotsize=0.1](2.8197112,0.42353547)
\psdots[linecolor=black, dotstyle=x, dotsize=0.1](2.8197112,0.42353547)
\psdots[linecolor=black, dotstyle=x, dotsize=0.1](2.8197112,0.023535462)
\psdots[linecolor=black, dotstyle=x, dotsize=0.1](2.4197114,0.023535462)
\psdots[linecolor=black, dotstyle=x, dotsize=0.1](2.8197112,0.42353547)
\psdots[linecolor=black, dotstyle=x, dotsize=0.1](2.8197112,0.023535462)
\psdots[linecolor=black, dotstyle=x, dotsize=0.1](2.4197114,0.023535462)
\psdots[linecolor=black, dotstyle=x, dotsize=0.1](2.8197112,0.42353547)
\psdots[linecolor=black, dotstyle=x, dotsize=0.1](2.8197112,0.023535462)
\psdots[linecolor=black, dotstyle=x, dotsize=0.1](3.6197114,0.42353547)
\psdots[linecolor=black, dotstyle=x, dotsize=0.1](3.6197114,0.42353547)
\psdots[linecolor=black, dotstyle=x, dotsize=0.1](2.8197112,-0.37646455)
\psdots[linecolor=black, dotstyle=x, dotsize=0.1](3.2197113,0.023535462)
\psdots[linecolor=black, dotstyle=x, dotsize=0.1](3.2197113,0.42353547)
\psdots[linecolor=black, dotstyle=x, dotsize=0.1](3.6197114,0.42353547)
\psdots[linecolor=black, dotstyle=x, dotsize=0.1](4.0197115,0.023535462)
\psdots[linecolor=black, dotstyle=x, dotsize=0.1](3.6197114,-0.7764645)
\psdots[linecolor=black, dotstyle=x, dotsize=0.1](3.2197113,-0.37646455)
\psdots[linecolor=black, dotstyle=x, dotsize=0.1](3.2197113,-0.37646455)
\psdots[linecolor=black, dotstyle=x, dotsize=0.1](4.0197115,0.023535462)
\psdots[linecolor=black, dotstyle=x, dotsize=0.1](3.2197113,-0.37646455)
\psdots[linecolor=black, dotstyle=x, dotsize=0.1](3.2197113,-0.7764645)
\psdots[linecolor=black, dotstyle=x, dotsize=0.1](4.0197115,-0.7764645)
\psdots[linecolor=black, dotstyle=x, dotsize=0.1](3.6197114,-0.7764645)
\psdots[linecolor=black, dotstyle=x, dotsize=0.1](3.6197114,0.023535462)
\psdots[linecolor=black, dotstyle=x, dotsize=0.1](3.6197114,-0.37646455)
\psdots[linecolor=black, dotstyle=x, dotsize=0.1](4.419711,-0.37646455)
\psdots[linecolor=black, dotstyle=x, dotsize=0.1](4.419711,-0.37646455)
\psdots[linecolor=black, dotstyle=x, dotsize=0.1](4.419711,0.023535462)
\psdots[linecolor=black, dotstyle=x, dotsize=0.1](4.419711,0.42353547)
\psdots[linecolor=black, dotstyle=x, dotsize=0.1](4.419711,0.42353547)
\psdots[linecolor=black, dotstyle=x, dotsize=0.1](4.0197115,0.42353547)
\psdots[linecolor=black, dotstyle=x, dotsize=0.1](3.6197114,-0.37646455)
\psdots[linecolor=black, dotstyle=x, dotsize=0.1](4.0197115,0.42353547)
\psdots[linecolor=black, dotstyle=x, dotsize=0.1](3.6197114,0.82353544)
\psdots[linecolor=black, dotstyle=x, dotsize=0.1](3.2197113,0.82353544)
\psdots[linecolor=black, dotstyle=x, dotsize=0.1](3.2197113,0.82353544)
\psdots[linecolor=black, dotstyle=x, dotsize=0.1](2.8197112,0.82353544)
\psdots[linecolor=black, dotstyle=x, dotsize=0.1](3.6197114,0.82353544)
\psdots[linecolor=black, dotstyle=x, dotsize=0.1](4.0197115,0.82353544)
\psdots[linecolor=black, dotstyle=x, dotsize=0.1](4.419711,0.82353544)
\psdots[linecolor=black, dotstyle=x, dotsize=0.1](4.419711,-0.37646455)
\psdots[linecolor=black, dotstyle=x, dotsize=0.1](4.419711,-0.7764645)
\psdots[linecolor=black, dotstyle=x, dotsize=0.1](3.6197114,-0.37646455)
\psdots[linecolor=black, dotstyle=x, dotsize=0.1](4.0197115,-0.37646455)
\psdots[linecolor=black, dotstyle=x, dotsize=0.1](3.6197114,-0.7764645)
\psdots[linecolor=black, dotstyle=x, dotsize=0.1](3.6197114,-1.1764646)
\psdots[linecolor=black, dotstyle=x, dotsize=0.1](3.6197114,-1.1764646)
\psdots[linecolor=black, dotstyle=x, dotsize=0.1](3.6197114,0.023535462)
\psdots[linecolor=black, dotstyle=x, dotsize=0.1](3.2197113,0.023535462)
\psdots[linecolor=black, dotstyle=x, dotsize=0.1](3.2197113,-0.7764645)
\psdots[linecolor=black, dotstyle=x, dotsize=0.1](2.8197112,-0.7764645)
\psdots[linecolor=black, dotstyle=x, dotsize=0.1](3.2197113,0.023535462)
\psdots[linecolor=black, dotstyle=x, dotsize=0.1](3.2197113,0.82353544)
\psdots[linecolor=black, dotstyle=x, dotsize=0.1](4.0197115,0.82353544)
\end{pspicture}
}
\caption{Các nhóm được gom dựa vào trung điểm}
\label{fig:pic13}
\end{figure}

\subsection{Dựa vào đồ thị}
Nếu như dữ liệu được biểu diễn như là đồ thị, với nốt là các đối tượng và các liên kết thể hiện phần kết nối giữa các đối tượng thì nhóm được định nghĩa như là một thành phần liên kết, nghĩa là một nhóm bao gồm các đối tượng liên kết với nhau nhưng mà không tồn tại liên kết với các đối tượng ở ngoài nhóm.
Một ví dụ quan trọng của gom nhóm dựa vào đồ thị là gom nhóm dựa vào tính liền kề, hai đối tượng được liên kết khi và chỉ khi hai đối tượng này nằm trong khoảng cách đặc tả của nhau.
Điều này có nghĩa là các đối tượng ở trong cùng nhóm luôn ở gần với nhau hơn so với các đối tượng ở trong nhóm khác.
Hình \ref{fig:pic14} chỉ ra ví dụ về gom nhóm dựa vào liền kề.

Cách sử dụng gom nhóm này chỉ hữu dụng khi hình dạng của các nhóm này dị biệt hoặc là quấn vào nhau.
Tuy nhiên, một vấn đề cần chú ý là cách gom nhóm này có thể xuất hiện nhiễu, như trong hình \ref{fig:pic14}, một đoạn thẳng xuất hiện tạo thành cầu nối giữa hai nhóm riêng biệt.
Những loại gom nhóm khác dựa vào đồ thị cũng có thể gặp trường hợp tương tự.
Một trong những hướng tiếp cận này là clique, một tập hợp các nốt trong đồ thị mà kết nối hoàn toàn với nhau.
Cũng như gom nhóm dựa vào mẫu, các nhóm được gom dựa vào phương pháp này thường có xu hướng tạo nên hình cầu.

\begin{figure}[htp]
\makeatletter % For spaces in paths
\patchcmd\Gread@eps{\@inputcheck#1 }{\@inputcheck"#1"\relax}{}{}
\makeatother
\psscalebox{1.0 1.0} % Change this value to rescale the drawing.
{
\begin{pspicture}(0,-2.3036544)(10.806001,2.3036544)
\definecolor{colour0}{rgb}{0.8,0.8,0.8}
\psline[linecolor=black, linewidth=0.04, linestyle=dotted, dotsep=0.10583334cm](1.2060003,1.0036933)(0.006000366,0.20369339)(1.6060004,-0.19630662)(0.006000366,-0.9963066)(0.006000366,-0.9963066)
\rput{-269.66956}(4.83741,-4.8022056){\psarc[linecolor=colour0, linewidth=0.6, dimen=outer](4.806,0.00369339){2.0}{0.0}{180.0}}
\pscircle[linecolor=black, linewidth=0.04, dimen=outer](4.806,0.20369339){1.2}
\pscircle[linecolor=black, linewidth=0.04, dimen=outer](9.606,0.20369339){1.2}
\psline[linecolor=black, linewidth=0.04, linestyle=dotted, dotsep=0.10583334cm](6.0060005,0.20369339)(8.406,0.20369339)(8.406,0.20369339)
\end{pspicture}
}
\caption{Các nhóm được gom dựa vào liền kề}
\label{fig:pic14}
\end{figure}


\subsection{Dựa vào mật độ}
Nhóm được định nghĩa là vùng mật độ của các đối tượng mà bao quanh nó là vùng có mật độ thấp.
Hình \ref{fig:pic15} cho ta thấy gom nhóm dựa vào mật độ với dữ liệu được tạo bằng cách thêm độ nhiễu cho dữ liệu từ hình \ref{fig:pic14}.
Hai hình tròn đã không còn đường kết nối như hình \ref{fig:pic14}, bởi vì đường kết nối giữa hai nhóm bị che mờ trong phần nhiễu.
Tương tự, đoạn đường cong thể hiện trong hình \ref{fig:pic14} cũng bị che mờ trong phần nhiễu và không tạo thành nhóm trong hình \ref{fig:pic15}.

\begin{figure}[htp]
\makeatletter % For spaces in paths
\patchcmd\Gread@eps{\@inputcheck#1 }{\@inputcheck"#1"\relax}{}{}
\makeatother
\psscalebox{1.0 1.0} % Change this value to rescale the drawing.
{
\begin{pspicture}(0,-2.967742)(11.383526,2.967742)
\definecolor{colour0}{rgb}{0.8,0.8,0.8}
\rput{-269.66956}(3.5690148,-3.380749){\psarc[linecolor=colour0, linewidth=0.6, dimen=outer](3.4651613,0.08387104){2.0}{0.0}{180.0}}
\pscircle[linecolor=black, linewidth=0.04, dimen=outer](3.4651613,0.28387105){1.2}
\pscircle[linecolor=black, linewidth=0.04, dimen=outer](8.2651615,0.28387105){1.2}
\psframe[linecolor=black, linewidth=0.04, linestyle=dotted, dotsep=0.10583334cm, dimen=outer](11.374839,2.967742)(0.020000149,-2.967742)
\psline[linecolor=black, linewidth=0.04, linestyle=dotted, dotsep=0.10583334cm](0.020000149,2.7096775)(11.374839,2.7096775)(11.374839,2.7096775)
\psline[linecolor=black, linewidth=0.04, linestyle=dotted, dotsep=0.10583334cm](0.020000149,2.451613)(11.374839,2.451613)(11.374839,2.451613)
\psline[linecolor=black, linewidth=0.04, linestyle=dotted, dotsep=0.10583334cm](0.020000149,2.1935484)(2.6006453,2.1935484)
\psline[linecolor=black, linewidth=0.04, linestyle=dotted, dotsep=0.10583334cm](3.6329033,2.1935484)(11.374839,2.1935484)
\psline[linecolor=black, linewidth=0.04, linestyle=dotted, dotsep=0.10583334cm](0.020000149,1.9354839)(2.0845163,1.9354839)(2.0845163,1.9354839)
\psline[linecolor=black, linewidth=0.04, linestyle=dotted, dotsep=0.10583334cm](3.6329033,1.9354839)(11.374839,1.9354839)
\psline[linecolor=black, linewidth=0.04, linestyle=dotted, dotsep=0.10583334cm](0.020000149,1.6774194)(1.8264518,1.6774194)(1.5683873,1.6774194)
\psline[linecolor=black, linewidth=0.04, linestyle=dotted, dotsep=0.10583334cm](2.8587098,1.6774194)(11.374839,1.6774194)
\psline[linecolor=black, linewidth=0.04, linestyle=dotted, dotsep=0.10583334cm](0.020000149,1.4193549)(1.5683873,1.4193549)
\psline[linecolor=black, linewidth=0.04, linestyle=dotted, dotsep=0.10583334cm](2.6006453,1.4193549)(3.1167743,1.4193549)
\psline[linecolor=black, linewidth=0.04, linestyle=dotted, dotsep=0.10583334cm](4.1490326,1.4193549)(7.7619357,1.4193549)
\psline[linecolor=black, linewidth=0.04, linestyle=dotted, dotsep=0.10583334cm](9.0522585,1.4193549)(11.116775,1.4193549)(11.374839,1.4193549)
\psline[linecolor=black, linewidth=0.04, linestyle=dotted, dotsep=0.10583334cm](0.020000149,1.1612904)(1.3103228,1.1612904)
\psline[linecolor=black, linewidth=0.04, linestyle=dotted, dotsep=0.10583334cm](2.3425808,1.1612904)(2.6006453,1.1612904)
\psline[linecolor=black, linewidth=0.04, linestyle=dotted, dotsep=0.10583334cm](4.407097,1.1612904)(7.503871,1.1612904)
\psline[linecolor=black, linewidth=0.04, linestyle=dotted, dotsep=0.10583334cm](9.310323,1.1612904)(11.374839,1.1612904)
\psline[linecolor=black, linewidth=0.04, linestyle=dotted, dotsep=0.10583334cm](0.020000149,0.9032259)(1.3103228,0.9032259)
\psline[linecolor=black, linewidth=0.04, linestyle=dotted, dotsep=0.10583334cm](2.0845163,0.9032259)(2.3425808,0.9032259)(2.3425808,0.9032259)
\psline[linecolor=black, linewidth=0.04, linestyle=dotted, dotsep=0.10583334cm](4.6651616,0.9032259)(6.987742,0.9032259)
\psline[linecolor=black, linewidth=0.04, linestyle=dotted, dotsep=0.10583334cm](9.310323,0.9032259)(11.374839,0.9032259)
\psline[linecolor=black, linewidth=0.04, linestyle=dotted, dotsep=0.10583334cm](0.020000149,0.6451614)(1.0522583,0.6451614)
\psline[linecolor=black, linewidth=0.04, linestyle=dotted, dotsep=0.10583334cm](1.8264518,0.6451614)(2.3425808,0.6451614)
\psline[linecolor=black, linewidth=0.04, linestyle=dotted, dotsep=0.10583334cm](4.6651616,0.6451614)(6.987742,0.6451614)(6.987742,0.6451614)
\psline[linecolor=black, linewidth=0.04, linestyle=dotted, dotsep=0.10583334cm](9.568387,0.6451614)(11.374839,0.6451614)
\psline[linecolor=black, linewidth=0.04, linestyle=dotted, dotsep=0.10583334cm](0.020000149,0.38709685)(0.020000149,0.38709685)(1.0522583,0.38709685)
\psline[linecolor=black, linewidth=0.04, linestyle=dotted, dotsep=0.10583334cm](1.0522583,0.38709685)(1.0522583,0.38709685)
\psline[linecolor=black, linewidth=0.04, linestyle=dotted, dotsep=0.10583334cm](1.8264518,0.38709685)(2.0845163,0.38709685)
\psline[linecolor=black, linewidth=0.04, linestyle=dotted, dotsep=0.10583334cm](4.6651616,0.38709685)(6.987742,0.38709685)
\psline[linecolor=black, linewidth=0.04, linestyle=dotted, dotsep=0.10583334cm](9.568387,0.38709685)(11.374839,0.38709685)
\psline[linecolor=black, linewidth=0.04, linestyle=dotted, dotsep=0.10583334cm](0.020000149,0.12903233)(1.0522583,0.12903233)
\psline[linecolor=black, linewidth=0.04, linestyle=dotted, dotsep=0.10583334cm](1.8264518,0.12903233)(2.0845163,0.12903233)
\psline[linecolor=black, linewidth=0.04, linestyle=dotted, dotsep=0.10583334cm](2.0845163,0.12903233)(2.3425808,0.12903233)
\psline[linecolor=black, linewidth=0.04, linestyle=dotted, dotsep=0.10583334cm](4.6651616,0.12903233)(6.987742,0.12903233)
\psline[linecolor=black, linewidth=0.04, linestyle=dotted, dotsep=0.10583334cm](9.568387,0.12903233)(11.374839,0.12903233)
\psline[linecolor=black, linewidth=0.04, linestyle=dotted, dotsep=0.10583334cm](0.020000149,-0.12903218)(1.0522583,-0.12903218)
\psline[linecolor=black, linewidth=0.04, linestyle=dotted, dotsep=0.10583334cm](1.8264518,-0.12903218)(2.3425808,-0.12903218)
\psline[linecolor=black, linewidth=0.04, linestyle=dotted, dotsep=0.10583334cm](4.6651616,-0.12903218)(6.987742,-0.12903218)
\psline[linecolor=black, linewidth=0.04, linestyle=dotted, dotsep=0.10583334cm](9.568387,-0.12903218)(11.116775,-0.12903218)
\psline[linecolor=black, linewidth=0.04, linestyle=dotted, dotsep=0.10583334cm](0.020000149,-0.3870967)(1.0522583,-0.3870967)
\psline[linecolor=black, linewidth=0.04, linestyle=dotted, dotsep=0.10583334cm](1.8264518,-0.3870967)(2.3425808,-0.3870967)
\psline[linecolor=black, linewidth=0.04, linestyle=dotted, dotsep=0.10583334cm](4.407097,-0.3870967)(7.2458067,-0.3870967)
\psline[linecolor=black, linewidth=0.04, linestyle=dotted, dotsep=0.10583334cm](9.310323,-0.3870967)(11.374839,-0.3870967)
\psline[linecolor=black, linewidth=0.04, linestyle=dotted, dotsep=0.10583334cm](0.020000149,-0.6451612)(1.3103228,-0.6451612)
\psline[linecolor=black, linewidth=0.04, linestyle=dotted, dotsep=0.10583334cm](2.0845163,-0.6451612)(2.6006453,-0.6451612)
\psline[linecolor=black, linewidth=0.04, linestyle=dotted, dotsep=0.10583334cm](4.407097,-0.6451612)(7.503871,-0.6451612)
\psline[linecolor=black, linewidth=0.04, linestyle=dotted, dotsep=0.10583334cm](9.0522585,-0.6451612)(11.374839,-0.6451612)
\psline[linecolor=black, linewidth=0.04, linestyle=dotted, dotsep=0.10583334cm](0.020000149,-0.9032257)(1.3103228,-0.9032257)
\psline[linecolor=black, linewidth=0.04, linestyle=dotted, dotsep=0.10583334cm](2.0845163,-0.9032257)(11.374839,-0.9032257)
\psline[linecolor=black, linewidth=0.04, linestyle=dotted, dotsep=0.10583334cm](0.020000149,-1.1612903)(1.3103228,-1.1612903)
\psline[linecolor=black, linewidth=0.04, linestyle=dotted, dotsep=0.10583334cm](2.3425808,-1.1612903)(11.374839,-1.1612903)
\psline[linecolor=black, linewidth=0.04, linestyle=dotted, dotsep=0.10583334cm](0.020000149,-1.4193548)(1.5683873,-1.4193548)
\psline[linecolor=black, linewidth=0.04, linestyle=dotted, dotsep=0.10583334cm](2.6006453,-1.4193548)(11.374839,-1.4193548)
\psline[linecolor=black, linewidth=0.04, linestyle=dotted, dotsep=0.10583334cm](0.020000149,-1.6774193)(1.8264518,-1.6774193)(1.8264518,-1.6774193)
\psline[linecolor=black, linewidth=0.04, linestyle=dotted, dotsep=0.10583334cm](3.6329033,-1.6774193)(11.374839,-1.6774193)
\psline[linecolor=black, linewidth=0.04, linestyle=dotted, dotsep=0.10583334cm](0.020000149,-1.9354838)(2.3425808,-1.9354838)
\psline[linecolor=black, linewidth=0.04, linestyle=dotted, dotsep=0.10583334cm](3.6329033,-1.9354838)(11.374839,-1.9354838)(11.374839,-1.9354838)
\psline[linecolor=black, linewidth=0.04, linestyle=dotted, dotsep=0.10583334cm](0.020000149,-2.1935482)(2.6006453,-2.1935482)
\psline[linecolor=black, linewidth=0.04, linestyle=dotted, dotsep=0.10583334cm](3.6329033,-2.1935482)(11.374839,-2.1935482)
\psline[linecolor=black, linewidth=0.04, linestyle=dotted, dotsep=0.10583334cm](0.020000149,-2.451613)(11.374839,-2.451613)
\psline[linecolor=black, linewidth=0.04, linestyle=dotted, dotsep=0.10583334cm](0.020000149,-2.7096775)(11.374839,-2.7096775)
\end{pspicture}
}
\caption{Các nhóm được gom dựa vào mật độ}
\label{fig:pic15}
\end{figure}


\subsection{Chia sẻ thuộc tính (gom nhóm theo ý tưởng)}
Ta có thể định nghĩa nhóm như là tập hợp của các đối tượng mà chia sẻ thuộc tính chung.
Khái niệm này bao gồm tất cả định nghĩa về nhóm ở các phần trước.
Chẳng hạn, khi gom nhóm dựa vào trung điểm, các đối tượng nằm gần khu vực trung tâm của nhóm có thuộc tính gần giống với điểm trung tâm hay là điểm tiêu biểu, như là các nhóm ở trong hình \ref{fig:pic16}.
Tuy nhiên, gom nhóm dựa vào chia sẻ thuộc tính có thể tạo ra dạng nhóm mới.
Dựa vào hình \ref{fig:pic16}, khu vực tam giác nằm kế cận khu vực tứ giác và 2 đường tròn quấn vào nhau.
Trong cả 2 trường hợp trên, thuật toán gom nhóm cần lưu ý hình dạng đặc tả của nhóm để có thể tìm được những nhóm này thành công.
Quá trình tìm kiếm những nhóm như vậy được gọi là gom nhom theo ý tưởng.
Tuy nhiên, nếu hình dạng của nhóm quá phức tạp có thể dẫn đến nhận diện mẫu.

\begin{figure}[htp]
\makeatletter % For spaces in paths
\patchcmd\Gread@eps{\@inputcheck#1 }{\@inputcheck"#1"\relax}{}{}
\makeatother
\psscalebox{1.0 1.0} % Change this value to rescale the drawing.
{
\begin{pspicture}(0,-1.2132502)(14.024741,1.2132502)
\definecolor{colour1}{rgb}{0.4,0.2,1.0}
\definecolor{colour2}{rgb}{0.4,0.4,0.0}
\psframe[linecolor=black, linewidth=0.04, dimen=outer](6.8247414,1.1932502)(2.4247413,-1.2067499)
\rput{-269.72128}(1.2239491,-1.2315094){\pstriangle[linecolor=black, linewidth=0.04, dimen=outer](1.2247412,-1.2067499)(2.4,2.4)}
\psline[linecolor=black, linewidth=0.04, linestyle=dotted, dotsep=0.10583334cm](1.6247412,0.79325014)(2.4247413,0.79325014)
\psline[linecolor=black, linewidth=0.04, linestyle=dotted, dotsep=0.10583334cm](1.6247412,0.79325014)(1.6247412,0.79325014)
\psline[linecolor=black, linewidth=0.04, linestyle=dotted, dotsep=0.10583334cm](1.2247412,0.3932501)(2.4247413,0.3932501)
\psline[linecolor=black, linewidth=0.04, linestyle=dotted, dotsep=0.10583334cm](0.4247412,-0.0067498777)(2.4247413,-0.0067498777)
\psline[linecolor=black, linewidth=0.04, linestyle=dotted, dotsep=0.10583334cm](1.2247412,-0.40674987)(2.4247413,-0.40674987)
\psline[linecolor=black, linewidth=0.04, linestyle=dotted, dotsep=0.10583334cm](1.6247412,-0.8067499)(2.4247413,-0.8067499)
\psline[linecolor=black, linewidth=0.04, linestyle=dotted, dotsep=0.10583334cm](2.8247411,1.1932502)(2.8247411,-1.2067499)
\psline[linecolor=black, linewidth=0.04, linestyle=dotted, dotsep=0.10583334cm](3.2247412,1.1932502)(3.2247412,-1.2067499)
\psline[linecolor=black, linewidth=0.04, linestyle=dotted, dotsep=0.10583334cm](3.6247413,1.1932502)(3.6247413,-1.2067499)(3.6247413,-0.40674987)
\psline[linecolor=black, linewidth=0.04, linestyle=dotted, dotsep=0.10583334cm](4.024741,1.1932502)(4.024741,-1.2067499)
\psline[linecolor=black, linewidth=0.04, linestyle=dotted, dotsep=0.10583334cm](4.4247413,1.1932502)(4.4247413,-1.2067499)
\psline[linecolor=black, linewidth=0.04, linestyle=dotted, dotsep=0.10583334cm](4.8247414,1.1932502)(4.8247414,-1.2067499)
\psline[linecolor=black, linewidth=0.04, linestyle=dotted, dotsep=0.10583334cm](5.224741,1.1932502)(5.224741,-1.2067499)
\psline[linecolor=black, linewidth=0.04, linestyle=dotted, dotsep=0.10583334cm](5.624741,1.1932502)(5.624741,-1.2067499)
\psline[linecolor=black, linewidth=0.04, linestyle=dotted, dotsep=0.10583334cm](6.024741,1.1932502)(6.024741,1.1932502)(6.024741,-1.2067499)
\psline[linecolor=black, linewidth=0.04, linestyle=dotted, dotsep=0.10583334cm](6.4247413,1.1932502)(6.4247413,-1.2067499)
\psellipse[linecolor=colour1, linewidth=0.4, dimen=outer](10.224741,-0.0067498777)(1.8,1.2)
\psellipse[linecolor=colour2, linewidth=0.4, dimen=outer](12.224741,-0.0067498777)(1.8,1.2)
\end{pspicture}
}
\caption{Gom nhóm ý tưởng}
\label{fig:pic16}
\end{figure}


\section{Phương pháp gom nhóm văn bản}
\label{sec:ppgn}
Dựa vào những loại gom nhóm được đề cập ở \ref{sec:clgn}, ta có các phương pháp gom nhóm văn bản sau:
\subsection{Gom nhóm văn bản phân chia}
~\cite{partitioning-clustering, partitioning-clustering-Krish} Gom nhóm văn bản phân chia tạo thành các nhóm dựa vào số lượng $K$ nhóm cho trước với điều kiện $K$ nhỏ hơn số lượng văn bản.
Các nhóm này được hình thành sao cho tối tiểu hóa tổng của bình phương khoảng cách.
\begin{equation}
\sum_{m=1}^k \sum{t_{mi} \in K_m} (C_m - t_{mi})^2
\end{equation}
Hướng tiếp cận này có tối ưu toàn cục với mỗi nhóm có ít nhất một văn bản và mỗi văn bản chỉ thuộc về một nhóm duy nhất.
Các thuật toán tiêu biểu là k-means (mỗi nhóm được đại diện bởi điểm trung tâm) và k-medoids (mỗi nhóm được đại diện bởi điểm tiêu biểu).

Trong đó, thuật toán k-means được triển khai tương đối đơn giản.
Trước hết, ta khởi tạo $K$ nhóm với điểm trung tâm ngẫu nhiên.
Mỗi văn bản sẽ được gán vào nhóm có điểm trung tâm gần nhất.
Sau đó, ta sẽ cập nhật lại vị trí điểm trung tâm cho từng nhóm.
Thuật toán sẽ được lặp cho đến khi hội tụ.

\textbf{Ưu điểm}
\begin{enumerate}
\item[•]Hướng tiếp cận này đơn giản, dễ hiểu.
\item[•]Tất cả các văn bản tự động được gom vào một nhóm nhất định.
\item[•]Phương pháp này tương đối hiệu quả và thường kết thúc ở tối ưu cục bộ.
\end{enumerate}

\textbf{Khuyết điểm}
\begin{enumerate}
\item[•]Hướng tiếp cận này phải chọn trước $K$ nhóm cần phải gom.
\item[•]Tất cả các văn bản buộc phải gom vào một nhóm cố định.
\item[•]Cách tiếp cận này không thể xử lý nhiễu hoặc văn bản tách biệt vì quá nhạy cảm với các văn bản tách biệt.
\item[•]Hướng tiếp cận này không thích hợp để tìm kiếm hình dạng các nhóm mà không phải là đa giác lồi.
\end{enumerate}

\subsection{Gom nhóm văn bản phân cấp}
\label{sec:gnvbpc}
~\cite{hierarchical-clustering} Gom nhóm văn bản phân cấp tạo ra các nhóm lồng vào nhau thành cấu trúc cây phân cấp.
Hình ảnh của cấu trúc cây phân cấp cho ta thấy được quan hệ giữa các nhóm.
Phương pháp này có 2 cách tiếp cận khác nhau: gom nhóm phân cấp tích tụ  và gom nhóm phân cấp phân chia.
Gom nhóm phân cấp tích tụ bắt đầu với mỗi nhóm là một văn bản rồi từ từ gom dần các văn bản lại với nhau cho đến khi chỉ còn một nhóm duy nhất.
Gom nhóm phân cấp phân chia bắt đầu với tất cả các văn bản là một nhóm lớn rồi chia nhỏ dần nhóm này đến khi mỗi nhóm chỉ có một văn bản.

\textbf{Ưu điểm}
\begin{enumerate}
\item[•]Hướng tiếp cận này không cần phải đưa trước số nhóm cụ thể cần phải gom.
\item[•]Phương pháp này có thể cho ta số nhóm mong muốn bằng việc cắt tại một mức nào đó của cây phân cấp.
\end{enumerate}

\textbf{Khuyết điểm}
\begin{enumerate}
\item[•]Quyết định gom nhóm khi đã thực hiện thì không thể làm ngược lại.
\item[•]Hướng tiếp cận này không có hàm mục tiêu nhất định để tối ưu.
\item[•]Cách tiếp cận này có độ phức tạp khá cao, ít nhất là $O(n^2)$ với n là số lượng nhóm cần phải gom.
\item[•]Đôi khi ta muốn tìm kiếm số lượng nhóm thực sự của dữ liệu nên ta sẽ cắt tại một mức nào đó tại đồ thị của cây phân cấp.
Tuy nhiên, việc xác định vết cắt không phải lúc nào cũng là dễ dàng.
\end{enumerate}

\subsection{Gom nhóm văn bản mật độ} 
~\cite{Manojit-Nandi} Gom nhóm văn bản mật độ xem mỗi văn bản như là tâm của đường tròn.
Thuật toán bắt đầu với bán kính $\epsilon$ cho trước, ta tính khoảng cách từ tâm đến các văn bản khác và nếu khoảng cách này nhỏ hơn bán kính $\epsilon$ thì văn bản đó sẽ được gom lại vào thành một nhóm.
Phương pháp gom nhóm này sẽ cho ta thấy được số nhóm sẽ có trong ngữ liệu mà không cần phải cho trước.
Thuật toán tiêu biểu cho hướng tiếp cận này là DBSCAN.

\textbf{Ưu điểm}
\begin{enumerate}
\item[•]Hướng tiếp cận không cần phải đưa trước số nhóm cụ thể cần phải gom.
\end{enumerate}

\textbf{Khuyết điểm}
\begin{enumerate}
\item[•]Những điểm nằm trong vùng mật độ thấp được xem như là nhiễu và bị loại bỏ.
\item[•]Cần xác định bán kính cụ thể và số lượng văn bản ít nhất để tạo thành nhóm.
\end{enumerate}

%@online{densitycluster,
% author               = {Manojit Nandi},
% title                = {Density-Based Clustering},
% url                  = {https://blog.dominodatalab.com/topology-and-density-based-clustering},
% urldate              = {9 September 2015},
%}

%\begin{enumerate}
%\item[•]\textbf{K-means}: đây là phương pháp gom nhóm theo mẫu và sử dụng kỹ thuật phân chia để có thể tìm được số lượng nhóm biết trước (K), và mỗi nhóm được đại diện bằng điểm trung tâm.
%\item[•]\textbf{Gom nhóm phân cấp tích tụ}: cách gom nhóm này tiếp cận theo một tập hợp của những nhóm liên quan mật thiết mà tạo ra được các nhóm phân cấp bằng việc bắt đầu với mỗi điểm như là một nhóm đơn lẻ và sau đó thì không ngừng trộn những nhóm ở gần với nhau nhất cho đến khi chỉ còn lại một nhóm duy nhất.
%Một vài kỹ thuật của cách tiếp cận này có kiểu thực thi theo cách gom nhóm theo đồ thị, trong khi một số khác thì theo hướng tiếp cận gom nhóm theo mẫu.
%\item[•]\textbf{DBSCAN}: đây là thuật toán gom nhóm the hướng dựa vào mật độ tạo ra các nhóm được phân chia, trong đó số lượng nhóm được xác định tự động bởi thuật toán.
%Những điểm nằm trong vùng mật độ thấp được xem như là nhiễu và bị loại bỏ.
%Vì vậy, DBSCAN không thể tạo ra gom nhóm hoàn chỉnh.
%\end{enumerate}

\section{Mục tiêu của đồ án}
\label{sec:mtcda}
Mục tiêu của đồ án là gom nhóm văn bản tiếng Việt.
Trong số những phương pháp đã liệt kê ở \ref{sec:ppgn} thì phương pháp gom nhóm phân cấp được sử dụng cho mục tiêu của đồ án vì:
\begin{enumerate}
\item[•]Gom nhóm phân cấp không cần biết trước số lượng nhóm (K-means cần biết trước số lượng nhóm).
\item[•]Gom nhóm phân cấp bảo toàn dữ liệu (DBSCAN có thể loại bỏ nhiễu).
\item[•]Gom nhóm phân cấp còn có thể cho ta có được kết quả của gom nhóm phân chia giống như K-means khi thực hiện vết cắt tại một mức nào đó trong cây phân cấp.
\end{enumerate}

Thông thường, gom nhóm thường hay sử dụng văn bản được biểu diễn dưới dạng vector tần số của các từ trong ngữ liệu.
Tuy nhiên, khi ta gom nhóm với ngữ liệu lớn thì số chiều của vector sẽ tăng lên rất nhiều lần và làm chậm quá trình thực thi của thuật toán gom nhóm.
Vì vậy, ta sẽ tìm cách để thay đổi cách thể hiện của văn bản để có thể làm giảm số chiều của vector.
Đồ án đề xuất sử dụng doc2vec để biểu diễn vector cho văn bản nhằm thay thế cách sử dụng tần số của từ.
Sự chuyển đổi này có thể làm giảm số chiều của vector trong văn bản và làm tăng hiệu năng và tốc độ trong quá trình thực thi.
Việc chuyển đối thể hiện của văn bản là điểm đóng góp của đồ án với nhiệm vụ làm giảm số chiều để tăng nhanh quá trình thực thi.


\chapter{Trình bày luận văn}
\label{Chapter2}

\section{Bố cục của khóa luận}

Nội dung khoá luận trình bày tối thiểu 50 trang khổ A4 và không nên vượt quá 100 trang (không kể các trang bìa, lời cám ơn, mục lục, tài liệu tham khảo \ldots) theo trình tự như sau:

\begin{itemize}
\item MỞ ĐẦU (thường đặt tên là ``Giới thiệu''): Trình bày lí do chọn đề tài, mục đích, đối tượng và phạm vi nghiên cứu.
Mô tả bài toán mà khóa luận giải quyết.
Bài toán này có gì hay?
Tại sao lại cần giải quyết bài toán này?
Bài toán này có gì khó?
Có những hướng nào để giải quyết bài toán này?
Những hướng giải quyết trước đây có những vấn đề gì chưa giải quyết được?
Các câu hỏi nghiên cứu mà khóa luận sẽ trả lời hoặc những vấn đề mà khóa luận sẽ giải quyết.
Các đóng góp của khóa luận.

\item TỔNG QUAN (thường đặt tên là ``Các công trình liên quan''): Phân tích đánh giá các hướng nghiên cứu đã có của các tác giả trong và ngoài nước liên quan đến đề tài; nêu những vấn đề còn tồn tại (những vấn đề nào mà các công trình khác chưa giải quyết được); chỉ ra những vấn đề mà đề tài cần tập trung, nghiên cứu giải quyết.

\item NGHIÊN CỨU THỰC NGHIỆM HOẶC LÍ THUYẾT (thường đặt tên là ``Phương pháp đề xuất''): Trình bày cơ sở lí thuyết, lí luận, giả thiết khoa học và phương pháp nghiên cứu đã được sử dụng trong khoá luận.
Nếu đề xuất hướng giải quyết mới, mô hình mới thì cần mô tả chi tiết cách giải quyết của mình (chi tiết tới mức người khác có thể dựa vào phần này mà cài đặt lại được đúng hoàn toàn phương pháp của mình đề ra).

\item TRÌNH BÀY, ĐÁNH GIÁ BÀN LUẬN VỀ CÁC KẾT QUẢ (thường đặt tên là ``Kết quả thí nghiệm''): Mô tả các kết quả nghiên cứu khoa học hoặc kết quả thực nghiệm.
Đối với các đề tài ứng dụng có kết quả là sản phẩm phần mềm phải có hồ sơ thiết kế, cài đặt,\ldots theo một trong các mô hình đã học (UML,\ldots).
Thông thường cần mô tả môi trường thí nghiệm trước như sử dụng dữ liệu nào, dùng độ đo nào để đánh giá, môi trường chạy thí nghiệm (cấu hình máy nếu cần phân tích thông tin về thời gian chạy thực nghiệm). Sau đó, nêu kết quả thực nghiệm, bàn luận và giải thích kết quả.

\item KẾT LUẬN VÀ HƯỚNG PHÁT TRIỂN (thường đặt tên là ``Kết luận''): Trình bày những kết quả đạt được, những đóng góp mới và những đề xuất mới, kiến nghị về những hướng nghiên cứu tiếp theo.

\item DANH MỤC TÀI LIỆU THAM KHẢO: Chỉ bao gồm các tài liệu được trích dẫn, sử dụng và đề cập tới để bàn luận trong khoá luận.
Phần này các bạn chuẩn bị 1 file BIB để lưu các tài liệu trích dẫn.
Khi các bạn trích dẫn một tài liệu nào đó, LaTeX sẽ tự động thêm vào danh mục tài liệu tham khảo giúp các bạn.
Các bạn xem hướng dẫn cách trích dẫn ở chương sau.

\item PHỤ LỤC: Phần này bao gồm nội dung cần thiết nhằm minh họa hoặc hỗ trợ cho nội dung khóa luận như số liệu, mẫu biểu, tranh ảnh,\ldots Phụ lục không được dày hơn phần chính của luận văn.
Nếu có công trình công bố thì để vào phần phụ lục này.
\end{itemize}

\section{Bảng biểu, hình vẽ, phương trình}

Những qui định dưới này các bạn có thể bỏ qua hoặc đọc để hiểu thêm.
Những định dạng này LaTeX đều tự động giúp các bạn.
Các bạn xem hướng dẫn chi tiết hơn ở chương sau.

Việc đánh số bảng biểu, hình vẽ, phương trình phải gắn với số chương; ví dụ hình 3.4 có nghĩa l hình thứ 4 trong Chương 3.
Mọi đồ thị, bảng biểu lấy từ các nguồn khác phải được trích dẫn đầy đủ.
Nguồn được trích dẫn phải được liệt kê chính xác trong danh mục Tài liệu tham khảo.
Đầu đề của bảng biểu ghi phía trên bảng, đầu đề của hình vẽ ghi phía dưới hình.
Thông thường, những bảng ngắn và đồ thị phải đi liền với phần nội dung đề cập tới các bảng và đồ thị này ở lần thứ nhất.
Các bảng dài có thể để ở những trang riêng nhưng cũng phải tiếp theo ngay phần nội dung đề cập tới bảng này ở lần đầu tiên.
Các bảng rộng vẫn nên trình bày theo chiều đứng dài 297mm của trang giấy, chiều rộng của trang giấy có thể hơn 210mm.
Chú ý gấp trang giấy sao cho số và đầu đề của hình vẽ hoặc bảng vẫn có thể nhìn thấy ngay mà không cần mở rộng tờ giấy.
Tuy nhiên hạn chế sử dụng các bảng quá rộng này.

Đối với những trang giấy có chiều đứng hơn 297mm (bản đồ, bản vẽ,\ldots) thì có thể để trong một phong bì cứng đính bên trong bìa sau của luận văn.
Các hình vẽ phải sạch sẽ bằng mực đen để có thể sao chụp lại; có đánh số và ghi đầy đủ đầu đề, cỡ chữ phải bằng cỡ chữ sử dụng trong văn bản luận văn.
Khi đề cập đến các bảng biểu và hình vẽ phải nêu rõ số của hình và bảng biểu đó, ví dụ ``... được nêu trong Bảng 4.1'' hoặc ``xem Hình 3.2'' mà không được viết ``… được nêu trong bảng dưới đây'' hoặc ``trong đồ thị của X và Y sau''.

Việc trình bày phương trình toán học trên một dòng đơn hoặc dòng kép tùy ý, tuy nhiên phải thống nhất trong toàn luận văn.
Khi ký hiệu xuất hiện lần đầu tiên thì phải giải thích và đơn vị tính phải đi kèm ngay trong phương trình có ký hiệu đó.
Nếu cần thiết, danh mục của tất cả các ký hiệu, chữ viết tắt và nghĩa của chúng cần được liệt kê và để ở phần đầu của luận văn.
Tất cả các phương trình cần được đánh số và để trong ngoặc đơn đặt bên phía lề phải.
Nếu một nhóm phương trình mang cùng một số thì những số này cũng được để trong ngoặc, hoặc mỗi phương trình trong nhóm phương trình (5.1) có thể được đánh số là (5.1.1), (5.1.2), (5.1.3).

\section{Viết tắt}

\textbf{Không lạm dụng việc viết tắt} trong luận văn.
Chỉ viết tắt những từ, cụm từ hoặc thuật ngữ được sử dụng nhiều lần trong luận văn.
Không viết tắt những cụm từ  dài, những mệnh đề; không viết tắt những cụm từ ít xuất hiện trong luận văn.
Nếu cần viết tắt những từ thuật ngữ, tên các cơ quan, tổ chức,\ldots thì được viết tắt sau lần viết thứ nhất có kèm theo chữ viết tắt trong ngoặc đơn.
Nếu luận văn có nhiều chữ viết tắt thì phải có bảng danh mục các chữ viết tắt (xếp theo thứ tự ABC) ở phần đầu luận văn.

Nhắc lại: \textbf{không lạm dụng việc viết tắt} trong luận văn.
Khi các bạn sử dụng từ viết tắt, người đọc sẽ phải lật lại những phần đã đọc, để tìm lại xem từ viết tắt đó nghĩa là gì.
Việc này sẽ làm chậm tốc độ đọc và sẽ khiến người đọc khó theo dõi khóa luận của bạn hơn.
Nếu có thể, hạn chế hoàn toàn việc dùng viết tắt.

\section{Tài liệu tham khảo và cách trích dẫn}

Mọi ý kiến, khái niệm có ý nghĩa, mang tính chất gợi ý không phải của riêng tác giả và mọi tham khảo khác phải được trích dẫn và chỉ ra nguồn trong danh mục tài liệu tham khảo của luận văn.
Không trích dẫn những kiến thức phổ biến, mọi người đều biết cũng như không làm luận văn nặng nề với những tham khảo trích dẫn.
Việc trích dẫn, tham khảo chủ yếu nhằm thừa nhận nguồn của những ý tưởng có giá trị giúp người đọc theo được mạch suy nghĩ của tác giả, không làm trở ngại việc đọc.
Nếu không có điều kiện tiếp cận được một tài liệu gốc mà phải trích dẫn thông qua một tài liệu khác thì phải nêu ra trích dẫn này, đồng thời tài liệu gốc đó không được liệt kê trong danh mục tài liệu tham khảo của khóa luận.

\chapter{Phương pháp đề xuất}
\label{Chapter3}

%Thuật toán gom nhóm phân cấp được thể hiện thông qua sử dụng ma trận tương đồng.
%Điều này đòi hỏi dung lượng cần thiết để lưu trữ ma trận tương đồng là $\frac{1}{2} m^2$ (giả định ma trận tương đồng là ma trận vuông) với $m$ là số điểm dữ liệu.
%Thuật toán cũng cần khoảng trống cần thiết để giữ cho việc đánh dấu tỷ lệ các nhóm được gom với tổng số nhóm, có giá trị là $m - 1$ và trừ đi những nhóm đơn lẻ.
%Vì thế, dung lượng cần thiết để chạy thuật toán gom nhóm phân cấp là $O(m^2)$.
%
%Phân tích cơ bản dành cho thuật toán gom nhóm phân cấp cũng liên quan trực tiếp đến độ phức tạp tính toán.
%$O(m^2)$ là thời gian cần thiết để tính ma trận tương đồng.
%Sau bước đó, dựa vào thuật toán \ref{agl:agglomerative}, ta còn $m - 1$ lần lặp cho bước 3 và 4 vì ta có $m$ nhóm lúc ban đầu và mối lần lặp thì có 2 nhóm được gom vào.
%Nếu ta thực thi như tìm kiếm tuyến tính của ma trận tương đồng thì sau lần lặp thứ $i$ thì thời gian ở bước 3 sẽ là $O(m - i + 1)^2)$.
%Điều này tỷ lệ với số lương hiện tại của bình phương nhóm.%
%Thời gian cần để chạy bước thứ 4 là $O(m - i + 1)$ để cập nhật ma trận tương đồng sau khi gom 2 nhóm (một nhóm được gom lại chỉ mất khoảng $O(m - i + 1)$).
%Nếu như không có thay đổi, thuật toán có độ phức tạp là $O(m^3)$.
%Trong trường hợp khoảng cách từ nhóm này đến các nhóm khác được lưu trữ theo thứ tự thì có khả năng làm giảm quá trình tìm kiếm 2 nhóm gần nhất.
%Tuy nhiên, trường hợp tổng quát cho độ phức tạp của thuật toán là $O(m^2 \log)m$.

\section{Thể hiện văn bản}
Như đã phân tích trong \ref{sec:dpt}, thuật toán gom nhóm phân cấp kết hợp có độ phức tạp khá cao và tốn nhiều dung lượng lúc thực thi.
Đồ án tập trung vào nhiệm vụ cải thiện dung lượng của thuật toán lúc thực thi để có thể mở rộng giới hạn dữ liệu để gom nhóm và tăng tốc thời gian thực thi.
Việc thuật toán tốn nhiều dung lượng để chạy thực nghiệm một phần đến từ biểu diễn văn bản bằng vector có số chiều quá lớn.
Chính vì vector có số chiều quá lớn nên không gian để lưu trữ cũng phải tăng theo.
Giả sử ngữ liệu có $m$ văn bản, số lượng từ vựng là $n$ thì dung lượng cần thiết để biểu diễn toàn bộ ngữ liệu dưới dạng vector sẽ là $m * n$.
Trong công thức trên, ta không thể thay đổi số lượng văn bản $m$ nên chỉ có thể cải thiện số chiều $n$ của vector để giảm bộ nhớ lúc thực thi.
Vì vậy, đồ án đã đề xuất thể hiện văn bản bằng doc2vec là biểu diễn vector có số chiều thấp hơn.

Việc văn bản được biểu diễn bằng vector là để giúp cho việc tính toán trở nên dễ dàng hơn do nội dung của văn bản đa phần bao gồm từ ngữ, câu chữ.
Nếu văn bản được để nguyên định dạng để gom nhóm thì gây ra nhiều khó khăn cho việc tính toán.
Vì vậy, văn bản sẽ được chuyển sang định dạng vector.
Tuy nhiên, việc biểu diễn văn bản bằng vector có nhiều cách khác nhau và cách biểu diễn này có thể ảnh hưởng đến hiệu năng của thuật toán nên phải chọn cách biểu diễn thích hợp với thuật toán gom nhóm phân cấp kết hợp.

Một trong số những cách biểu diễn văn bản bằng vector căn bản nhất là sử dụng vector nhị phân để xác định một từ có nằm trong vector hay không?
Cách biểu diễn này xây dựng tập hợp từ của các văn bản, số lượng từ trong tập hợp cũng chính là số chiều của vector.
Mỗi văn bản sẽ biểu diễn bằng vector theo thứ tự của tập hợp các từ và có giá trị 1 (từ này nằm trong văn bản) hoặc là 0 (từ này không nằm trong văn bản).
Ví dụ, ta có 2 văn bản như sau:

\textbf{VB1}: ``Tôi đi học''

\textbf{VB2}: ``Tôi đang đến trường''

Để biểu diễn vector nhị phân cho văn bản, trước hết ta sẽ xây dựng tập hợp từ thành bảng sau \ref{tab:3_1}:
\begin{table}[h!]
\centering
\caption{Bảng thể hiện vị trí của từ}
\label{tab:3_1}
\begin{tabular}{|c|c|}
\hline
Vị Trí & Từ							\\ \hline
0    & Tôi                \\ \hline
1    & đi               \\ \hline
2    & học               \\ \hline
3    & đang               \\ \hline
4    & đến               \\ \hline
5    & trường               \\ \hline
\end{tabular}
\end{table}

Vector nhị phân của 2 văn bản \textbf{VB1} và \textbf{VB2} là:

\textbf{VB1}: (1, 1, 1, 0, 0, 0)

\textbf{VB2}: (1, 0, 0, 1, 1, 1)

Điểm yếu của vector nhị phân là cách thể hiện này chỉ cho ta biết được từ nào có xuất hiện trong văn bản chứ không thể xác định được số lượng của một từ có trong văn bản là bao nhiêu.
Giả sử ta gộp \textbf{VB1} và \textbf{VB2} thành \textbf{VB3}.

\textbf{VB3}: ``Tôi đi học. Tôi đang đến trường''

Khi đó, vector nhị phân cho \textbf{VB3} sẽ là:

\textbf{VB3}: (1, 1, 1, 1, 1, 1)

Như vậy, vector nhị phân không thể cho ta biết được từ ``Tôi'' đã xuất hiện đến 2 lần trong \textbf{VB3}.
Vì vậy, một cách khác được đề xuất có tên gọi là TF-IDF (term frequency - inverse document frequency).
Theo như~\cite{tf-idf}: ``TF-IDF là tích của 2 số TF và IDF. Trong đó TF là tần số của một từ xuất hiện trong văn bản và IDF là tần số ngược của văn bản, nghĩa là số lần xuất hiện ở những văn bản khác nhau của từ đó''.
Công thức để tính TFIDF cho một từ trong văn bản là:

\begin{equation}
tfidf (t, d, D) = tf(t, d) * idf (t, D)
\end{equation}

\begin{equation}
idf (t, D) = log \frac{N}{|{d \in D: t \in d}|}
\end{equation}

\begin{enumerate}
\item[•]$D$: tập ngữ liệu
\item[•]$d$: văn bản thuộc tập $D$
\item[•]$N$: tổng số văn bản trong ngữ liệu $N = |D|$.
\item[•]$|{d \in D: t \in d}|$: số lượng văn bản mà từ $t$ xuất hiện. Tuy nhiên trong trường hợp từ nào đó không xuất hiện trong văn bản thì có thể dẫn đến chia cho 0 nên phần này sẽ được sửa lại thành  $1 + |{d \in D: t \in d}|$
\end{enumerate}

Chiều của vector theo cách biểu diễn TFIDF cũng chính là số lượng từ trong tập hợp từ.
Tuy nhiên, TFIDF có số chiều phụ thuộc vào số lượng từ trong tập hợp nên số chiều của vector là rất lớn.
Ngoài ra, dù vector có số chiều là rất lớn nhưng số lượng phần tử bằng 0 (từ này không xuất hiện trong văn bản) trong vector cũng nhiều không kém.
Điều này dẫn đến tốn kém quá nhiều dung lượng lưu trữ để biểu diễn cho vector TFIDF.
Vì vậy, một phương pháp biểu diễn khác để cải thiện vấn đề này, đó chính là doc2vec.
%Áp dụng cách biểu diễn TFIDF cho \textbf{VB1} và \textbf{VB2} thì ta có:
%
%\textbf{VB1}: ()
%
%\textbf{VB2}: ()


\section{Gom nhóm phân cấp kết hợp với văn bản doc2vec}
Doc2vec~\cite{doc2vec-original} là thuật toán học không giám sát để tạo ra vector thể hiện cho câu, đoạn văn hay là văn bản~\cite{doc2vec-1}.
Thuật toán doc2vec là phần chỉnh sửa, nâng cấp từ word2vec.
Đồ án sử dụng thuật toán doc2vec từ thư viện gensim\ref{gensim} để huấn luyện tập dữ liệu.
Vì doc2vec là phần mở rộng từ word2vec nên có thể tái sử dụng lại những mẫu từ word2vec~\cite{doc2vec-2}. 
Ta có thể dễ dàng chỉnh sửa lại chiều của vector, kích thước của tham số \textit{sliding window}, số lượng \textit{workers} hay là bất kì tham số nào tương ứng với việc thay đổi trong mô hình Word2vec.

Điều duy nhất khác biệt giữa thuật toán doc2vec với word2vec là liên quan đến phương pháp huấn luyện khi sử dụng mô hình doc2vec.
Trong kiến trúc word2vec, hai thuật toán được sử dụng là continuous bag of word (cbow) và skip-gram (sg).
Đối với doc2vec, thuật toán tương ứng là distributed memory (dm) và distributed bags of words (dbow).
Vì mô hình dm có hiệu năng tốt hơn đây sẽ là thuật toán mặc định lúc chạy doc2vec.
Tuy nhiên, ta vẫn có thể chuyển sang mô hình dbow nếu muốn bằng việc thiết lập cờ \textit{dm=0} trong hàm khởi tạo.

Phần dữ liệu đầu vào của doc2vec là một dãy những đối tượng \textbf{TaggedDocument} với mỗi đối tượng thể hiện văn bản và có 2 phần cơ bản là tập hợp các từ và nhãn như sau:
\begin{lstlisting}[language=Python]
taggedDocument = TaggedDocument(words=[u'some', u'words', u'here'], labels=[u'SENT_1'])
\end{lstlisting}

Thuật toán sẽ chạy qua \textbf{TaggedDocument} 2 lần lặp.
Lần đầu để xây dựng tập từ, lần thứ hai để huấn luyện mô hình cho tập dữ liệu đầu vào, học cách thể hiện vector cho mỗi từ và nhãn cho ngữ liệu.
Đồ án sẽ thực thi một lớp để tạo ra những vector cho tập văn bản.
\begin{lstlisting}[language=Python]
class DocIterator:
    def __init__(self, doclist, labellist):
        self.DocList = doclist
        self.LabelList = labellist

    def __iter__(self):
        for idx, doc in enumerate(self.DocList):
            yield gensim.models.doc2vec.TaggedDocument(doc.split(), [self.LabelList[idx]])
\end{lstlisting}

Doc2vec sẽ học cách thể hiện đồng thời cho từ và nhãn.
Một điều lưu ý là quá trình huấn luyện của doc2vec có tỷ lệ học giảm dần qua tiến trình lặp qua dữ liệu, nhãn.
Để khắc phục điều này, tỷ lệ học cần phải được điều chỉnh một cách thủ công và được thực thi trong đồ án như sau:
\label{code:training}
\begin{lstlisting}[language=Python]
model = Doc2Vec(size=100, window=10, min_count=1, workers=4, alpha=0.025, min_alpha=0.025)
    model.build_vocab(documents)
    for epoch in range(10):
        model.train(documents)
        model.alpha -= 0.002
        model.min_alpha = model.alpha
\end{lstlisting}

Ý nghĩa của các tham số của doc2vec trong phần mã nguồn trên:
\begin{enumerate}
\item[•]$documents$: là dãy bao gồm các phần tử TaggedDocuments.
\item[•]$dm$: cách huấn luyện mặc định, có giá trị mặc định là 1.
\item[•]$size$: số chiều được thiết lập cho vector.
\item[•]$window$: là khoảng cách tối đa giữa từ được dự đoán và từ được sử dụng trong ngữ cảnh dự đoán trong cùng một văn bản.
\item[•]$alpha$: giá trị khởi tạo cho tỷ lệ học.
\item[•]$workers$:  số lượng luồng được sử dụng để huấn luyện doc2vec.
\end{enumerate}

Sau khi huấn luyện xong, ta có thể trích xuất vector của văn bản dựa vào nhãn của từng văn bản.
\begin{lstlisting}[language=Python]
model.docvecs[label]
\end{lstlisting}

Đồ án sẽ huấn luyện doc2vec cho tập dữ liệu tiếng Anh \ref{sec:dlta} và tiếng Việt \ref{sec:dltv}.
Trong đó, tập dữ liệu tiếng Việt sẽ bao gồm 2 phần: tập huấn luyện và tập kiểm nghiệm kết quả.
Khi huấn luyện doc2vec cho phần dữ liệu tiếng Việt, ta sẽ trộn 2 tập này thành dữ liệu đầu vào của thuật toán.
Các tham số được sử dụng cũng như là giá trị của một số tham số sẽ được thiết lập như trong phần mã nguồn \ref{code:training}.
Sau đó, đồ án sẽ trích xuất những vector từ mô hình doc2vec dựa vào nhãn của văn bản, riêng đối với dữ liệu tiếng Việt thì chỉ trích xuất từ tập kiểm nghiệm kết quả.
Cuối cùng, những thể hiện vector có số chiều là 100 sẽ được đem đi gom nhóm phân cấp kết hợp.

Thuật toán gom nhóm phân cấp kết hợp sử dụng trong đồ án đến từ thư viện scikit-learn~\cite{hac-scikit}.

\begin{lstlisting}[language=Python]
AgglomerativeClustering(n_clusters=2, affinity=euclidean, memory=None, connectivity=None, compute_full_tree=auto, linkage=ward, pooling_func=<function mean>)
\end{lstlisting}
\label{ahc-scikit}
%\textit{AgglomerativeClustering(n{\_}clusters=2, affinity=`euclidean', \\
%memory=None, connectivity=None, compute{\_}full{\_}tree=`auto', \\
%linkage=`ward', pooling{\_}func=<function mean>)}

Ý nghĩa của các tham số trong thuật toán gom nhóm phân cấp kết hợp:
\begin{enumerate}
\item[•]$n{\_}clusters$: số lượng nhóm cần phải tìm, có giá trị mặc định là 2.
\item[•]$affinity$: công thức để tính khoảng cách giữa 2 văn bản, có nhiều công thức khác nhau: ``euclidean'', ``l1'', ``l2'', ``manhattan'', ``cosine'' và ``precomputed''.
Trong trường hợp cách liên kết ``ward'' được sử dụng thì chỉ có thể kết hợp với công thức tính khoảng cách ``euclidean''.
\item[•]$memory$: thể hiện của lớp sklearn.externals.joblib.Memory hoặc là chuỗi, có giá trị mặc định là \textit{None}.
Memory được sử dụng để lưu trữ cục bộ giá trị đầu ra của cây phân cấp.
Mặc định thì giá trị này không được lưu.
Trong trường hợp giá trị của memory là \textit{string} thì chính là đường dẫn đến thư mục lưu trữ cục bộ.
\item[•]$connectivity$: tham số tùy chọn, là tham số dạng mảng hoặc là hàm.
Connectivity matrix là ma trận kết nối dùng để định nghĩa mỗi mẫu thì có mẫu lân cận theo cấu trúc dữ liệu có sẵn.
Ma trận kết nối có thể là chính bản thân nó hoặc là dùng để chuyển hóa dữ liệu vào nó như dẫn xuất từ kneighbors{\_}graph.
Giá trị mặc định của tham số này là \textit{None}.
\item[•]$compute{\_}full{\_}tree$: có giá trị bool hoặc là ``auto'' và là tham số tùy chọn.
Tham số này có tác dụng kết thúc sớm việc tạo cây phân cấp tại nhóm thứ \textit{n{\_}clusters}.
Điều này hữu dụng với việc giảm thời gian tính toán và nếu kết hợp với tham số ma trận kết nối.
\item[•]$linkage$: cách thức liên kết trong gom nhóm phân cấp kết hợp với các tùy chọn như sau: {``ward'', ``complete'', ``average''} và cách liên kết mặc định là ``ward''.
\item[•]$pooling{\_}func$: hàm, có giá trị mặc định là \textit{np.mean}.
Tham số này kết hợp giá trị của đặc trưng tích tụ thành giá trị duy nhất.
Ta có mảng [M, N] và tham số này là \textit{axis=1} thì mảng sẽ giảm xuống còn [M].
\end{enumerate}

Đồ án sử dụng gom nhóm phân cấp kết hợp như đã nêu ở phần \textit{AgglomerativeClustering} \ref{ahc-scikit} với các tham số $n{\_}clusters$, $linkage$, $affinity$.
Trong đó, giá trị $n{\_}clusters$ sẽ được thiết lập tùy vào dữ liệu đầu vào.
Tiếp đến, đồ án sẽ tiến hành gom nhóm phân cấp kết hợp cho cả 2 mô hình doc2vec và TF-IDF.
Sau khi kết thúc gom nhóm, đồ án sử dụng 2 chỉ số \ref{sec:NMI} và \ref{sec:ARI} để so sánh kết quả gom nhóm của doc2vec với TFIDF.
Chi tiết phần thực nghiệm sẽ được nói rõ ở phần \ref{Chapter4}.
%Dựa vào phân tích về tốc độ thực thi cũng như là ảnh hưởng đến lưu trữ bộ nhớ của gom nhóm phân cấp, ta có thể tiến hành cải thiện thuật toán dựa vào 2 hướng tiếp cận này.
%Đồ án tập trung vào cải thiện bộ nhớ lưu trữ trong quá trình gom nhóm phân cấp.
%Như đã đề cập ở trên, dung lượng cần thiết để chạy chương trình là $O(m^2)$, với m là số lượng điểm trong dữ liệu.
%Tuy nhiên, các thực nghiệm chạy trong các ví dụ trên chỉ gồm các điểm có 2 chiều.
%Nhưng trên thức tế, các văn bản để gom nhóm thường rất lớn, nên dung lương lúc chạy thuật toán là rất lớn.
%
%Mục tiêu của đồ án là gom nhóm văn bản tin tức tiếng Việt nên ngữ liệu sẽ rất lớn.
%Do văn bản là từ ngữ sẽ gây khó khăn khi tính toán nên ta sẽ phải chuyển đồi thể hiện từ từ ngữ sang số.
%Ta sẽ chuyển đổi mỗi văn bản thành định dạng vector, với chiều của vector tương ứng với số lượng từ ngữ trong ngữ liệu.
%Điều này đồng nghĩa chiều của vector sẽ phụ thuộc vào số lương từ có trong ngữ liệu.
%Nếu số lượng từ càng nhiều thì chiều của văn bản càng lớn, điều này dẫn lớn lúc chạy thuật toán gom nhóm phân cấp thì dung lượng sẽ rất lớn.
%
%Việc sử dụng thể hiện của văn bản với vector có số chiều tương ứng với số lượng từ trong ngữ liệu sẽ khiến cho thuật toán bị giới hạn do tốn quá nhiều dung lượng bộ nhớ.
%Vì vậy, ta có thể thay đổi thể hiện của văn bản chuyển từ sử dụng tần số sang doc2vec.
%Doc2vec là thuật toán không giám sát để tạo ra vector cho câu, đoạn văn hoặc là văn bản.
%Thuật toán là phiên bản tương thích với word2vec, dùng để tạo ra vector cho từ.
%
%Vector được tạo bởi doc2vec thường được sử dụng cho các nhiệm vụ như tìm độ tương đồng giữa câu, đoạn văn và văn bản.
%Không như các mô hình câu như RNN, các chuỗi từ được giữ lại trong quá trình tạo ra vector câu, doc2vec độc lập với thứ tự từ.
%Trong nhiệm vụ tìm kiếm độ tương đồng, doc2vec biểu diễn định dạng của văn bản rất tốt vì giới hạn được số lượng chiều.
%Vì vậy, doc2vec được chọn để biểu diễn văn bản cho thuật toán gom nhóm phân cấp.
%
%Do vector biểu diễn bằng doc2vec có số chiều nhỏ hơn rất nhiều so với vector biểu diễn tần số, dung lượng bộ nhớ khi sử dụng thuật toán gom nhóm phân cấp sẽ được giảm đáng kể.
%Vector biểu diễn bằng tần số có số chiều tương ứng với số lượng từ trong ngữ liệu.
%Trong khi đó, số chiều trong doc2vec được thiết lập cố định.
%Vì doc2vec là thuật toán dùng để tìm kiếm thể hiện cho văn bản nên ta phải thiết lập cố định số chiều lúc huấn luyện.
%Chính vì điều này nên doc2vec có số chiều nhỏ hơn so với tần số và sẽ giúp cho dung lượng bộ nhớ giảm đi nhiều khi thực thi chương trình gom nhóm phân cấp.

%Đây là hướng tiếp cận kinh điển trong việc thể hiện văn bản, có tên gọi là TFIDF (term-frequecny - inverse frequency document).
%TFIDF là một dạng thống kê số học 



%Phân tích điểm hạn chế của thuật toán(2-3 đoạn)

%Biến liên tục

%Biến rời rạc

%Độ đo kết hợp giữa hai biến rời rạc

%Các độ đo kết hợp cho biến rời rạc(2-3)

%Độ đo kết hợp giữa hai biến liên tục

%Các độ đo kêt hợp cho biến liên tục(2-3)

%Goodman kruskal chỉ áp dụng cho biến rời rạc

%Tìm cách cải thiện thuật toán(1 - 2)

%Sử dụng doc2vec cho thể hiện văn bản

%Tìm kiếm công thức tính khoảng cách tốt nhất có thể (8-10)

%Áp dụng vào thuật toán hiện hành

%kết quả

%%23-29

\chapter{Thực nghiệm và kết quả}
\label{Chapter4}
Chương này trình bày môi trường thực nghiệm của đồ án, các công cụ hỗ trợ thực nghiệm, các tập ngữ liệu được sử dụng, các công cụ dùng để đánh giá kết quả gom nhóm và kết quả của thực nghiệm.

\section{Thực nghiệm}
\subsection{Môi trường thực nghiệm}
%Môi trường thực nghiệm
Quá trình thực nghiệm đồ án được thực hiện với môi trường như sau:
\begin{enumerate}
\item[•]Ngôn ngữ lập trình Python phiên bản 2.7
\item[•]Một số gói lệnh Python cài đặt thêm:
\begin{enumerate}
\item[-] gensim phiên bản 0.11.1: gói lệnh hỗ trợ trích xuất ngữ nghĩa của các chủ đề trong các văn bản.
Gensim được thiết kế để có thể thực thi với dữ liệu thô.
Các thuật toán trong gensim như Latent Semantics Analysis(LSA), Latent Dirichlet Allocation(LDA) và Random Projections dùng để khai thác cấu trúc ngữ nghĩa của văn bản bằng các mẫu thống kê.
Sau khi chạy các thuật toán này, văn bản thô sẽ chuyển sang dạng thể hiện mới có nghĩa và có thể truy vấn độ tương đồng với các văn bản khác.
Gensim là gói lệnh dùng để giải quyết các ý tưởng về corpus, vector và mô hình.
\item[-] sklearn phiên bản 0.16.1: gói lệnh hỗ trợ máy học trong Python.
Đây là công cụ đơn giản, hiệu quả cho khai thác và phân tích dữ liệu.
Các thuật toán trong sklearn là công cụ hỗ trợ rất tốt cho các đề tài nghiên cứu.
Sklearn là mã nguồn mở, có thể được sử dụng trong thương mại.
Vì vậy, mọi người đều có thể sử dụng gói lệnh và chỉnh sửa thoải mái.
\end{enumerate}
\item[•]Ngôn ngữ lập trình C++.
\item[•]Máy tính chạy thực nghiệm
\begin{enumerate}
\item[-]CPU core i5 3.3GHz
\item[-]RAM 8GB
\item[-]Hệ điều hành windows
\end{enumerate}
\end{enumerate}

\subsection{Công cụ hỗ trợ}
\begin{table}[ht]
\begin{center}
\begin{tabularx}{\textwidth}{|c|c|X|}
\hline
STT & Tên công cụ & \makecell[c]{Mô tả} \\
\hline
1 & Google News & Công dụng: dùng để lọc ra những bài báo đã được gom nhóm. \\
\hline
2 & Java Tokenizer & Tác giả: Lê Hồng Phương - Đại học Khoa Học Tự nhiên - Đại học Quốc Gia Hà Nội. \newline Công dụng: Tách câu và tách từ cho tiếng Việt.\\
\hline
3 & sgmllib & Hỗ trợ: thư viện của Python 2.7. \newline Công dụng: dùng để trích xuất dữ liệu dưới định dạng SGML.\\
\hline
4 & C++ editor & Microsoft Visual Studio C++ 2008 \\
\hline

\end{tabularx}
\caption[Các công cụ hỗ trợ thực nghiệm]{Các công cụ hỗ trợ thực nghiệm}
\label{bang_4_1}
\end{center}
\end{table}
%\floatbarrier

\section{Dữ liệu}
%Giới thiệu dữ liệu sử dụng
Dữ liệu sử dụng trong chương trình bao gồm dữ liệu tiếng Anh và dữ liệu tiếng Việt.
Trong đó, bộ dữ liệu tiếng Việt được tổng hợp từ các trang tin tức nổi tiếng của Việt Nam như vnexpress, dân trí, tuổi trẻ, \ldots 
Các bài báo của các bộ dữ liệu được thu nhặt được thông qua trang tin tức tổng hợp của Google.
Còn bộ dữ liệu tiếng Anh là các bài báo được lấy từ trang tin tức Reuters được tổng hợp lại thành Reuters-21578.
Ngoài ra, cả hai bộ dữ liệu đều là các bài báo đã được gom nhóm sơ bộ.

%giới thiệu dữ liệu tiếng Anh
\subsection{Dữ liệu tiếng Anh}
% Giới thiệu Reuters-21578
Reuters-21578 bao gồm 21,578 bài báo thuộc nhiều lĩnh vực khác nhau.
Bộ dữ liệu này được tổng hợp bởi David D. Lewis vào năm 1987.
Đây là nguồn tài nguyên cho mục đích nghiên cứu trong lĩnh vực truy vấn thông tin, máy học, và những nghiên cứu dựa vào corpus khác.
Bộ dữ liệu này xuất bản và phân phối miễn phí cho mục đích nghiên cứu.
Nếu người nào muốn sử dụng bộ dữ liệu này thì phải đề cập đến nguồn~\footnote{http://www.research.att.com/~lewis} cũng như là thông tin tên của bộ dữ liệu cho mọi người biết.

% Mô tả Reuters-21578
Reuters-21578 là tập dữ liệu bao gồm hai mươi hai tập tin dữ liệu.
Mỗi tập tin dữ liệu là có định dạng là SGML(Standard Generalized Markup Language) và có khoảng 1000 bài báo.
Ngoài ra, bộ dữ liệu còn có sáu tập tin mô tả phân loại dùng để chỉ mục cho dữ liệu.
Tập dữ liệu còn có thêm vào những tập tin dựa vào đóng góp của những nhà nghiên cứu khác.
Tất cả những tập tin đều là không có nén, được gom lại vào thành một thư mục được nén lại thành cái tên reuters21578.tar.gz

%Giới thiệu sơ bộ về SGML
SGML~\footnote{https://www.w3.org/TR/WD-html40-970708/intro/sgmltut.html} là chuẩn cho việc định nghĩa ngôn ngữ đánh dấu cho văn bản, HTML là một trong những ứng dụng của SGML.
SGML bao gồm phần khai báo, định nghĩa loại văn bản, đặc tả và thể hiện văn bản đánh dấu.
Phần khai báo SGML là chỉ ra các kí tự và các loại dấu phân cách sẽ sử dụng.
Phần định nghĩa loại văn bản định nghĩa cú pháp của cấu trúc đánh dấu và có thể có thêm các phần định nghĩa khác như là số và tên kí tự.
Phần đặc tả mô tả ngữ nghĩa được tạo bởi đánh dấu và phần đặc tả này ràng buộc sự giới hạn của cú pháp để không thể xuất hiện trong phần định nghĩa loại văn bản.
Thể hiện văn bản chứa nội dung và đánh dấu, mỗi thể hiện chứa tham chiếu đến phần định nghĩa văn bản để thực thi nó.

%  phần phân mục của Reuters-21578
Như đã đề cập, các bài báo trong reuters-21578 đã được gom nhóm lại thành các tập phân nhóm khác nhau
\begin{table}[ht]
\begin{center}
\begin{tabularx}{\textwidth}{|Y|Y|Y|Y|}
\hline
Tập phân loại & Số lượng phân phân nhóm & Số lượng phân nhóm mà tần số lớn hơn một & Số lượng phân nhóm mà tần số lớn hơn 20  \\
\hline
EXCHANGES & 39 & 32 & 7\\
\hline
ORGS & 56 & 32 & 9\\
\hline
PEOPLE & 267 & 114 & 15 \\
\hline
PLACES & 175 & 147 & 60\\
\hline
TOPICS & 135 & 120 & 57\\
\hline
\end{tabularx}
\caption[Các tập phân loại]{Các tập phân loại}
\label{bang_4_2}
\end{center}
\end{table}

%Các tập phân loại
Phần phân loại TOPICS bao gồm các chủ đề về kinh tế.
Ví dụ phần này sẽ bao gồm các từ chính như "include", "gold", "inventories", và "money-supply".
Tập phân loại này là tập đã được sử dụng hầu hết trong các nghiên cứu về dữ liệu của Reuters.
Các tập phân loại như EXCHANGES, ORGS, PEOPLE, và PLACES được xếp nhóm có tên như mô tả, ví dụ như "nasdaq"(EXCHANGES), "gatt"(ORGS), "perez-de-cuellar"(PEOPLE), và "australia"(PLACES).

%giới thiệu dữ liệu tiếng Việt
\subsection{Dữ liệu tiếng Việt}
Bộ dữ liệu tiếng Việt do Ung Văn Giàu lấy thủ công từ Google News bao gồm 1945 bài có tổng cộng là 300 phân nhóm.
Các bài báo này đều đến từ các trang báo như: vnexpress, dân trí, tuổi trẻ, \ldots.
Bộ dữ liệu có 300 phân nhóm là do bạn Giàu gán nhãn những bài vào các phân nhóm.
Mỗi phân nhóm có ít nhất là 5 bài, nhiều nhất là 10 bài.
Do đây là bộ dữ liệu làm bằng thủ công nên để đảm bảo tính chính xác thì em phân nhóm lại một lần nữa. 
Việc có hai bạn độc lập kiểm tra bộ dữ liệu nên đây là bộ dữ liệu đã được qua kiểm chứng và có thể sử dụng cho mục đích nghiên cứu.


%Các bước chuẩn bị cần thiết khi sử dụng dữ liệu

\section{Các phương pháp đánh giá}
%Các phương pháp đánh giá

	%Giới thiệu các cách đánh giá
	\subsection{Giới thiệu phương pháp đánh giá}
	Để đánh giá kết quả gom nhóm văn bản, ta có hai loại chỉ số để sử dụng: chỉ số ngoại vi và chỉ số nội tại.
	Chỉ số nội tại dùng để đo độ tốt của cấu trúc gom nhóm không cần thông tin ngoài.
	Chỉ số ngoại vi dùng dùng để đo độ tương đồng giữa hai phân nhóm.
	Trong đó, phân nhóm thứ nhất là cấu trúc gom nhóm gốc đã được biết.
	Còn phân nhóm thứ hai là kết quả từ quá trình gom nhóm.
	Trong bài toán, ta sử dụng hai chỉ số đánh giá ngoại vi là : NMI(normalized mutual information) và ARI (adjusted rand index).
	
	Như đã đề cập ở phần trên, ta sẽ sử dụng hai chỉ số ngoại vi để đánh giá.
	Ta có tập $\textbf{C} \, = {C_1 \ldots C_j}$ là tập phân nhóm của đối tượng được xây dựng ở một cấp độ nhất định.
	Tập $\textbf{P} \, = {P_1 \ldots P_j}$ là tập hợp được chia bởi phân lớp ban đầu. $J$ và $I$ là tương đương với số phân nhóm của $(|\textbf{C}|)$ và số phân lớp của  $(|\textbf{P}|)$.
	Ta biểu diễn $n$ là tổng số đối tượng trong thuật toán.

	%Cách đánh giá NMI
	\subsection{Cách đánh giá NMI}
		%Giới thiệu
		\subsubsection{Giới thiệu}
		NMI có nguốn gốc từ MI(mutual information), được sử dụng nhiều trong lý thuyết xác suất và lý thuyết thông tin.
		NMI là phương pháp đo độ phụ thuộc lẫn nhau giữa hai biến.
		Trong đây, NMI được nâng cấp để đo phụ thuộc lẫn nhau giữa hai nhóm.
		Từ đó, NMI cung cấp thông tin cân bằng liên quan đến số lượng phân nhóm.
		Ngoài ra, NMI còn cho ra kết quả chia sẻ thông tin với lớp thực sự được gán và thông tin hỗn hợp trung bình giữa những cặp của phân nhóm và phân lớp.
		
		%Công thức
		\subsubsection{Công thức}
		\begin{center}
		\begin{equation}
			\textbf{NMI} \, = \frac{\sum^I_{i=1} \sum^J_{j=1} x_{ij} \log \frac{n x_{ij}}{x_i x_j}}{\sqrt{\sum^I_{i=1} x_i \log \frac{x_i}{n} \sum^J_{j=1} x_j \log \frac{x_j}{n}}}
		\end{equation}
		\end{center}
		
		Với $x_{ij}$ là số lượng phần tử của các đối tượng mà xuất hiện trong cả tập $C_j$ và $P_i$. $x_j$ là số lượng phần tử chỉ xuất hiện trong tập $C_j$. $x_i$ là số lượng phần tử chỉ xuất hiện trong tập $P_i$. Giá trị của chỉ số này nằm trong khoảng từ 0 đến 1.

		%Ví dụ
		\subsubsection{Ví dụ}
		
	%Cách đánh giá ARI
	\subsection{Cách đánh giá ARI}
		%Giới thiệu
		\subsubsection{Giới thiệu}
		ARI có nguồn gốc từ RI (rand index), được sử dụng thống kê và gom nhóm dữ liệu.
		RI dùng để đo độ tương đồng giữa các nhóm dữ liệu.
		Vấn đề của RI là giá trị mong muốn của hai phân nhóm ngẫu nhiên nằm trong khoảng từ $0$ và $1$.
		Vì vậy, ARI ra đời là phiên bản chỉnh sửa có thể định nghĩa cho việc điều chỉnh cho cơ hội gom nhóm các thành phần.
		Giá trị của ARI có thể nằm trong phạm vi từ -1 đến 1.

		%Công thức
		\subsubsection{Công thức}
		\begin{center}
		\begin{equation}
			E[\alpha] \, = \frac{\pi(C) \cdot \pi(P)}{n(n - 1) / 2}
		\end{equation}
		\end{center}
		
		Với $\pi(C)$ và $\pi(P)$ biểu thị tương ứng với số lượng các cặp đối tượng của cùng phân nhóm trong $	\textbf{C}$ và cùng phân lớp trong $\textbf{P}$. Giá trị lớn nhất cho $\alpha$ có thể đạt được là:
		\begin{center}
		\begin{equation}
			\max(\alpha) = \frac{1}{2} (\pi(C) + \pi(P))
		\end{equation}
		\end{center}
		
		Độ tương đồng giữa $\textbf{C}$ và $\textbf{P}$ có thể được ước lượng bởi adjusted rand index như sau:
		\begin{center}
		\begin{equation}
			AR(\textbf{C}, \textbf{P}) = \frac{\alpha - E[\alpha]}{\max(\alpha) - E[\alpha]}
		\end{equation}
		\end{center}

		%Ví dụ
		\subsubsection{Ví dụ}

\section{Kết quả}
%Thực nghiệm kết quả

%Giải thích chi tiết(5 đoạn)

%%18




\chapter{Kết luận và hướng phát triển}
\label{Chapter5}
\section{Kết luận}
Thuật toán gom nhóm phân cấp kết hợp khi sử dụng văn bản với thể hiện vector của mô hình doc2vec cho ra kết quả tốt hơn so với sử dụng TFIDF.
Kết quả của gom nhóm tiếng Việt khi sử dụng mô hình doc2vec có chỉ số ARI cao hơn so với TFIDF nhưng chỉ số NMI thì ngược lại.
Nhìn chung, kết quả gom nhóm như vậy là khả quan.
Điều quan trọng là khi sử dụng mô hình doc2vec thì chiều của vector để thể hiện văn bản đã giảm đi rất nhiều.
Việc số chiều giảm đã đem lại các lợi ích sau cho gom nhóm phân cấp kết hợp:
\begin{enumerate}
\item[•]Số chiều của vector giảm đồng nghĩa với dung lượng để lưu trữ lúc thực thi cũng giảm theo.
Từ đó, dữ liệu dành cho gom nhóm phân cấp có thể được mở rộng lớn hơn.
\item[•]Khi thuật toán thực hiện các phép tính trên vector của mô hình doc2vec, số chiều của doc2vec thấp hơn rất nhiều so với TFIDF nên kết quả tính toán sẽ nhanh hơn và giúp cho thuật toán thực thi nhanh hơn.
\end{enumerate}

Như vậy, khi gom nhóm phân cấp kết hợp cho văn bản có thể hiện vector theo mô hình doc2vec không những giúp cho việc giảm dung lượng bộ nhớ lúc chạy thuật toán mà còn có thể tăng nhanh quá trình thực thi.
Không những vậy, doc2vec còn cho ra kết quả tốt hơn so với TFIDF.
Vì vậy, doc2vec là mô hình thích hợp để biểu diễn vector cho văn bản dùng để gom nhóm.

\section{Hướng phát triển}
Hiện tại, vấn đề về quá tải dung lượng bộ nhớ lúc chạy thuật gom nhóm phân cấp đã được khắc phục khi sử dụng doc2vec.
Đồng thời, doc2vec cũng giúp cho thời gian thực thi thuật toán được giảm xuống.
Tuy nhiên, ta vẫn còn tồn tại điểm yếu về tìm kiếm vết cắt trên cây phân cấp như đã đề cập trong phần khuyết điểm ở \ref{sec:gnvbpc}.
Chính vì vậy mà lúc thực thi, ta phải truyền giá trị số lượng nhóm cho tham số của thuật toán.
Nếu như thuật toán có thể tìm kiếm vết cắt thích hợp thì ta có thể biết được số lượng nhóm tiềm năng trong ngữ liệu.
Vì vậy, trong tương lai, công trình sẽ tiếp tục nghiên cứu việc tìm kiếm vết cắt phù hợp để có thể thấy được số lượng nhóm trong dữ liệu là bao nhiêu.

%kết quả đạt được (1-3 đoạn)

%đóng góp mới cho bài toán
	%sử dụng Doc2vec
	%sử dụng biến liên tục

%Kiến nghị, đề xuất cho hướng nghiên cứu tiếp theo

%%(4-5)

% Công trình của tác giả (nếu không có thì comment 02 dòng dưới)
\addcontentsline{toc}{chapter}{Danh mục công trình của tác giả}
\chapter*{Danh mục công trình của tác giả}
\label{Appendix1}

\begin{enumerate}
\item Tạp chí ABC
\item Tạp chí XYZ
\end{enumerate}

% In tài liệu tham khảo
\addcontentsline{toc}{chapter}{Tài liệu tham khảo}
\printbibheading[title={Tài liệu tham khảo}]

\printbibliography[heading=subbibliography, title={Tiếng Việt}, keyword=Viet, resetnumbers=true]

\DeclareNameAlias{sortname}{last-first}
\DeclareNameAlias{default}{last-first}

\printbibliography[heading=subbibliography, title={Tiếng Anh}, notkeyword=Viet, resetnumbers=4] 
% ===================================================================== %
% CHÚ Ý: phải gán lại resetnumbers=số tài liệu tham khảo tiếng Việt + 1 %
% ===================================================================== %

% Phần phụ lục
\appendix

\chapter{Ngữ pháp tiếng Việt}
\label{Appendix1}

Đây là phụ lục.
\chapter{Ngữ pháp tiếng Nôm}
\label{Appendix2}

Đây là phụ lục 2.

\end{document} 