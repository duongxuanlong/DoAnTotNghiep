\chapter{Giới thiệu}
\label{Chapter1}

\section{Giới thiệu về gom nhóm văn bản}

%Gom nhóm là gì?
Gom nhóm là công việc tìm kiếm và chia những đối tượng gần giống nhau vào thành những nhóm trong dữ liệu sao cho nhũng nhóm này có ý nghĩa, hữu dụng hoặc cả hai. %\cite{1984-TeX-Knuth}
Nếu như mục đích của gom nhóm là tìm kiếm ngữ nghĩa của dữ liệu thì các nhóm được gom sẽ thể hiện cấu trúc của dữ liệu.
Trong một vài trường hợp, gom nhóm được xem là cầu nối để thực hiện cho những mục đích khác như tóm tắt dữ liệu.
Bất kể việc sử dụng gom nhóm cho mục đích ngữ nghĩa hay là hữu dụng thì gom nhóm đóng vai trò quan trọng trong nhiều lĩnh vực khác nhau như: tâm lý học và các ngành khoa học xã hội, sinh học, thống kê, nhận diện mô hình, truy vấn thông tin, máy học và khai thác dữ liệu.

Gom nhóm văn bản là một ứng dụng của gom nhóm.
Trong đó, các văn bản gần tương đồng sẽ được gom chung với nhau thành nhóm riêng biệt theo cách thức không giám sát.
Các văn bản gần liên quan với nhau có độ tương đồng lớn hơn so với những văn bản dị biệt.
Như vậy, những văn bản tương đồng này có thể tạo thành nhóm có cùng một chủ để.
Các nhóm này được hình thành từ quá trình gom nhóm không giám sát nên giúp cho chúng ta thấy được cấu trúc của ngữ liệu.

Ngày nay, việc tìm kiếm thông tin trên mạng đã trở thành kỹ năng cần thiết cho bất kì ai.
Tuy nhiên, khi ta truy vấn thông tin với từ khóa bất kì thì có thể cho ra quá nhiều kết quả khác nhau.
Vì vậy, ta cần tổ chức lại thông tin cần truy vấn thành cấu trúc phù hợp.
Gom nhóm văn bản có thể giúp chúng ta sắp xếp các văn bản thành những chủ để phù hợp và giúp cho việc truy vấn dễ dàng hơn.

Gom nhóm văn bản tự động tạo ra những nhóm có văn bản liên quan với nhau mà không cần phải tạo trước tập huấn luyện hay nguyên tắc phân loại.
Ngoài ra gom nhóm văn bản dựa vào độ tương đồng của nội dung để cải thiện hiệu quả tìm kiếm.
\begin{enumerate}
\item[•]Cải thiện hồi quy trong tìm kiếm: khi câu truy vấn trùng với một văn bản thì kết quả trả về có thể bao gồm toàn bộ nhóm chứa văn bản.
\item[•]Cải thiện độ chính xác trong tìm kiếm: gom nhóm văn bản thành những nhóm nhỏ hơn trong nhóm của những văn bản liên quan.
\item[•]Phân tán hoặc tập hợp: khi một câu truy vấn không thể được công thức hóa thì ta có thể cho phép người dùng duyệt văn bản theo nhóm.
\item[•]Gom nhóm theo truy vấn đặc tả: những văn bản gần liên quan đến câu truy vấn nhất sẽ nằm trong nhóm con lồng vào bên trong nhóm lớn hơn có độ tương đồng thấp hơn.
\end{enumerate}

Nghiên cứu của vấn đề gom nhóm đứng trước tính khả dụng khi ứng dụng vào văn bản. Các phương pháp truyền thống cho gom nhóm thường tập trung vào dữ liệu lớn, khi thuộc tính của dữ liệu là số. Vấn đề này cũng được nghiên cứu trong phân loại dữ liệu, khi mà thuộc tính có giá trị nặc danh. Vấn đề của gom nhóm là tìm được tính khả dụng trong các nhiệm vụ sau:
\begin{enumerate}
\item[•]Duyệt và tổ chức văn bản: tổ chức phân cấp của văn bản vào trong các hạng mục mạch lạc. Điều này có thể giúp ích cho việc duyệt hệ thống của tập hợp văn bản. Ví dụ kinh kiển cho phương pháp này là phân tán hoặc tập hợp. Phương pháp này cung cấp kỹ thuật duyệt hệ thống với sử dụng gom nhóm tổ chức của tập hợp văn bản.
\item[•]Tóm tắt ngữ liệu: kỹ thuật gom nhóm cung cấp tóm tắt mạch lạc của tập hợp trong dạng nhóm tài tiệu hoặc nhóm từ. Điều này được sử dụng đề cung cấp tóm tắt trong phần nội dung tổng kết của ngữ liệu căn bản. Lĩnh vực này có nhiều phương pháp, đặc biệt là gom nhóm câu dùng để tóm tắt văn bản. Vấn đề của gom nhóm liên quan đến việc giảm số chiều và mô hình hóa chủ đề. 
\item[•]Phân loại văn bản : Gom nhóm là phương pháp học không giám sát. Nó thể được đòn bẩy hóa để cải thiện chất lượng kết quả trong giám sát. Cụ thể, các nhóm từ và phương thức đồng huấn luyện có thể được sử dụng để cải thiện độ chính xác phân loại của ứng dụng giám sát với tác dụng của kỹ thuật gom nhóm.
\end{enumerate}

\section{Bài toán gom nhóm văn bản}
Gom nhóm là quá trình gom các đối tượng dữ liệu dựa vào thông tin được tìm thấy trong dữ liệu mô tả đối tượng và quan hệ giữa các đối tượng.
Mục tiêu của gom nhóm là những đối tượng gần giống nhau (tương đồng) sẽ ở trong cùng một nhóm so với đối tượng dị biệt sẽ ở trong nhóm khác.
Độ tương đồng càng lớn trong một nhóm kết hợp với độ khác biệt sẽ càng lớn giữa các nhóm có thể tạo ra khoảng cách càng lớn để kết nối giữa các nhóm.

Trong nhiều ứng dụng, ý tưởng để tạo thành một nhóm không được định nghĩa rõ.
Vì vậy, ta cần hiểu hiểu rõ khó khăn của quá trình tạo thành một nhóm.
Nhìn vào hình \ref{fig:pic11}, ta thấy có 20 điểm nhưng lại có đến 3 cách khác nhau để gom nhóm.
Hình dạng của dữ liệu sẽ chỉ dẫn mối quan hệ giữa các nhóm.
Hình \ref{fig:pic11}(a) và hình \ref{fig:pic11}(b) chia dữ liệu thành 2 nhóm khác nhau với mỗi nhóm gồm 6 điểm.
Tuy nhiên, ta có thể nhìn thấy là ở mỗi nhóm lớn ta còn có thể chia nhỏ tiếp được thành 2 nhóm nhỏ hơn.
Điều này mang tính chất chủ quan dựa vào cách nhìn của mỗi người và có thể tạo thành 4 nhóm khác nhau như hình \ref{fig:pic11}(c).
Hình \ref{fig:pic11} cho ta thấy định nghĩa một nhóm không có đúng tuyệt đối và cách tiếp cận tốt nhất là ta phải dựa vào bản chất của dữ liệu và kết quả mong muốn.

\begin{figure}[htp]
\makeatletter % For spaces in paths
\patchcmd\Gread@eps{\@inputcheck#1 }{\@inputcheck"#1"\relax}{}{}
\makeatother
\psscalebox{1.0 1.0} % Change this value to rescale the drawing.
{
\begin{pspicture}(0,-3.465664)(13.02909,3.465664)
\definecolor{colour0}{rgb}{0.0,0.0,0.4}
\definecolor{colour1}{rgb}{0.0,0.4,0.4}
\definecolor{colour2}{rgb}{0.4,0.0,0.4}
\definecolor{colour3}{rgb}{0.4,0.4,0.0}
\psdots[linecolor=black, dotsize=0.2](0.51454985,2.934336)
\psdots[linecolor=black, dotsize=0.2](0.9145499,3.334336)
\psdots[linecolor=black, dotsize=0.2](0.9145499,2.934336)
\psdots[linecolor=black, dotsize=0.2](0.9145499,2.934336)
\psdots[linecolor=black, dotsize=0.2](1.3145499,3.334336)
\psdots[linecolor=black, dotsize=0.2](1.7145499,2.934336)
\psdots[linecolor=black, dotsize=0.2](1.7145499,2.534336)
\psdots[linecolor=black, dotsize=0.2](2.1145499,2.934336)
\psdots[linecolor=black, dotsize=0.2](0.51454985,1.734336)
\psdots[linecolor=black, dotsize=0.2](0.51454985,2.134336)
\psdots[linecolor=black, dotsize=0.2](0.11454987,1.734336)
\psdots[linecolor=black, dotsize=0.2](4.5145497,3.334336)
\psdots[linecolor=black, dotsize=0.2](4.1145496,2.934336)
\psdots[linecolor=black, dotsize=0.2](4.5145497,2.934336)
\psdots[linecolor=black, dotsize=0.2](4.5145497,1.734336)
\psdots[linecolor=black, dotsize=0.2](4.91455,1.734336)
\psdots[linecolor=black, dotsize=0.2](4.1145496,1.3343359)
\psdots[linecolor=black, dotsize=0.2](4.5145497,1.3343359)
\psdots[linecolor=black, dotsize=0.2](5.31455,2.534336)
\psdots[linecolor=black, dotsize=0.2](5.31455,2.134336)
\psdots[linecolor=black, dotsize=0.2](5.71455,2.134336)
\rput[bl](1.3145499,0.534336){(a) Các điểm gốc}
\psdots[linecolor=colour0, dotstyle=square*, dotsize=0.2](7.71455,2.934336)
\psdots[linecolor=colour0, dotstyle=square*, dotsize=0.2](8.11455,3.334336)
\psdots[linecolor=colour0, dotstyle=square*, dotsize=0.2](8.11455,2.934336)
\psdots[linecolor=colour0, dotstyle=square*, dotsize=0.2](8.11455,2.934336)
\psdots[linecolor=colour0, dotstyle=square*, dotsize=0.2](8.51455,3.334336)
\psdots[linecolor=colour0, dotstyle=square*, dotsize=0.2](8.91455,2.934336)
\psdots[linecolor=colour0, dotstyle=square*, dotsize=0.2](8.91455,2.534336)
\psdots[linecolor=colour0, dotstyle=square*, dotsize=0.2](9.314549,2.934336)
\psdots[linecolor=colour0, dotstyle=square*, dotsize=0.2](7.71455,1.734336)
\psdots[linecolor=colour0, dotstyle=square*, dotsize=0.2](7.71455,2.134336)
\psdots[linecolor=colour0, dotstyle=square*, dotsize=0.2](7.31455,1.734336)
\psdots[linecolor=colour1, dotstyle=triangle*, dotsize=0.2](11.71455,3.334336)
\psdots[linecolor=colour1, dotstyle=triangle*, dotsize=0.2](11.314549,2.934336)
\psdots[linecolor=colour1, dotstyle=triangle*, dotsize=0.2](11.71455,2.934336)
\psdots[linecolor=colour1, dotstyle=triangle*, dotsize=0.2](11.71455,1.734336)
\psdots[linecolor=colour1, dotstyle=triangle*, dotsize=0.2](12.11455,1.734336)
\psdots[linecolor=colour1, dotstyle=triangle*, dotsize=0.2](11.314549,1.3343359)
\psdots[linecolor=colour1, dotstyle=triangle*, dotsize=0.2](11.71455,1.3343359)
\psdots[linecolor=colour1, dotstyle=triangle*, dotsize=0.2](12.51455,2.534336)
\psdots[linecolor=colour1, dotstyle=triangle*, dotsize=0.2](12.51455,2.134336)
\psdots[linecolor=colour1, dotstyle=triangle*, dotsize=0.2](12.91455,2.134336)
\rput[bl](8.51455,0.534336){(b) 2 nhóm}
\psdots[linecolor=colour0, dotstyle=+, dotsize=0.2](0.51454985,-1.065664)
\psdots[linecolor=colour0, dotstyle=+, dotsize=0.2](0.9145499,-0.665664)
\psdots[linecolor=colour0, dotstyle=+, dotsize=0.2](0.9145499,-1.065664)
\psdots[linecolor=colour0, dotstyle=+, dotsize=0.2](0.9145499,-1.065664)
\psdots[linecolor=colour0, dotstyle=+, dotsize=0.2](1.3145499,-0.665664)
\psdots[linecolor=colour0, dotstyle=+, dotsize=0.2](1.7145499,-1.065664)
\psdots[linecolor=colour0, dotstyle=+, dotsize=0.2](1.7145499,-1.465664)
\psdots[linecolor=colour0, dotstyle=+, dotsize=0.2](2.1145499,-1.065664)
\psdots[linecolor=colour1, dotstyle=triangle*, dotsize=0.2](0.51454985,-2.265664)
\psdots[linecolor=colour1, dotstyle=triangle*, dotsize=0.2](0.51454985,-1.865664)
\psdots[linecolor=colour1, dotstyle=triangle*, dotsize=0.2](0.11454987,-2.265664)
\psdots[linecolor=colour2, dotstyle=asterisk, dotsize=0.2](4.5145497,-0.665664)
\psdots[linecolor=colour2, dotstyle=asterisk, dotsize=0.2](4.1145496,-1.065664)
\psdots[linecolor=colour2, dotstyle=asterisk, dotsize=0.2](4.5145497,-1.065664)
\psdots[linecolor=colour3, dotstyle=diamond*, dotsize=0.2](4.5145497,-2.265664)
\psdots[linecolor=colour3, dotstyle=diamond*, dotsize=0.2](4.91455,-2.265664)
\psdots[linecolor=colour3, dotstyle=diamond*, dotsize=0.2](4.1145496,-2.665664)
\psdots[linecolor=colour3, dotstyle=diamond*, dotsize=0.2](4.5145497,-2.665664)
\psdots[linecolor=colour3, dotstyle=diamond*, dotsize=0.2](5.31455,-1.465664)
\psdots[linecolor=colour3, dotstyle=diamond*, dotsize=0.2](5.31455,-1.865664)
\psdots[linecolor=colour3, dotstyle=diamond*, dotsize=0.2](5.71455,-1.865664)
\rput[bl](1.3145499,-3.465664){(c) 4 nhóm}
\psdots[linecolor=colour0, dotstyle=oplus, dotsize=0.2](7.71455,-1.065664)
\psdots[linecolor=colour0, dotstyle=oplus, dotsize=0.2](8.11455,-0.665664)
\psdots[linecolor=colour0, dotstyle=oplus, dotsize=0.2](8.11455,-1.065664)
\psdots[linecolor=colour0, dotstyle=oplus, dotsize=0.2](8.11455,-1.065664)
\psdots[linecolor=colour0, dotstyle=oplus, dotsize=0.2](8.51455,-0.665664)
\psdots[linecolor=black, dotsize=0.2](8.91455,-1.065664)
\psdots[linecolor=black, dotsize=0.2](8.91455,-1.465664)
\psdots[linecolor=black, dotsize=0.2](9.314549,-1.065664)
\psdots[linecolor=brown, dotstyle=triangle*, dotsize=0.2](7.71455,-2.265664)
\psdots[linecolor=brown, dotstyle=triangle*, dotsize=0.2](7.71455,-1.865664)
\psdots[linecolor=brown, dotstyle=triangle*, dotsize=0.2](7.31455,-2.265664)
\psdots[linecolor=colour3, dotstyle=asterisk, dotsize=0.2](11.71455,-0.665664)
\psdots[linecolor=colour3, dotstyle=asterisk, dotsize=0.2](11.314549,-1.065664)
\psdots[linecolor=colour3, dotstyle=asterisk, dotsize=0.2](11.71455,-1.065664)
\psdots[linecolor=colour2, dotstyle=diamond*, dotsize=0.2](11.71455,-2.265664)
\psdots[linecolor=colour2, dotstyle=diamond*, dotsize=0.2](12.11455,-2.265664)
\psdots[linecolor=colour2, dotstyle=diamond*, dotsize=0.2](11.314549,-2.665664)
\psdots[linecolor=colour2, dotstyle=diamond*, dotsize=0.2](11.71455,-2.665664)
\psdots[linecolor=colour1, dotstyle=square*, dotsize=0.2](12.51455,-1.465664)
\psdots[linecolor=colour1, dotstyle=square*, dotsize=0.2](12.51455,-1.865664)
\psdots[linecolor=colour1, dotstyle=square*, dotsize=0.2](12.91455,-1.865664)
\rput[bl](8.51455,-3.465664){(d) 6 nhóm}
\end{pspicture}
}
\caption{Những cách khác nhau về gom nhóm trên cùng một tập điểm}
\label{fig:pic11}
\end{figure}

Gom nhóm có thể liên quan đến các kỹ thuật khác như chia dữ liệu vào  các nhóm khác nhau.
Chẳng hạn, gom nhóm có thể liên quan đến một dạng phân lớp mà nó tạo ra nhãn của đối tượng với lớp (hoặc nhóm) nhãn.
Tuy nhiên, gom nhóm dẫn xuất các nhãn này từ dữ liệu.
Trong khi đó, phân lớp lại là phương pháp có tên gọi là phân lớp giám sát, những đối tượng chưa có nhãn sẽ được gán với nhãn của đối tượng biết trước.
Vì vậy, gom nhóm hay còn được biết đến như là phân lớp không có giám sát.

Có thể nói, bài toán gom nhóm là bài toán dùng để giải quyết việc tìm kiếm số lượng nhóm có thể có trong dữ liệu.
Quá trình này có thể được tiếp cận từ các phương pháp khác nhau nhưng đều có đặc điểm chung đều là không có giám sát.
Vì vậy, tìm kiếm số lượng nhóm cần có là mục tiêu của gom nhóm.

\section{Các hướng tiếp cận}
\subsection{Phân cấp và phân chia}
Vấn đề thường được thảo luận về điểm khác biệt giữa những kiểu gom nhóm là tập hợp các nhóm được lồng vào nhau hay tách biệt, thường được gọi với thuật ngữ phân cấp và phân chia.
Gom nhóm theo phân chia đơn giản chỉ là chia tập dữ liệu đối tượng vào thành các nhóm không đè lên nhau.
Phương pháp gom nhóm phân chia chia mỗi đối tượng dữ liệu vào một nhóm duy nhất.
Do vậy, phương pháp này cho ra kết quả bao gồm các nhóm riêng biệt, độc lập với nhau.
Nhìn vào hình \ref{fig:pic11} từ (b) đến (d) cho chúng ta thấy được phương pháp gom nhóm phân chia.

Nếu như ta có thể chia những nhóm hiện hữu thành các nhóm con thì ta sẽ có được cấu trúc phân cấp, thường được gọi là gom nhóm phân cấp.
Gom nhóm phân cấp là tập hợp các nhóm lồng vào nhau tạo thành cấu trúc cây.
Mỗi một nốt (nhóm) trong cây (ngoại trừ nốt lá) là tập hợp của những nốt con (nhóm con) và nốt gốc của cây trong nhóm chứa toàn bộ đối tượng.
Thông thường, nốt lá trong cây là nhóm đơn lẻ chứa dữ liệu.
Như vậy, gom nhóm phân cấp cho ta thấy được mối quan hệ giữa các nhóm trong dữ liệu dựa vào cấu trúc lồng phân cấp trong cây.

Nếu như ta cho phép các nhóm được phép lồng vào nhau, dựa vào hình \ref{fig:pic11}(a), ta có thể có được 2 nhóm con.
Tiếp đến hình \ref{fig:pic11}(b), ứng với mỗi nhóm, ta có thể lần lượt chia thành 3 nhóm con (như hình \ref{fig:pic11}(d)).
Các nhóm trong hình \ref{fig:pic11} từ (a) đến (d), khi thực thi theo thứ tự có thể tạo thành nhóm phân cấp tương ứng 1, 2, 4, và 6 nhóm ứng với mỗi mức độ.
Cuối cùng, ta có thể xem gom nhóm phân cấp là là chuỗi liên tiếp của gom nhóm phân chia.
Nghĩa là ta có thể lấy được kết quả của gom nhóm phân chia bằng việc cắt cây trong gom nhóm phân cấp ở một mức độ nhất định.

\subsection{Duy nhất, chồng đè và mờ}
Các nhóm trong hình \ref{fig:pic11} đều đi theo hướng duy nhất, nghĩa là mỗi đối tượng được gán vào một nhóm duy nhất.
Thực tế, ta có rất nhiều trường hợp mà một đối tượng có thể được gắn với nhiều nhóm khác nhau.
Với những trường hợp như vậy, ta đang đi theo hướng tiếp cận không duy nhất hay còn gọi là chồng đè.
Trong trường hợp tổng quát, hướng tiếp cận chồng đè hay không duy nhất dùng để phản ánh một đối tượng có thể được gán đồng thời ở nhóm này cũng như là nhóm khác.
Hướng tiếp cận duy nhất cho ta thấy được các nhóm độc lập, riêng lẻ với nhau còn hướng tiếp cận không duy nhất hay còn gọi là chồng đè cho ta thấy được quan hệ giữa các nhóm.

Ví dụ, một người trong công ty có thể kiêm nhiều chức vụ khác nhau.
Hướng tiếp cận không duy nhất thường được sử dụng khi mà một đối tượng ở giữa một hoặc nhiều nhóm khác nhau.
Đồng thời, đối tượng này có thể được gán vào một trong những nhóm này mà vẫn đảm bảo được tính hợp lý.

Trong gom nhóm mờ, mỗi đối tượng thuộc về mỗi nhóm với trọng số có thể là 0 (không thuộc về) hoặc là 1 (hoàn toàn thuộc về).
Nói cách khác, các nhóm được xem như là tập hợp mờ.
Nghĩa là các phần tử trong các nhóm này có một trong hai khả năng là 0 hoặc 1, thuộc về hoặc không thuộc về.
Gom nhóm mờ thường có điều kiện ràng buộc là tổng của trọng số trong mỗi đối tượng phải bằng 1.
Tương tự, kỹ thuật gom nhóm xác suất tính ra xác suất của mỗi điểm thuộc về ứng với nhóm, và tổng của xác suất này phải bằng 1.

Vì tổng của trọng số hay là xác suất cho bất kỳ đối tượng nào cũng phải bằng 1, gom nhóm mờ hay là xác suất không thể chỉ ra được tình huống đa lớp, tình huống mà một đối tượng có thể thuộc về nhiều lớp.
Thay vào đó, những hướng tiếp cận này thường thích hợp cho việc tránh ngẫu nhiên gán một đối tượng vào một nhóm duy nhất khi mà nó có thể thích hợp cho nhiều nhóm khác nhau.
Trong thực tế, gom nhóm mờ hay là xác suất thường chuyển đổi thành gom nhóm duy nhất bằng việc gán mỗi đối tượng vào nhóm mà có trọng số hay là xác suất là cao nhất.

\subsection{Toàn phần và cục bộ}
Gom nhóm toàn phần gán mỗi đối tượng vào một nhóm, trong khi gom nhóm cục bộ thì lại khác.
Điều thúc đẩy gom nhóm cục bộ là các đối tượng trong dữ liệu có thể không thuộc về các nhóm hiện tại.
Vì sự hiện diện của các đối tượng này có thể gây ra nhiễu và nằm ngoài so với các đối tượng còn lại.
Ví dụ, nhiều bài báo có thể cùng chia sẻ chủ đề như sự ấm lên của toàn cầu, trong khi đó những bài báo khác lại mô tả chung chung.
Cho nên, đề tìm kiếm những chủ đề quan trọng trong các bài báo, ta muốn tìm những nhóm văn bản mà liên quan chặt chẽ đến chủ đề mong muốn.
Trong những trường hợp khác, gom nhóm toàn phần thường được sử dụng. 

\section{Các loại gom nhóm}
\label{sec:clgn}
Gom nhóm dùng để tìm được các nhóm đối tượng hữu dụng, thế nào được xem là hữu dụng tùy vào mục đích của việc phân tích dữ liệu.
Ta có rất nhiều cách cũng như là ý tưởng khác nhau về gom nhóm để thể chứng minh được sự hữu dụng trong thực tế.
Để có cái nhìn tổng quát về những cách gom nhóm khác nhau, ta sẽ sử dụng dữ liệu là điểm có hai chiều.
Tuy nhiên, những phương pháp được thảo luận dưới đây đều có thể áp dụng cho bất kì dữ liệu dạng nào.
Nghĩa là các phương pháp có thể được sử dụng độc lập với dạng dữ liệu.

\subsection{Phân tách nhiều}
Một nhóm là tập hợp của các đối tượng mà mỗi đối tượng có độ tương đồng gần hơn ở các đối tượng trong nhóm.
Thỉnh thoảng, ta sẽ đặt ngưỡng để đặc tả, phân chia dữ liệu trong nhóm để có được độ tương đồng mà các phần tử trong nhóm phải ở mức từ ngưỡng trở lên để đảm bảo nhóm được hình thành ít nhiễu nhất có thể.
Tuy nhiên, ý tưởng này chỉ khả thi khi mà nhóm cần phân chia phải thỏa mãn được khoảng cách giữa các nhóm cách nhau xa.
Hình \ref{fig:pic12} đưa ra ví dụ về gom nhóm phân tách nhiều, ta có hai nhóm điểm tách biệt nhau.
Khoảng cách giữa hai điểm bất kì của hai nhóm khác nhau luôn lớn hơn so với khoảng cách giữa hai điểm trong một nhóm.
Gom nhóm phân tách nhiều không nhất thiết phải là hình cầu mà có thể là kì hình dạng nào.

\begin{figure}[htp]
\makeatletter % For spaces in paths
\patchcmd\Gread@eps{\@inputcheck#1 }{\@inputcheck"#1"\relax}{}{}
\makeatother
\psscalebox{1.0 1.0} % Change this value to rescale the drawing.
{
\begin{pspicture}(0,-1.3333334)(11.466666,1.3333334)
\pscircle[linecolor=black, linewidth=0.04, dimen=outer](1.3333334,0.0){1.3333334}
\pscircle[linecolor=black, linewidth=0.04, dimen=outer](10.133333,0.0){1.3333334}
\end{pspicture}
}
\caption{Các nhóm được gom theo phân tách nhiều}
\label{fig:pic12}
\end{figure}

\subsection{Dựa vào mẫu}
Một nhóm là một tập hợp các đối tượng mà trong đó mỗi đối tượng gần tương đồng với mẫu.
Nghĩa là mỗi nhóm sẽ có một mẫu, các đối tượng gần với mẫu nào sẽ nằm trong nhóm có mẫu đó.
Đối với dữ liệu có thuộc tính liên tục, mẫu của một nhóm thường là trung điểm (nghĩa là điểm trung bình của nhóm đó).
Khi mà trung điểm được sử dụng không hiệu quả, trong trường hợp dữ liệu tồn tại thuộc tính phân lớp, mẫu sẽ sử dụng điểm tiêu biểu, nghĩa là điểm đại diện cho toàn bộ nhóm.
Đối với nhiều loại dữ liệu, mẫu thường được xem như là trung điểm, và trong những trường hợp như vậy, ta thường gọi gom nhóm dựa vào mẫu là gom nhóm dựa vào trung điểm.
Hình dạng thường thấy của dạng gom nhóm này là hình cầu, hình \ref{fig:pic13}(b) cho ta thấy điều đó.

\begin{figure}[htp]
\makeatletter % For spaces in paths
\patchcmd\Gread@eps{\@inputcheck#1 }{\@inputcheck"#1"\relax}{}{}
\makeatother
\psscalebox{1.0 1.0} % Change this value to rescale the drawing.
{
\begin{pspicture}(0,-1.2235354)(4.819711,1.2235354)
\pscircle[linecolor=black, linewidth=0.04, dimen=outer](1.2197113,0.023535462){1.2}
\psdots[linecolor=black, dotsize=0.04](0.4197113,0.82353544)
\psdots[linecolor=black, dotsize=0.04](0.4197113,0.42353547)
\psdots[linecolor=black, dotsize=0.04](0.4197113,0.023535462)
\psdots[linecolor=black, dotsize=0.04](0.4197113,-0.37646455)
\psdots[linecolor=black, dotsize=0.04](0.4197113,-0.37646455)
\psdots[linecolor=black, dotsize=0.04](0.4197113,0.023535462)
\psdots[linecolor=black, dotsize=0.04](0.4197113,0.42353547)
\psdots[linecolor=black, dotsize=0.04](0.4197113,0.42353547)
\psdots[linecolor=black, dotsize=0.04](0.4197113,0.023535462)
\psdots[linecolor=black, dotsize=0.04](0.019711304,0.023535462)
\psdots[linecolor=black, dotsize=0.04](0.4197113,0.42353547)
\psdots[linecolor=black, dotsize=0.04](0.4197113,0.023535462)
\psdots[linecolor=black, dotsize=0.04](0.019711304,0.023535462)
\psdots[linecolor=black, dotsize=0.04](0.4197113,0.42353547)
\psdots[linecolor=black, dotsize=0.04](0.4197113,0.023535462)
\psdots[linecolor=black, dotsize=0.04](1.2197113,0.42353547)
\psdots[linecolor=black, dotsize=0.04](1.2197113,0.42353547)
\psdots[linecolor=black, dotsize=0.04](0.4197113,-0.37646455)
\psdots[linecolor=black, dotsize=0.04](0.8197113,0.023535462)
\psdots[linecolor=black, dotsize=0.04](0.8197113,0.42353547)
\psdots[linecolor=black, dotsize=0.04](1.2197113,0.42353547)
\psdots[linecolor=black, dotsize=0.04](1.6197113,0.023535462)
\psdots[linecolor=black, dotsize=0.04](1.2197113,-0.7764645)
\psdots[linecolor=black, dotsize=0.04](0.8197113,-0.37646455)
\psdots[linecolor=black, dotsize=0.04](0.8197113,-0.37646455)
\psdots[linecolor=black, dotsize=0.04](1.6197113,0.023535462)
\psdots[linecolor=black, dotsize=0.04](0.8197113,-0.37646455)
\psdots[linecolor=black, dotsize=0.04](0.8197113,-0.7764645)
\psdots[linecolor=black, dotsize=0.04](1.6197113,-0.7764645)
\psdots[linecolor=black, dotsize=0.04](1.2197113,-0.7764645)
\psdots[linecolor=black, dotsize=0.04](1.2197113,0.023535462)
\psdots[linecolor=black, dotsize=0.04](1.2197113,-0.37646455)
\psdots[linecolor=black, dotsize=0.04](2.0197113,-0.37646455)
\psdots[linecolor=black, dotsize=0.04](2.0197113,-0.37646455)
\psdots[linecolor=black, dotsize=0.04](2.0197113,0.023535462)
\psdots[linecolor=black, dotsize=0.04](2.0197113,0.42353547)
\psdots[linecolor=black, dotsize=0.04](2.0197113,0.42353547)
\psdots[linecolor=black, dotsize=0.04](1.6197113,0.42353547)
\psdots[linecolor=black, dotsize=0.04](1.2197113,-0.37646455)
\psdots[linecolor=black, dotsize=0.04](1.6197113,0.42353547)
\psdots[linecolor=black, dotsize=0.04](1.2197113,0.82353544)
\psdots[linecolor=black, dotsize=0.04](0.8197113,0.82353544)
\psdots[linecolor=black, dotsize=0.04](0.8197113,0.82353544)
\psdots[linecolor=black, dotsize=0.04](0.4197113,0.82353544)
\psdots[linecolor=black, dotsize=0.04](1.2197113,0.82353544)
\psdots[linecolor=black, dotsize=0.04](1.6197113,0.82353544)
\psdots[linecolor=black, dotsize=0.04](2.0197113,0.82353544)
\psdots[linecolor=black, dotsize=0.04](2.0197113,-0.37646455)
\psdots[linecolor=black, dotsize=0.04](2.0197113,-0.7764645)
\psdots[linecolor=black, dotsize=0.04](1.2197113,-0.37646455)
\psdots[linecolor=black, dotsize=0.04](1.6197113,-0.37646455)
\psdots[linecolor=black, dotsize=0.04](1.2197113,-0.7764645)
\psdots[linecolor=black, dotsize=0.04](1.2197113,-1.1764646)
\psdots[linecolor=black, dotsize=0.04](1.2197113,-1.1764646)
\psdots[linecolor=black, dotsize=0.04](1.2197113,0.023535462)
\psdots[linecolor=black, dotsize=0.04](0.8197113,0.023535462)
\psdots[linecolor=black, dotsize=0.04](0.8197113,-0.7764645)
\psdots[linecolor=black, dotsize=0.04](0.4197113,-0.7764645)
\psdots[linecolor=black, dotsize=0.04](0.8197113,0.023535462)
\psdots[linecolor=black, dotsize=0.04](0.8197113,0.82353544)
\psdots[linecolor=black, dotsize=0.04](1.6197113,0.82353544)
\pscircle[linecolor=black, linewidth=0.04, dimen=outer](3.6197114,0.023535462){1.2}
\psdots[linecolor=black, dotstyle=x, dotsize=0.1](2.8197112,0.82353544)
\psdots[linecolor=black, dotstyle=x, dotsize=0.1](2.8197112,0.42353547)
\psdots[linecolor=black, dotstyle=x, dotsize=0.1](2.8197112,0.023535462)
\psdots[linecolor=black, dotstyle=x, dotsize=0.1](2.8197112,-0.37646455)
\psdots[linecolor=black, dotstyle=x, dotsize=0.1](2.8197112,-0.37646455)
\psdots[linecolor=black, dotstyle=x, dotsize=0.1](2.8197112,0.023535462)
\psdots[linecolor=black, dotstyle=x, dotsize=0.1](2.8197112,0.42353547)
\psdots[linecolor=black, dotstyle=x, dotsize=0.1](2.8197112,0.42353547)
\psdots[linecolor=black, dotstyle=x, dotsize=0.1](2.8197112,0.023535462)
\psdots[linecolor=black, dotstyle=x, dotsize=0.1](2.4197114,0.023535462)
\psdots[linecolor=black, dotstyle=x, dotsize=0.1](2.8197112,0.42353547)
\psdots[linecolor=black, dotstyle=x, dotsize=0.1](2.8197112,0.023535462)
\psdots[linecolor=black, dotstyle=x, dotsize=0.1](2.4197114,0.023535462)
\psdots[linecolor=black, dotstyle=x, dotsize=0.1](2.8197112,0.42353547)
\psdots[linecolor=black, dotstyle=x, dotsize=0.1](2.8197112,0.023535462)
\psdots[linecolor=black, dotstyle=x, dotsize=0.1](3.6197114,0.42353547)
\psdots[linecolor=black, dotstyle=x, dotsize=0.1](3.6197114,0.42353547)
\psdots[linecolor=black, dotstyle=x, dotsize=0.1](2.8197112,-0.37646455)
\psdots[linecolor=black, dotstyle=x, dotsize=0.1](3.2197113,0.023535462)
\psdots[linecolor=black, dotstyle=x, dotsize=0.1](3.2197113,0.42353547)
\psdots[linecolor=black, dotstyle=x, dotsize=0.1](3.6197114,0.42353547)
\psdots[linecolor=black, dotstyle=x, dotsize=0.1](4.0197115,0.023535462)
\psdots[linecolor=black, dotstyle=x, dotsize=0.1](3.6197114,-0.7764645)
\psdots[linecolor=black, dotstyle=x, dotsize=0.1](3.2197113,-0.37646455)
\psdots[linecolor=black, dotstyle=x, dotsize=0.1](3.2197113,-0.37646455)
\psdots[linecolor=black, dotstyle=x, dotsize=0.1](4.0197115,0.023535462)
\psdots[linecolor=black, dotstyle=x, dotsize=0.1](3.2197113,-0.37646455)
\psdots[linecolor=black, dotstyle=x, dotsize=0.1](3.2197113,-0.7764645)
\psdots[linecolor=black, dotstyle=x, dotsize=0.1](4.0197115,-0.7764645)
\psdots[linecolor=black, dotstyle=x, dotsize=0.1](3.6197114,-0.7764645)
\psdots[linecolor=black, dotstyle=x, dotsize=0.1](3.6197114,0.023535462)
\psdots[linecolor=black, dotstyle=x, dotsize=0.1](3.6197114,-0.37646455)
\psdots[linecolor=black, dotstyle=x, dotsize=0.1](4.419711,-0.37646455)
\psdots[linecolor=black, dotstyle=x, dotsize=0.1](4.419711,-0.37646455)
\psdots[linecolor=black, dotstyle=x, dotsize=0.1](4.419711,0.023535462)
\psdots[linecolor=black, dotstyle=x, dotsize=0.1](4.419711,0.42353547)
\psdots[linecolor=black, dotstyle=x, dotsize=0.1](4.419711,0.42353547)
\psdots[linecolor=black, dotstyle=x, dotsize=0.1](4.0197115,0.42353547)
\psdots[linecolor=black, dotstyle=x, dotsize=0.1](3.6197114,-0.37646455)
\psdots[linecolor=black, dotstyle=x, dotsize=0.1](4.0197115,0.42353547)
\psdots[linecolor=black, dotstyle=x, dotsize=0.1](3.6197114,0.82353544)
\psdots[linecolor=black, dotstyle=x, dotsize=0.1](3.2197113,0.82353544)
\psdots[linecolor=black, dotstyle=x, dotsize=0.1](3.2197113,0.82353544)
\psdots[linecolor=black, dotstyle=x, dotsize=0.1](2.8197112,0.82353544)
\psdots[linecolor=black, dotstyle=x, dotsize=0.1](3.6197114,0.82353544)
\psdots[linecolor=black, dotstyle=x, dotsize=0.1](4.0197115,0.82353544)
\psdots[linecolor=black, dotstyle=x, dotsize=0.1](4.419711,0.82353544)
\psdots[linecolor=black, dotstyle=x, dotsize=0.1](4.419711,-0.37646455)
\psdots[linecolor=black, dotstyle=x, dotsize=0.1](4.419711,-0.7764645)
\psdots[linecolor=black, dotstyle=x, dotsize=0.1](3.6197114,-0.37646455)
\psdots[linecolor=black, dotstyle=x, dotsize=0.1](4.0197115,-0.37646455)
\psdots[linecolor=black, dotstyle=x, dotsize=0.1](3.6197114,-0.7764645)
\psdots[linecolor=black, dotstyle=x, dotsize=0.1](3.6197114,-1.1764646)
\psdots[linecolor=black, dotstyle=x, dotsize=0.1](3.6197114,-1.1764646)
\psdots[linecolor=black, dotstyle=x, dotsize=0.1](3.6197114,0.023535462)
\psdots[linecolor=black, dotstyle=x, dotsize=0.1](3.2197113,0.023535462)
\psdots[linecolor=black, dotstyle=x, dotsize=0.1](3.2197113,-0.7764645)
\psdots[linecolor=black, dotstyle=x, dotsize=0.1](2.8197112,-0.7764645)
\psdots[linecolor=black, dotstyle=x, dotsize=0.1](3.2197113,0.023535462)
\psdots[linecolor=black, dotstyle=x, dotsize=0.1](3.2197113,0.82353544)
\psdots[linecolor=black, dotstyle=x, dotsize=0.1](4.0197115,0.82353544)
\end{pspicture}
}
\caption{Các nhóm được gom dựa vào trung điểm}
\label{fig:pic13}
\end{figure}

\subsection{Dựa vào đồ thị}
Nếu như dữ liệu được biểu diễn như là đồ thị, với nốt là các đối tượng và các liên kết thể hiện phần kết nối giữa các đối tượng.
Khi đó, nhóm được định nghĩa như là một thành phần liên kết, nghĩa là một nhóm bao gồm các đối tượng liên kết với nhau nhưng mà không tồn tại liên kết với các đối tượng ở ngoài nhóm.
Một ví dụ quan trọng của gom nhóm dựa vào đồ thị là gom nhóm dựa vào tính liền kề, khi mà hai đối tượng được liên kết khi và chỉ khi hai điểm nằm trong khoảng cách đặc tả của nhau.
Điều này ngầm chỉ định mỗi đối tượng trong nhóm liền kề luôn ở gần hơn so với một vài đối tượng khác ở trong cùng nhóm và hơn hẳn những đối tượng ở khác nhóm.
Hình \ref{fig:pic14} chỉ ra ví dụ về gom nhóm dựa vào liền kề.

Cách sử dụng gom nhóm này chỉ hữu dụng khi các nhóm này dị biệt hoặc là quấn vào nhau.
Tuy nhiên, một vấn đề cần chú ý là cách gom nhóm này có thể xuất hiện nhiễu, như trong hình \ref{fig:pic14}, một cây cầu nhỏ xuất hiện tạo thành cầu nối giữa hai nhóm riêng biệt.
Những loại gom nhóm dựa vào đồ thị cũng có thể sử dụng được.
Một trong những hướng tiếp cận này là clique, một tập hợp các nốt trong đồ thị mà kết nối hoàn toàn với nhau.
Đặc biệt, nếu ta có thể thêm vào phần kết nối giữa các đối tượng theo thứ tự dựa vào khoảng cách, nhóm được hình thành rồi từ đó tạo thành tập đối tượng tạo nên clique.
Cũng như gom nhóm dựa vào mẫu, các nhóm theo phương pháp này thường có xu hướng tạo nên hình cầu.

\begin{figure}[htp]
\makeatletter % For spaces in paths
\patchcmd\Gread@eps{\@inputcheck#1 }{\@inputcheck"#1"\relax}{}{}
\makeatother
\psscalebox{1.0 1.0} % Change this value to rescale the drawing.
{
\begin{pspicture}(0,-2.3036544)(10.806001,2.3036544)
\definecolor{colour0}{rgb}{0.8,0.8,0.8}
\psline[linecolor=black, linewidth=0.04, linestyle=dotted, dotsep=0.10583334cm](1.2060003,1.0036933)(0.006000366,0.20369339)(1.6060004,-0.19630662)(0.006000366,-0.9963066)(0.006000366,-0.9963066)
\rput{-269.66956}(4.83741,-4.8022056){\psarc[linecolor=colour0, linewidth=0.6, dimen=outer](4.806,0.00369339){2.0}{0.0}{180.0}}
\pscircle[linecolor=black, linewidth=0.04, dimen=outer](4.806,0.20369339){1.2}
\pscircle[linecolor=black, linewidth=0.04, dimen=outer](9.606,0.20369339){1.2}
\psline[linecolor=black, linewidth=0.04, linestyle=dotted, dotsep=0.10583334cm](6.0060005,0.20369339)(8.406,0.20369339)(8.406,0.20369339)
\end{pspicture}
}
\caption{Các nhóm được gom dựa vào liền kề}
\label{fig:pic14}
\end{figure}


\subsection{Dựa vào mật độ}
Nhóm được định nghĩa là vùng mật độ của các đối tượng mà bao quanh nó là vùng có mật độ thấp.
Hình \ref{fig:pic15} cho ta thấy gom nhóm dựa vào mật độ cho dữ liệu được tạo bằng cách thêm độ nhiễu cho dữ liệu từ hình \ref{fig:pic14}.
Hai hình tròn không có trộn vào nhau như hình \ref{fig:pic14}, bởi vì phần cầu nối giữa hai nhóm bị che mờ trong phần nhiễu.
Tương tự, phần cong thể hiện trong hình \ref{fig:pic14} cũng bị che mờ trong phần nhiễu và không tạo thành nhóm trong hình \ref{fig:pic15}.

\begin{figure}[htp]
\makeatletter % For spaces in paths
\patchcmd\Gread@eps{\@inputcheck#1 }{\@inputcheck"#1"\relax}{}{}
\makeatother
\psscalebox{1.0 1.0} % Change this value to rescale the drawing.
{
\begin{pspicture}(0,-2.967742)(11.383526,2.967742)
\definecolor{colour0}{rgb}{0.8,0.8,0.8}
\rput{-269.66956}(3.5690148,-3.380749){\psarc[linecolor=colour0, linewidth=0.6, dimen=outer](3.4651613,0.08387104){2.0}{0.0}{180.0}}
\pscircle[linecolor=black, linewidth=0.04, dimen=outer](3.4651613,0.28387105){1.2}
\pscircle[linecolor=black, linewidth=0.04, dimen=outer](8.2651615,0.28387105){1.2}
\psframe[linecolor=black, linewidth=0.04, linestyle=dotted, dotsep=0.10583334cm, dimen=outer](11.374839,2.967742)(0.020000149,-2.967742)
\psline[linecolor=black, linewidth=0.04, linestyle=dotted, dotsep=0.10583334cm](0.020000149,2.7096775)(11.374839,2.7096775)(11.374839,2.7096775)
\psline[linecolor=black, linewidth=0.04, linestyle=dotted, dotsep=0.10583334cm](0.020000149,2.451613)(11.374839,2.451613)(11.374839,2.451613)
\psline[linecolor=black, linewidth=0.04, linestyle=dotted, dotsep=0.10583334cm](0.020000149,2.1935484)(2.6006453,2.1935484)
\psline[linecolor=black, linewidth=0.04, linestyle=dotted, dotsep=0.10583334cm](3.6329033,2.1935484)(11.374839,2.1935484)
\psline[linecolor=black, linewidth=0.04, linestyle=dotted, dotsep=0.10583334cm](0.020000149,1.9354839)(2.0845163,1.9354839)(2.0845163,1.9354839)
\psline[linecolor=black, linewidth=0.04, linestyle=dotted, dotsep=0.10583334cm](3.6329033,1.9354839)(11.374839,1.9354839)
\psline[linecolor=black, linewidth=0.04, linestyle=dotted, dotsep=0.10583334cm](0.020000149,1.6774194)(1.8264518,1.6774194)(1.5683873,1.6774194)
\psline[linecolor=black, linewidth=0.04, linestyle=dotted, dotsep=0.10583334cm](2.8587098,1.6774194)(11.374839,1.6774194)
\psline[linecolor=black, linewidth=0.04, linestyle=dotted, dotsep=0.10583334cm](0.020000149,1.4193549)(1.5683873,1.4193549)
\psline[linecolor=black, linewidth=0.04, linestyle=dotted, dotsep=0.10583334cm](2.6006453,1.4193549)(3.1167743,1.4193549)
\psline[linecolor=black, linewidth=0.04, linestyle=dotted, dotsep=0.10583334cm](4.1490326,1.4193549)(7.7619357,1.4193549)
\psline[linecolor=black, linewidth=0.04, linestyle=dotted, dotsep=0.10583334cm](9.0522585,1.4193549)(11.116775,1.4193549)(11.374839,1.4193549)
\psline[linecolor=black, linewidth=0.04, linestyle=dotted, dotsep=0.10583334cm](0.020000149,1.1612904)(1.3103228,1.1612904)
\psline[linecolor=black, linewidth=0.04, linestyle=dotted, dotsep=0.10583334cm](2.3425808,1.1612904)(2.6006453,1.1612904)
\psline[linecolor=black, linewidth=0.04, linestyle=dotted, dotsep=0.10583334cm](4.407097,1.1612904)(7.503871,1.1612904)
\psline[linecolor=black, linewidth=0.04, linestyle=dotted, dotsep=0.10583334cm](9.310323,1.1612904)(11.374839,1.1612904)
\psline[linecolor=black, linewidth=0.04, linestyle=dotted, dotsep=0.10583334cm](0.020000149,0.9032259)(1.3103228,0.9032259)
\psline[linecolor=black, linewidth=0.04, linestyle=dotted, dotsep=0.10583334cm](2.0845163,0.9032259)(2.3425808,0.9032259)(2.3425808,0.9032259)
\psline[linecolor=black, linewidth=0.04, linestyle=dotted, dotsep=0.10583334cm](4.6651616,0.9032259)(6.987742,0.9032259)
\psline[linecolor=black, linewidth=0.04, linestyle=dotted, dotsep=0.10583334cm](9.310323,0.9032259)(11.374839,0.9032259)
\psline[linecolor=black, linewidth=0.04, linestyle=dotted, dotsep=0.10583334cm](0.020000149,0.6451614)(1.0522583,0.6451614)
\psline[linecolor=black, linewidth=0.04, linestyle=dotted, dotsep=0.10583334cm](1.8264518,0.6451614)(2.3425808,0.6451614)
\psline[linecolor=black, linewidth=0.04, linestyle=dotted, dotsep=0.10583334cm](4.6651616,0.6451614)(6.987742,0.6451614)(6.987742,0.6451614)
\psline[linecolor=black, linewidth=0.04, linestyle=dotted, dotsep=0.10583334cm](9.568387,0.6451614)(11.374839,0.6451614)
\psline[linecolor=black, linewidth=0.04, linestyle=dotted, dotsep=0.10583334cm](0.020000149,0.38709685)(0.020000149,0.38709685)(1.0522583,0.38709685)
\psline[linecolor=black, linewidth=0.04, linestyle=dotted, dotsep=0.10583334cm](1.0522583,0.38709685)(1.0522583,0.38709685)
\psline[linecolor=black, linewidth=0.04, linestyle=dotted, dotsep=0.10583334cm](1.8264518,0.38709685)(2.0845163,0.38709685)
\psline[linecolor=black, linewidth=0.04, linestyle=dotted, dotsep=0.10583334cm](4.6651616,0.38709685)(6.987742,0.38709685)
\psline[linecolor=black, linewidth=0.04, linestyle=dotted, dotsep=0.10583334cm](9.568387,0.38709685)(11.374839,0.38709685)
\psline[linecolor=black, linewidth=0.04, linestyle=dotted, dotsep=0.10583334cm](0.020000149,0.12903233)(1.0522583,0.12903233)
\psline[linecolor=black, linewidth=0.04, linestyle=dotted, dotsep=0.10583334cm](1.8264518,0.12903233)(2.0845163,0.12903233)
\psline[linecolor=black, linewidth=0.04, linestyle=dotted, dotsep=0.10583334cm](2.0845163,0.12903233)(2.3425808,0.12903233)
\psline[linecolor=black, linewidth=0.04, linestyle=dotted, dotsep=0.10583334cm](4.6651616,0.12903233)(6.987742,0.12903233)
\psline[linecolor=black, linewidth=0.04, linestyle=dotted, dotsep=0.10583334cm](9.568387,0.12903233)(11.374839,0.12903233)
\psline[linecolor=black, linewidth=0.04, linestyle=dotted, dotsep=0.10583334cm](0.020000149,-0.12903218)(1.0522583,-0.12903218)
\psline[linecolor=black, linewidth=0.04, linestyle=dotted, dotsep=0.10583334cm](1.8264518,-0.12903218)(2.3425808,-0.12903218)
\psline[linecolor=black, linewidth=0.04, linestyle=dotted, dotsep=0.10583334cm](4.6651616,-0.12903218)(6.987742,-0.12903218)
\psline[linecolor=black, linewidth=0.04, linestyle=dotted, dotsep=0.10583334cm](9.568387,-0.12903218)(11.116775,-0.12903218)
\psline[linecolor=black, linewidth=0.04, linestyle=dotted, dotsep=0.10583334cm](0.020000149,-0.3870967)(1.0522583,-0.3870967)
\psline[linecolor=black, linewidth=0.04, linestyle=dotted, dotsep=0.10583334cm](1.8264518,-0.3870967)(2.3425808,-0.3870967)
\psline[linecolor=black, linewidth=0.04, linestyle=dotted, dotsep=0.10583334cm](4.407097,-0.3870967)(7.2458067,-0.3870967)
\psline[linecolor=black, linewidth=0.04, linestyle=dotted, dotsep=0.10583334cm](9.310323,-0.3870967)(11.374839,-0.3870967)
\psline[linecolor=black, linewidth=0.04, linestyle=dotted, dotsep=0.10583334cm](0.020000149,-0.6451612)(1.3103228,-0.6451612)
\psline[linecolor=black, linewidth=0.04, linestyle=dotted, dotsep=0.10583334cm](2.0845163,-0.6451612)(2.6006453,-0.6451612)
\psline[linecolor=black, linewidth=0.04, linestyle=dotted, dotsep=0.10583334cm](4.407097,-0.6451612)(7.503871,-0.6451612)
\psline[linecolor=black, linewidth=0.04, linestyle=dotted, dotsep=0.10583334cm](9.0522585,-0.6451612)(11.374839,-0.6451612)
\psline[linecolor=black, linewidth=0.04, linestyle=dotted, dotsep=0.10583334cm](0.020000149,-0.9032257)(1.3103228,-0.9032257)
\psline[linecolor=black, linewidth=0.04, linestyle=dotted, dotsep=0.10583334cm](2.0845163,-0.9032257)(11.374839,-0.9032257)
\psline[linecolor=black, linewidth=0.04, linestyle=dotted, dotsep=0.10583334cm](0.020000149,-1.1612903)(1.3103228,-1.1612903)
\psline[linecolor=black, linewidth=0.04, linestyle=dotted, dotsep=0.10583334cm](2.3425808,-1.1612903)(11.374839,-1.1612903)
\psline[linecolor=black, linewidth=0.04, linestyle=dotted, dotsep=0.10583334cm](0.020000149,-1.4193548)(1.5683873,-1.4193548)
\psline[linecolor=black, linewidth=0.04, linestyle=dotted, dotsep=0.10583334cm](2.6006453,-1.4193548)(11.374839,-1.4193548)
\psline[linecolor=black, linewidth=0.04, linestyle=dotted, dotsep=0.10583334cm](0.020000149,-1.6774193)(1.8264518,-1.6774193)(1.8264518,-1.6774193)
\psline[linecolor=black, linewidth=0.04, linestyle=dotted, dotsep=0.10583334cm](3.6329033,-1.6774193)(11.374839,-1.6774193)
\psline[linecolor=black, linewidth=0.04, linestyle=dotted, dotsep=0.10583334cm](0.020000149,-1.9354838)(2.3425808,-1.9354838)
\psline[linecolor=black, linewidth=0.04, linestyle=dotted, dotsep=0.10583334cm](3.6329033,-1.9354838)(11.374839,-1.9354838)(11.374839,-1.9354838)
\psline[linecolor=black, linewidth=0.04, linestyle=dotted, dotsep=0.10583334cm](0.020000149,-2.1935482)(2.6006453,-2.1935482)
\psline[linecolor=black, linewidth=0.04, linestyle=dotted, dotsep=0.10583334cm](3.6329033,-2.1935482)(11.374839,-2.1935482)
\psline[linecolor=black, linewidth=0.04, linestyle=dotted, dotsep=0.10583334cm](0.020000149,-2.451613)(11.374839,-2.451613)
\psline[linecolor=black, linewidth=0.04, linestyle=dotted, dotsep=0.10583334cm](0.020000149,-2.7096775)(11.374839,-2.7096775)
\end{pspicture}
}
\caption{Các nhóm được gom dựa vào mật độ}
\label{fig:pic15}
\end{figure}


\subsection{Chia sẻ thuộc tính (gom nhóm theo ý tưởng)}
Ta có thể định nghĩa nhóm như là tập hợp của các đối tượng mà chia sẻ chung thuộc tính.
Cách định nghĩa này bao gồm tất cả định nghĩa về gom nhóm ở các phần trước.
Các đối tượng sẽ nằm trong nhóm trung tâm chia sẻ thuộc tính mà tất cả cùng ở gần điểm trung tâm hay là điểm tiêu biểu, như là các nhóm ở trong hình \ref{fig:pic16}.
Khu vực tam giác nằm kế cận khu vực tứ giác và 2 đường tròn quấn vào nhau.
Trong cả 2 trường hợp trên, thuật toán gom nhóm cần ý tưởng đặc tả của nhóm để có thể tìm được thành công những nhóm này.
Quá trình tìm kiếm những nhóm như vậy được gọi là gom nhom theo ý tưởng.
Tuy nhiên, nếu ý tưởng của nhóm quá phức tạp có thể dẫn đến nhận diện mẫu.

\begin{figure}[htp]
\makeatletter % For spaces in paths
\patchcmd\Gread@eps{\@inputcheck#1 }{\@inputcheck"#1"\relax}{}{}
\makeatother
\psscalebox{1.0 1.0} % Change this value to rescale the drawing.
{
\begin{pspicture}(0,-1.2132502)(14.024741,1.2132502)
\definecolor{colour1}{rgb}{0.4,0.2,1.0}
\definecolor{colour2}{rgb}{0.4,0.4,0.0}
\psframe[linecolor=black, linewidth=0.04, dimen=outer](6.8247414,1.1932502)(2.4247413,-1.2067499)
\rput{-269.72128}(1.2239491,-1.2315094){\pstriangle[linecolor=black, linewidth=0.04, dimen=outer](1.2247412,-1.2067499)(2.4,2.4)}
\psline[linecolor=black, linewidth=0.04, linestyle=dotted, dotsep=0.10583334cm](1.6247412,0.79325014)(2.4247413,0.79325014)
\psline[linecolor=black, linewidth=0.04, linestyle=dotted, dotsep=0.10583334cm](1.6247412,0.79325014)(1.6247412,0.79325014)
\psline[linecolor=black, linewidth=0.04, linestyle=dotted, dotsep=0.10583334cm](1.2247412,0.3932501)(2.4247413,0.3932501)
\psline[linecolor=black, linewidth=0.04, linestyle=dotted, dotsep=0.10583334cm](0.4247412,-0.0067498777)(2.4247413,-0.0067498777)
\psline[linecolor=black, linewidth=0.04, linestyle=dotted, dotsep=0.10583334cm](1.2247412,-0.40674987)(2.4247413,-0.40674987)
\psline[linecolor=black, linewidth=0.04, linestyle=dotted, dotsep=0.10583334cm](1.6247412,-0.8067499)(2.4247413,-0.8067499)
\psline[linecolor=black, linewidth=0.04, linestyle=dotted, dotsep=0.10583334cm](2.8247411,1.1932502)(2.8247411,-1.2067499)
\psline[linecolor=black, linewidth=0.04, linestyle=dotted, dotsep=0.10583334cm](3.2247412,1.1932502)(3.2247412,-1.2067499)
\psline[linecolor=black, linewidth=0.04, linestyle=dotted, dotsep=0.10583334cm](3.6247413,1.1932502)(3.6247413,-1.2067499)(3.6247413,-0.40674987)
\psline[linecolor=black, linewidth=0.04, linestyle=dotted, dotsep=0.10583334cm](4.024741,1.1932502)(4.024741,-1.2067499)
\psline[linecolor=black, linewidth=0.04, linestyle=dotted, dotsep=0.10583334cm](4.4247413,1.1932502)(4.4247413,-1.2067499)
\psline[linecolor=black, linewidth=0.04, linestyle=dotted, dotsep=0.10583334cm](4.8247414,1.1932502)(4.8247414,-1.2067499)
\psline[linecolor=black, linewidth=0.04, linestyle=dotted, dotsep=0.10583334cm](5.224741,1.1932502)(5.224741,-1.2067499)
\psline[linecolor=black, linewidth=0.04, linestyle=dotted, dotsep=0.10583334cm](5.624741,1.1932502)(5.624741,-1.2067499)
\psline[linecolor=black, linewidth=0.04, linestyle=dotted, dotsep=0.10583334cm](6.024741,1.1932502)(6.024741,1.1932502)(6.024741,-1.2067499)
\psline[linecolor=black, linewidth=0.04, linestyle=dotted, dotsep=0.10583334cm](6.4247413,1.1932502)(6.4247413,-1.2067499)
\psellipse[linecolor=colour1, linewidth=0.4, dimen=outer](10.224741,-0.0067498777)(1.8,1.2)
\psellipse[linecolor=colour2, linewidth=0.4, dimen=outer](12.224741,-0.0067498777)(1.8,1.2)
\end{pspicture}
}
\caption{Gom nhóm ý tưởng}
\label{fig:pic16}
\end{figure}


\section{Phương pháp gom nhóm}
\label{sec:ppgn}
Dựa vào những loại gom nhóm được đề cập ở \ref{sec:clgn}, ta có thể liệt kê ra các phương pháp gom nhóm sau:
\begin{enumerate}
\item[•]\textbf{K-means}: đây là phương pháp gom nhóm theo mẫu và sử dụng kỹ thuật phân chia để có thể tìm được số lượng nhóm biết trước (K), và mỗi nhóm được đại diện bằng điểm trung tâm.
\item[•]\textbf{Gom nhóm phân cấp tích tụ}: cách gom nhóm này tiếp cận theo một tập hợp của những nhóm liên quan mật thiết mà tạo ra được các nhóm phân cấp bằng việc bắt đầu với mỗi điểm như là một nhóm đơn lẻ và sau đó thì không ngừng trộn những nhóm ở gần với nhau nhất cho đến khi chỉ còn lại một nhóm duy nhất.
Một vài kỹ thuật của cách tiếp cận này có kiểu thực thi theo cách gom nhóm theo đồ thị, trong khi một số khác thì theo hướng tiếp cận gom nhóm theo mẫu.
\item[•]\textbf{DBSCAN}: đây là thuật toán gom nhóm the hướng dựa vào mật độ tạo ra các nhóm được phân chia, trong đó số lượng nhóm được xác định tự động bởi thuật toán.
Những điểm nằm trong vùng mật độ thấp được xem như là nhiễu và bị loại bỏ.
Vì vậy, DBSCAN không thể tạo ra gom nhóm hoàn chỉnh.
\end{enumerate}

\section{Mục tiêu của đồ án}
Dựa vào những phương pháp đã liệt kê ở \ref{sec:ppgn} thì phương pháp gom nhóm phân cấp cho ta thấy được kết quả tốt hơn so với K-means và DBSCAN.
Vì gom nhóm phân cấp không cần biết trước số lượng nhóm (K-means cần biết trước số lượng nhóm) và không làm mất dữ liệu (như DBSCAN).
Ngoài ra, ta có thể có được kết quả phân chia giống như K-means khi thực hiện vết cắt tại một mức độ trong gom nhóm phân cấp.
Mục tiêu của đồ án là gom nhóm tin tức tiếng Việt nên cần lựa chọn phương pháp tối ưu nhất đẻ thực hiện.
Tuy nhiên, khi gom nhóm với dữ liệu lớn thì số chiều của dữ liệu tăng lên rất nhiều làm chậm quá trình thực thi.
Vì vậy, việc sử dụng cách thể hiện dữ liệu khác để làm giảm số chiều như doc2vec có thể làm tăng hiệu năng và làm nhanh quá trình thực thi.
