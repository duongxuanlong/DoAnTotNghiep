\chapter{Giới thiệu}
\label{Chapter1}

\section{Giới thiệu về gom nhóm văn bản}

%Gom nhóm là gì?
Gom nhóm là công việc tìm kiếm và chia những đối tượng gần giống nhau vào thành những nhóm trong dữ liệu sao cho nhũng nhóm này có ý nghĩa, hữu dụng hoặc cả hai. %\cite{1984-TeX-Knuth}
Nếu như mục đích của gom nhóm là tìm kiếm ngữ nghĩa của dữ liệu thì các nhóm được gom sẽ thể hiện cấu trúc của dữ liệu.
Trong một vài trường hợp, gom nhóm được xem là cầu nối để thực hiện cho những mục đích khác như tóm tắt dữ liệu.
Bất kể việc sử dụng gom nhóm cho mục đích ngữ nghĩa hay là hữu dụng thì gom nhóm đóng vai trò quan trọng trong nhiều lĩnh vực khác nhau như: tâm lý học và các ngành khoa học xã hội, sinh học, thống kê, nhận diện mô hình, truy vấn thông tin, máy học và khai thác dữ liệu.

Gom nhóm văn bản là một ứng dụng của gom nhóm.
Trong đó, các văn bản gần tương đồng sẽ được gom chung với nhau thành nhóm riêng biệt theo cách thức không giám sát.
Các văn bản gần liên quan với nhau có độ tương đồng lớn hơn so với những văn bản dị biệt.
Như vậy, những văn bản tương đồng này có thể tạo thành nhóm có cùng một chủ để.
Các nhóm này được hình thành từ quá trình gom nhóm không giám sát nên giúp cho chúng ta thấy được cấu trúc của ngữ liệu.

Ngày nay, việc tìm kiếm thông tin trên mạng đã trở thành kỹ năng cần thiết cho bất kì ai.
Tuy nhiên, khi ta truy vấn thông tin với từ khóa bất kì thì có thể cho ra quá nhiều kết quả khác nhau.
Vì vậy, ta cần tổ chức lại thông tin cần truy vấn thành cấu trúc phù hợp.
Gom nhóm văn bản có thể giúp chúng ta sắp xếp các văn bản thành những chủ để phù hợp và giúp cho việc truy vấn dễ dàng hơn.

Gom nhóm văn bản tự động tạo ra những nhóm có văn bản liên quan với nhau mà không cần phải tạo trước tập huấn luyện hay nguyên tắc phân loại. Ngoải ra gom nhóm văn bản dựa vào độ tương đồng của nội dung để cải thiện hiệu quả tìm kiếm.
\begin{enumerate}
\item[•]Cải thiện hồi quy trong tìm kiếm: khi câu truy vấn trùng với một văn bản thì kết quả trả về có thể bao gồm toàn bộ nhóm chứa văn bản.
\item[•]Cải thiện độ chính xác trong tìm kiếm: gom nhóm văn bản thành những nhóm nhỏ hơn trong nhóm của những văn bản liên quan.
\item[•]Phân tán hoặc tập hợp: khi một câu truy vấn không thể được công thức hóa thì ta có thể cho phép người dùng duyệt văn bản theo nhóm.
\item[•]Gom nhóm theo truy vấn đặc tả: những văn bản gần liên quan đến câu truy vấn nhất sẽ nằm trong nhóm con lồng vào bên trong nhóm lớn hơn có độ tương đồng thấp hơn.
\end{enumerate}

Nghiên cứu của vấn đề gom nhóm đứng trước tính khả dụng khi ứng dụng vào văn bản. Các phương pháp truyền thống cho gom nhóm thường tập trung vào dữ liệu lớn, khi thuộc tính của dữ liệu là số. Vấn đề này cũng được nghiên cứu trong phân loại dữ liệu, khi mà thuộc tính có giá trị nặc danh. Vấn đề của gom nhóm là tìm được tính khả dụng trong các nhiệm vụ sau:
\begin{enumerate}
\item[•]Duyệt và tổ chức văn bản: tổ chức phân cấp của văn bản vào trong các hạng mục mạch lạc. Điều này có thể giúp ích cho việc duyệt hệ thống của tập hợp văn bản. Ví dụ kinh kiển cho phương pháp này là phân tán hoặc tập hợp. Phương pháp này cung cấp kỹ thuật duyệt hệ thống với sử dụng gom nhóm tổ chức của tập hợp văn bản.
\item[•]Tóm tắt ngữ liệu: kỹ thuật gom nhóm cung cấp tóm tắt mạch lạc của tập hợp trong dạng nhóm tài tiệu hoặc nhóm từ. Điều này được sử dụng đề cung cấp tóm tắt trong phần nội dung tổng kết của ngữ liệu căn bản. Lĩnh vực này có nhiều phương pháp, đặc biệt là gom nhóm câu dùng để tóm tắt văn bản. Vấn đề của gom nhóm liên quan đến việc giảm số chiều và mô hình hóa chủ đề. 
\item[•]Phân loại văn bản : Gom nhóm là phương pháp học không giám sát. Nó thể được đòn bẩy hóa để cải thiện chất lượng kết quả trong giám sát. Cụ thể, các nhóm từ và phương thức đồng huấn luyện có thể được sử dụng để cải thiện độ chính xác phân loại của ứng dụng giám sát với tác dụng của kỹ thuật gom nhóm.
\end{enumerate}

\section{Bài toán gom nhóm văn bản}
Gom nhóm là quá trình gom các đối tượng dữ liệu dựa vào thông tin được tìm thấy trong dữ liệu mô tả đối tượng và quan hệ giữa các đối tượng.
Mục tiêu của gom nhóm là những đối tượng gần giống nhau (tương đồng) sẽ ở trong cùng một nhóm so với đối tượng dị biệt sẽ ở trong nhóm khác.
Độ tương đồng càng lớn trong một nhóm kết hợp với độ khác biệt sẽ càng lớn giữa các nhóm có thể tạo ra khoảng cách càng lớn để kết nối giữa các nhóm.

Trong nhiều ứng dụng, ý tưởng để tạo thành một nhóm không được định nghĩa rõ.
Vì vậy, ta cần hiểu hiểu rõ khó khăn của quá trình tạo thành một nhóm.
Cùng nhau xem hình 

\makeatletter % For spaces in paths
\patchcmd\Gread@eps{\@inputcheck#1 }{\@inputcheck"#1"\relax}{}{}
\makeatother
\psscalebox{1.0 1.0} % Change this value to rescale the drawing.
{
\begin{pspicture}(0,-1.2492789)(6.5971155,1.2492789)
\psdots[linecolor=black, dotsize=0.2](0.49855775,0.35072115)
\psdots[linecolor=black, dotsize=0.2](0.8985577,0.75072116)
\psdots[linecolor=black, dotsize=0.2](0.8985577,0.35072115)
\psdots[linecolor=black, dotsize=0.2](0.8985577,0.35072115)
\psdots[linecolor=black, dotsize=0.2](1.2985578,0.75072116)
\psdots[linecolor=black, dotsize=0.2](2.0985577,0.35072115)
\psdots[linecolor=black, dotsize=0.2](2.0985577,-0.04927885)
\psdots[linecolor=black, dotsize=0.2](2.4985578,0.35072115)
\psdots[linecolor=black, dotsize=0.2](0.49855775,-0.44927886)
\psdots[linecolor=black, dotsize=0.2](0.09855774,-0.04927885)
\psdots[linecolor=black, dotsize=0.2](0.09855774,-0.44927886)
\psdots[linecolor=black, dotsize=0.2](4.8985577,1.1507212)
\psdots[linecolor=black, dotsize=0.2](4.4985576,0.75072116)
\psdots[linecolor=black, dotsize=0.2](4.8985577,0.75072116)
\psdots[linecolor=black, dotsize=0.2](4.8985577,-0.04927885)
\psdots[linecolor=black, dotsize=0.2](5.2985578,-0.04927885)
\psdots[linecolor=black, dotsize=0.2](4.4985576,-0.44927886)
\psdots[linecolor=black, dotsize=0.2](4.8985577,-0.44927886)
\psdots[linecolor=black, dotsize=0.2](6.098558,0.75072116)
\psdots[linecolor=black, dotsize=0.2](6.098558,0.35072115)
\psdots[linecolor=black, dotsize=0.2](6.4985576,0.35072115)
\rput[bl](1.6985577,-1.2492789){(a) Các điểm gốc}
\end{pspicture}
}

\section{Các phương pháp gom nhóm văn bản}

\section{Mục tiêu của đồ án}

