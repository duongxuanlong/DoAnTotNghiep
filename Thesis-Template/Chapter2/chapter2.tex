\chapter{Các công trình liên quan}
\label{Chapter2}

%Giới thiệu phương pháp gom nhóm phân cấp

%Các hướng tiếp cận
	%Hướng tiếp cận agglomerative
		%Giới thiệu
		%Ví dụ
	%Hướng tiếp cận divisive	
		%Giới thiệu
		%Ví dụ
		
%Đồng gom nhóm phân cấp
	%Giới hạn của gom nhóm
	%Giới thiệu đồng gom nhóm
	%Ví sao chọn đồng gom nhóm
	%Lợi ích của đồng gom nhóm
	%Sự kết hợp của đồng gom nhóm với gom nhóm phân cấp
	%Công dụng của đồng gom nhóm phân cấp
	%Lợi ích của đồng gom nhóm phân cấp
	%Giải thích việc tương tác hai chiều dữ liệu
	%Kết luận về đồng gom nhóm phân cấp

%%14

%Độ tương đồng Goodman kruskal
	%Giới thiệu về contingency table
	%Ví dụ về contingency table
	%Giới thiệu về độ tương đồng Goodman Kruskal
	%Vì sao chọn độ tương đồng Goodman Kruskal
	%Sử dụng Goodman kruskal trong đồng gom nhóm phân cấp
	%Luật sử dụng cho độ tương đồng Goodman kruskal
		%Giới thiệu
		%Luật thứ nhất
		%Luật thứ hai
	%Ví dụ về cách tính Goodman kruskal
	%Cách để gom nhóm những hàng với nhau
	%Cách để gom nhóm những cột với nhau
%%11

%Thuật toán đồng gom nhóm phân cấp
	%Các ý cần giải thích
		%Dòng, cột, ma trận trong thuật toán(2 đoạn)
		%Mục tiêu của đồng gom nhóm phân cấp
		%Điều kiện của các nhóm trong hàng, cột
		%Cách để gom lại các nhóm trong ma trận
		%Cách để tạo ra phân nhóm đầu tiên(2 đoạn)
	%%7
	%Sử dụng Goodman kruskal trong đồng gom nhóm phân cấp
		%Giới thiệu
		%Công thức
		%Ý tưởng
			%Định nghĩa một
			%Định nghĩa hai
		%Giải thích
		%Cách áp dụng để tính nhóm thích hợp
		%Giải thích cách tính	
		%Sự tối ưu hóa của cách tính
		%Sự lựa chọn nhóm được gom
		%kết quả cuối cùng khi áp dụng
		%kết luận
	%%11
	%Thuật toán
		%Giải thuật 1
		%Giải thuật 2
		%Giải thuật 3
		%Giải thuật 4
		%Giải thích toàn bộ giải thuật của thuật toán(15 đoạn)
	%%19
	%Những lưu ý về đồng gom nhóm phân cấp
		%Các tác động đến đồng góm nhóm phân cấp
		%Sự ảnh hưởng của từng tác động này(6 đoạn)
	%%7
%%44
		
%Đồng gom nhóm phân cấp theo hướng tăng trưởng
	%Điểm hạn chế hiện thời của đồng gom nhóm phân cấp
	%Phân tích tình huống của thuật toán hiện thời
	%Phát triển thuật toán cho dữ liệu tăng trưởng theo thời gian thực
	%hướng cần phải giải quyết
	%Các giải thuật
		%Giải thuật 5
		%Giải thuật 6
	%Giải thích các giải thuật theo hướng tổng quát(15 đoạn)
%%21
%%90	


%%%%%%%%%%%%%%%%%%%%%%%%%%%%%%%%%%%%%%%%%%%%%%%%%%%%%%%%%%%%%%%%%%55
%\section{Các phương pháp gom nhóm}
%%kể ra một vài phương pháp gom nhóm: kmeans, DBScan
%
%\section{Phương pháp gom nhóm phân cấp}
%\subsection{Các hướng tiếp cận}
%\subsubsection{Hướng tiếp cận agglomerative}
%%Giới thiệu hướng tiếp cận agglomerative
%
%%Ví dụ về agglomerative
%
%\subsubsection{Hướng tiếp cận divisive}
%%Giới thiệu hướng tiếp cận divisive
%
%%Ví dụ về divisive
%
%\section{Đồng gom nhóm phân cấp}
%\subsection{Giới thiệu}
%%Điểm hạn chế về gom nhóm
%
%%Giới thiệu về đồng gom nhóm
%
%%Sự kết hợp của đồng gom nhóm phân cấp
%
%%Công dụng của đồng gom nhóm phân cấp
%
%%Giải thích việc tương tác hai chiều của dữ liệu
%
%%Kết luận về đồng gom nhóm phân cấp
%
%\subsection{Độ tương đồng Goodman Kruskal $\tau$}
%%Vì sao chọn Goodman Kruskal $\tau$
%
%%Giới thiệu Goodman Kruskal $\tau$
%
%%Luật sử dụng Goodman Kruskal $\tau$
%
%%Ví dụ về Goodman Kruskal
%
%\subsection{Thuật toán}
%%Giới thiệu thuật toán
%
%%Các giải thuật
%
