\chapter{Các công trình liên quan}
\label{Chapter2}
\section{Các phương pháp gom nhóm}
%kể ra một vài phương pháp gom nhóm: kmeans, DBScan

\section{Phương pháp gom nhóm phân cấp}
\subsection{Các hướng tiếp cận}
\subsubsection{Hướng tiếp cận agglomerative}
%Giới thiệu hướng tiếp cận agglomerative

%Ví dụ về agglomerative

\subsubsection{Hướng tiếp cận divisive}
%Giới thiệu hướng tiếp cận divisive

%Ví dụ về divisive

\section{Đồng gom nhóm phân cấp}
\subsection{Giới thiệu}
%Điểm hạn chế về gom nhóm

%Giới thiệu về đồng gom nhóm

%Sự kết hợp của đồng gom nhóm phân cấp

%Công dụng của đồng gom nhóm phân cấp

%Giải thích việc tương tác hai chiều của dữ liệu

%Kết luận về đồng gom nhóm phân cấp

\subsection{Độ tương đồng Goodman Kruskal $\tau$}
%Vì sao chọn Goodman Kruskal $\tau$

%Giới thiệu Goodman Kruskal $\tau$

%Luật sử dụng Goodman Kruskal $\tau$

%Ví dụ về Goodman Kruskal

\subsection{Thuật toán}
%Giới thiệu thuật toán

%Các giải thuật

