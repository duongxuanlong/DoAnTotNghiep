\chapter{Gom nhóm phân cấp}
\label{Chapter2}

\section{Giới thiệu}
%Giới thiệu phương pháp gom nhóm phân cấp
Gom nhóm phân cấp là kỹ thuật quan trọng trong việc phân loại nhóm. Phương pháp gom nhóm này dùng để xây dựng các nhóm thành các cấp bậc khác nhau.
Điều này khác với phương pháp gom nhóm phân chia khi mà kết quả có được chỉ là những đối tượng nằm ở trong các nhóm riêng biệt.
Ngoài ra, phương pháp gom nhóm phân chia cần phải xác định số lượng nhóm cần phải gom.
Gom nhóm phân cấp khắc phục được nhược điểm này của gom nhóm phân chia đồng thời cho ta thấy được cấu trúc phân cấp của các nhóm trong ngữ liệu.

Gom nhóm phân cấp thường được biểu diễn dưới dạng đồ thị hình cây với tên gọi là dendrogram, thể hiện mối quan hệ giữa nhóm và nhóm con~\cite{Vipin-Kumar}.
Đồ thị này thể hiện thứ tự các nhóm được gom (hướng tiếp cận kết hợp) hoặc là tách (hướng tiếp cận phân rã). 
Cho một tập gồm các điểm 2 chiều, gom nhóm phân cấp có thể biểu diễn các nhóm lồng vào nhau hoặc dưới dạng cây phân cấp.
Hình \ref{fig:pic21} (dựa vào hình 8.13~\cite{Vipin-Kumar}) cho thấy 2 loại biểu diễn của gom nhóm phân cấp cho tập điểm 2 chiều.
Các điểm này được gom nhóm theo kỹ thuật liên kết đơn (\ref{sec:lkd}).

\begin{figure}[htp]
%\makeatletter % For spaces in paths
%\patchcmd\Gread@eps{\@inputcheck#1 }{\@inputcheck"#1"\relax}{}{}
%\makeatother
\psscalebox{1.2 1.2} % Change this value to rescale the drawing.
{
\begin{pspicture}(0,-2.21)(14.42,2.21)
\psline[linecolor=black, linewidth=0.04](0.02,2.19)(0.02,-1.81)(0.02,-1.41)
\psline[linecolor=black, linewidth=0.04](2.02,-1.81)(2.02,-0.21)(2.02,-0.21)
\psline[linecolor=black, linewidth=0.04](3.22,-0.21)(3.22,-1.81)
\psline[linecolor=black, linewidth=0.04](4.42,-1.81)(4.42,0.99)
\psline[linecolor=black, linewidth=0.04](2.02,-0.21)(3.22,-0.21)
\psline[linecolor=black, linewidth=0.04](2.82,-0.21)(2.82,-0.21)
\psline[linecolor=black, linewidth=0.04](2.82,-0.21)(2.82,0.99)(4.42,0.99)
\psline[linecolor=black, linewidth=0.04](3.62,0.99)(3.62,2.19)(0.02,2.19)
\rput[bl](0.02,-2.21){p1}
\rput[bl](2.02,-2.21){p2}
\rput[bl](3.22,-2.21){p3}
\rput[bl](4.42,-2.21){p4}
\psellipse[linecolor=black, linewidth=0.04, dimen=outer](11.02,0.19)(3.4,2.0)
\psellipse[linecolor=black, linewidth=0.04, dimen=outer](12.02,-0.01)(2.0,1.4)
\psellipse[linecolor=black, linewidth=0.04, dimen=outer](11.42,-0.01)(1.0,1.0)
\psdots[linecolor=black, dotsize=0.2](8.82,0.99)
\psdots[linecolor=black, dotsize=0.2](10.82,0.19)
\psdots[linecolor=black, dotsize=0.2](10.82,-0.61)
\psdots[linecolor=black, dotsize=0.2](13.22,-0.21)
\rput[bl](8.82,0.19){p1}
\rput[bl](11.22,-0.61){p2}
\rput[bl](11.22,0.19){p3}
\rput[bl](12.82,-0.61){p4}
\end{pspicture}
}
\caption{Gom nhóm phân cấp với 4 điểm được biểu diễn dưới đồ thị dạng cây và đồ thị hình ellip lồng vào nhau}
\label{fig:pic21}
\end{figure}

\section{Các hướng tiếp cận}
%Các hướng tiếp cận
Để xây dựng nên cấu trúc cây của các nhóm trong gom nhóm phân cấp, ta có hai hướng tiếp cận khác nhau. Đó là hướng tiếp cận phân rã và hướng tiếp cận kết hợp.

\subsection{Gom nhóm phân cấp phân rã (top down hierarchical clustering)}
%Hướng tiếp cận divisive	
%Giới thiệu
Hướng tiếp cận phân rã là một trong những cách tiếp cận của gom nhóm phân cấp~\cite{Vipin-Kumar, hierarchical-clustering, cluster-analysis}.
Tuy nhiên, hướng tiếp cận này ít phổ biến hơn so với hướng tiếp cận kết hợp.
Đây là phương pháp tiếp cận gom nhóm theo cách đi từ trên xuống.
Kỹ thuật này còn có tên gọi khác là tối thiểu hóa khoảng cách của độ tương đồng đồ thị.

Kỹ thuật tối thiểu hóa khoảng cách cây của độ tương đồng đồ thị được xây dựng bắt đầu bằng cách gom tất cả các điểm lại thành một nhóm lớn.
Sau đó ở mỗi bước, ta sẽ tách đôi một nhóm thành 2 nhóm con dựa vào khoảng cách xa nhất của hai nhóm (độ tương đồng thấp nhất).
Thuật toán kết thúc khi mỗi nhóm chỉ còn lại một đơn vị dữ liệu.
Sau đây là thuật toán thể hiện ý tưởng trên:

\begin{algorithm}
\caption{Divisive Hierarchical Clustering Algorithm}
\label{alg:Divisive}
\begin{algorithmic}[1]
\State Compute a minimum spanning tree for the proximity graph.
\Repeat
\State Create a new cluster by breaking the link corresponding to the largest distance (smallest similarity).
\Until{Only singleton clusters remains.}
\end{algorithmic}
\end{algorithm}

Theo như~\cite{wiki-HAC}, thuật toán gom nhóm phân cấp theo hướng phân rã có độ phức tạp rất lớn: $O(2^n)$.

%Phương pháp này bắt đầu từ việc quan sát tất cả chỉ trong một phân nhóm, và khi di chuyển xuống sẽ tách dần dần thành các phân nhóm con.
%Quá trình gộp và tách thường sẽ tốn nhiều chi phí cho thuật toán.
%Khác với hướng tiếp cận kết hợp, hướng tiếp cận phân rã có độ phức tạp lớn hơn, $O(2^n)$.
		
%Ví dụ
		
\subsection{Gom nhóm phân cấp kết hợp (bottom up hierarchical clustering)}
\label{sec:htctt}
%Hướng tiếp cận agglomerative
%Giới thiệu
Gom nhóm phân cấp theo hướng kết hợp là hướng tiếp cận phổ biến nhất trong gom nhóm phân cấp~\cite{Vipin-Kumar, hierarchical-clustering, cluster-analysis}.
Phương pháp này bắt đầu bằng việc xem mỗi điểm như là một nhóm, sau đó dần dần gom các nhóm đơn lẻ này lại với nhau thành nhóm lớn hơn và cho đến khi chỉ còn lại một nhóm duy nhất.
Sau đây là thuật toán mô tả quá trình gom nhóm theo hướng kết hợp.

\begin{algorithm}
\caption{Basic Agglomerative Hierarchical Clustering Algorithm}
\label{agl:agglomerative}
\begin{algorithmic}[1]
\State Compute the proximity graph, if neccessary.
\Repeat
\State Merge the closest two clusters.
\State Update the proximity matrix to reflect the proximity between the new cluster and the original clusters.
\Until{Only one cluster remains.}
\end{algorithmic}
\end{algorithm}

Theo như~\cite{wiki-HAC}, thuật toán gom nhóm phân cấp theo hướng kết hợp có độ phức tạp là $O(n^2\log(n))$.

Đã có nhiều giải pháp đề xuất để cải thiện thuật toán gom nhóm phân cấp kết hợp.
Một trong số đó là sử dụng ràng buộc~\cite{hac-constraints} để làm giảm các phép tính toán và tăng nhanh quá trình thực thi cho thuật toán gom nhóm phân cấp.
Khi ta sử dụng gom nhóm phân cấp kết hợp cùng với các ràng buộc, ngoài việc sử dụng kỹ thuật liên kết thì ta còn áp dụng thêm một tập các ràng buộc để quyết định xem hai nhóm có được gom vào nhau hay không?
Việc hạn chế gom nhóm sẽ giúp ta làm giảm được các phép toán trong thuật toán.
Tuy nhiên, hướng tiếp cận này cần phải giải quyết được 2 vấn đề có thể phát sinh ra lỗi trong thuật toán.
Vấn đề thứ nhất là liệu các ràng buộc có dẫn thuật toán gặp phải tình huống np-complete~\cite{np-complete} hay không?
Vấn đề thứ hai là các ràng buộc có khả năng dẫn cây phân cấp không thể cập nhật thêm được nhóm mới hay không?
Bài viết~\cite{hac-constraints} đã liệt kê kết quả của sự kết hợp giữa các ràng buộc để tránh gặp phải 2 vấn đề trên.
Ngoài ra, bài viết còn đề xuất thêm ràng buộc $\gamma$ để giúp giảm tính toán cho thuật toán.

Khi gom nhóm phân cấp kết hợp sử dụng kỹ thuật liên kết đơn\ref{sec:lkd} để tìm độ tương đồng giữa các nhóm thì những nhóm ở gần nhau thường có xu hướng được gom vào nhau.
Dựa vào đặc tính này, một phương pháp được đề xuất để tìm kiếm nhanh các nhóm gần kề bằng việc sử dụng Locality-Sennsitive Hashing (LSH)~\cite{single-link-hash}.
LSH sẽ băm tất cả các điểm vào các khu vực khác nhau.
Khi ta muốn tìm kiếm những điểm lân cận với điểm $p$ bất kì, ta sẽ tìm trong những khu vực băm mà $p$ thuộc về và điểm nào ở gần với $p$ nhất sẽ được gom vào nhau thành một nhóm.
Quá trình tìm kiếm những điểm ở gần với $p$ trong những khu vực băm sẽ nhanh hơn rất nhiều so với việc phải tìm kiếm tất cả các điểm trong dữ liệu nên độ phức tạp của thuật toán sẽ được giảm xuống rất nhiều, chỉ còn $O(nB)$ với $B$ là số điểm nhiều nhất nằm trong khu vực băm.
Qua thực nghiệm, nhóm tác giả đã cho thấy kết quả sau khi gom nhóm phân cấp kết hợp với kỹ thuật liên kết đơn kết hợp với LSH giống với kết quả khi không có sử dụng LSH nhưng thuật toán lại chạy nhanh hơn và có thể chạy trên được dữ liệu lớn hơn.

\section{Độ phức tạp và lưu trữ của thuật toán gom nhóm phân cấp kết hợp}
\label{sec:dpt}
%~\cite{Vipin-Kumar}
Thuật toán gom nhóm phân cấp kết hợp sử dụng ma trận tương đồng (ma trận khoảng cách) để xem xét độ tương đồng giữa các nhóm.
Ma trận tương đồng cần khoảng $\frac{1}{2} m^2$ dung lượng để lưu trữ (xem ma trận tương đồng là ma trận vuông) với $m$ là số điểm dữ liệu.
Thuật toán cũng cần khoảng trống cần thiết để đánh dấu tỷ lệ các nhóm được gom với tổng số nhóm và tốn khoảng $m - 1$.
Vì thế, dung lượng cần thiết để chạy thuật toán gom nhóm phân cấp kết hợp là $O(m^2)$.

Phân tích dành cho hướng tiếp cận cơ bản của gom nhóm phân cấp kết hợp cho thấy độ phức tạp của thuật toán liên quan trực tiếp đến quá trình tính toán.
$O(m^2)$ là thời gian cần thiết để tính ma trận tương đồng.
Sau khi tính xong ma trận tương đồng, dựa vào thuật toán \ref{agl:agglomerative}, ta còn $m - 1$ lần lặp cho bước 3 và 4 vì ta có $m$ nhóm lúc ban đầu và mỗi lần lặp thì chỉ có 2 nhóm được gom lại thành một.
Nếu ta thực thi tìm kiếm tuyến tính cho ma trận tương đồng thì sau lần lặp thứ $i$, thời gian cần dành cho ở bước 3 sẽ là $O((m - i + 1)^2)$ tỷ lệ với bình phương của số lượng nhóm.
Thời gian cần để chạy bước thứ 4 là $O(m - i + 1)$ để cập nhật ma trận tương đồng sau khi gom 2 nhóm (một nhóm được gom lại chỉ mất khoảng $O(m - i + 1)$).
Nếu như không có thay đổi, thuật toán có độ phức tạp là $O(m^3)$.
Trong trường hợp khoảng cách từ nhóm này đến các nhóm khác được lưu trữ theo thứ tự thì có khả năng làm giảm khoảng $O(m - i + 1)$ dành thời gian tìm kiếm 2 nhóm gần nhất.
Trường hợp tổng quát cho độ phức tạp của thuật toán \ref{agl:agglomerative} là $O(m^2 \log m)$.

Dựa vào phân tích trên, thuật toán gom nhóm phân cấp kết hợp có độ phức tạp khá cao và tốn nhiều dung lượng để triển khai.
Vì tốn nhiều dung lượng để thực thi nên thuật toán gặp nhiều khó khăn khi cần gom nhóm dữ liệu lớn.
Chính vì vậy, không có gì ngạc nhiên khi các phương pháp đề xuất để cải thiện thuật toán như đã đề cập trong phần \ref{sec:htctt} đa phần tập trung vào việc làm giảm độ phức tạp~\cite{hac-constraints, single-link-hash} để giúp cho thuật toán chạy nhanh hơn hoặc là giảm không gian lưu trữ lúc thực thi để có thể mở rộng giới hạn dữ liệu.
Như đã đề cập trong phần \ref{sec:mtcda}, đồ án sẽ sử dụng doc2vec làm thể hiện cho văn bản.
Doc2vec được sử dụng để rút ngắn số chiều của vector trong văn bản và thu nhỏ không gian lưu trữ lúc thực thi nên sẽ giúp mở rộng dữ liệu gom nhóm.
Đây là hướng tiếp cận được đề xuất để làm giảm không gian lưu trữ trong lúc thực thi thuật toán.

\section{Các phương pháp liên kết trong gom nhóm phân cấp kết hợp}
\label{sec:cpplk}
Trong thuật toán \ref{agl:agglomerative}, mấu chốt của vấn đề là cách tính độ tương đồng giữa các nhóm~\cite{AHC, Vipin-Kumar}.
Cách tính độ tương đồng khác nhau có thể dẫn đến kết quả khác nhau của gom nhóm phân cấp kết hợp.
Độ tương đồng của các nhóm được tính bằng cách sử dụng nhiều kỹ thuật liên kết khác nhau.
Các kỹ thuật liên kết thường có trong gom nhóm phân cấp kết hợp: liên kết đơn (còn có tên gọi khác là nhỏ nhất), liên kết toàn vẹn (hay còn gọi là lớn nhất) và liên kết nhóm trung bình.
Liên kết đơn định nghĩa độ tương đồng của nhóm là độ tương đồng giữa 2 điểm gần nhất ở 2 nhóm khác nhau, nghĩa là khoảng cách gần nhất giữa hai nốt của hai tập khác nhau.
Liên kết đơn tạo ra các nhóm liền kề như trong hình \ref{fig:pic14}.
Hình \ref{fig:pic22} (dựa vào hình 8.14(a)~\cite{Vipin-Kumar}) cho ta thấy gom nhóm phân cấp kết hợp theo kỹ thuật liên kết đơn.
\begin{figure}[htp]
\psscalebox{1.0 1.0} % Change this value to rescale the drawing.
{
\begin{pspicture}(0,-2.4)(12.4,2.4)
\psellipse[linecolor=black, linewidth=0.04, dimen=outer](2.0,0.4)(2.0,1.2)
\pscircle[linecolor=black, linewidth=0.04, dimen=outer](10.0,0.0){2.4}
\psdots[linecolor=black, dotsize=0.2](2.0,1.2)
\psdots[linecolor=black, dotsize=0.2](1.2,-0.4)
\psdots[linecolor=black, dotsize=0.2](8.4,0.0)
\psdots[linecolor=black, dotsize=0.2](9.6,1.6)
\psdots[linecolor=black, dotsize=0.2](11.2,-1.2)
\psdots[linecolor=black, dotsize=0.2](3.2,0.4)
\psline[linecolor=black, linewidth=0.04, linestyle=dotted, dotsep=0.10583334cm](3.2,0.4)(8.4,0.0)(8.4,0.0)
\end{pspicture}
}
\caption{Liên kết đơn}
\label{fig:pic22}
\end{figure}

Trong khi đó, liên kết toàn vẹn tính độ tương đồng của nhóm dựa vào 2 điểm xa nhất ở 2 hai nhóm khác nhau, nghĩa là khoảng cách xa nhất giữa 2 nốt của 2 tập khác nhau.
Hình \ref{fig:pic23} (dựa vào hình 8.14(b)~\cite{Vipin-Kumar}) cho ta thấy kỹ thuật liên kết hoàn toàn.
\begin{figure}[htp]
\psscalebox{1.0 1.0} % Change this value to rescale the drawing.
{
\begin{pspicture}(0,-2.4)(12.4,2.4)
\psellipse[linecolor=black, linewidth=0.04, dimen=outer](2.0,0.4)(2.0,1.2)
\pscircle[linecolor=black, linewidth=0.04, dimen=outer](10.0,0.0){2.4}
\psdots[linecolor=black, dotsize=0.2](2.0,1.2)
\psdots[linecolor=black, dotsize=0.2](1.2,-0.4)
\psdots[linecolor=black, dotsize=0.2](8.4,0.0)
\psdots[linecolor=black, dotsize=0.2](9.6,1.6)
\psdots[linecolor=black, dotsize=0.2](11.2,-1.2)
\psdots[linecolor=black, dotsize=0.2](3.2,0.4)
\psline[linecolor=black, linewidth=0.04, linestyle=dotted, dotsep=0.10583334cm](1.2,-0.4)(1.2,-0.4)(11.2,-1.2)
\end{pspicture}
}
\caption{Liên kết toàn vẹn}
\label{fig:pic23}
\end{figure}

Một hướng tiếp cận khác là kỹ thuật gom nhóm trung bình với định nghĩa độ tương đồng là khoảng cách trung bình của tất cả các điểm trong hai nhóm.
Hình \ref{fig:pic24} (dựa vào hình 8.14(c)~\cite{Vipin-Kumar}) cho ta thấy kỹ thuật nhóm trung bình.
\begin{figure}[htp]
\psscalebox{1.0 1.0} % Change this value to rescale the drawing.
{
\begin{pspicture}(0,-2.4)(12.4,2.4)
\psellipse[linecolor=black, linewidth=0.04, dimen=outer](2.0,0.4)(2.0,1.2)
\pscircle[linecolor=black, linewidth=0.04, dimen=outer](10.0,0.0){2.4}
\psdots[linecolor=black, dotsize=0.2](2.0,1.2)
\psdots[linecolor=black, dotsize=0.2](1.2,-0.4)
\psdots[linecolor=black, dotsize=0.2](8.4,0.0)
\psdots[linecolor=black, dotsize=0.2](9.6,1.6)
\psdots[linecolor=black, dotsize=0.2](11.2,-1.2)
\psdots[linecolor=black, dotsize=0.2](3.2,0.4)
\psline[linecolor=black, linewidth=0.04, linestyle=dotted, dotsep=0.10583334cm](1.2,-0.4)(1.2,-0.4)(11.2,-1.2)
\psline[linecolor=black, linewidth=0.04, linestyle=dotted, dotsep=0.10583334cm](1.2,-0.4)(8.4,0.0)
\psline[linecolor=black, linewidth=0.04, linestyle=dotted, dotsep=0.10583334cm](1.2,-0.4)(9.6,1.6)
\psline[linecolor=black, linewidth=0.04, linestyle=dotted, dotsep=0.10583334cm](2.0,1.2)(9.6,1.6)
\psline[linecolor=black, linewidth=0.04, linestyle=dotted, dotsep=0.10583334cm](2.0,1.2)(8.4,0.0)
\psline[linecolor=black, linewidth=0.04, linestyle=dotted, dotsep=0.10583334cm](2.0,1.2)(11.2,-1.2)
\psline[linecolor=black, linewidth=0.04, linestyle=dotted, dotsep=0.10583334cm](3.2,0.4)(9.6,1.6)
\psline[linecolor=black, linewidth=0.04, linestyle=dotted, dotsep=0.10583334cm](3.2,0.4)(8.4,0.0)
\psline[linecolor=black, linewidth=0.04, linestyle=dotted, dotsep=0.10583334cm](3.2,0.4)(11.2,-1.2)
\end{pspicture}
}
\caption{Nhóm trung bình}
\label{fig:pic24}
\end{figure}

Ngoài ra, ta có thể gom nhóm phân cấp kết hợp dựa vào mẫu khi ta xem một nhóm được đại diện bởi điểm trung tâm thì khi đó độ tương đồng giữa 2 nhóm sẽ trở thành độ tương đồng giữa 2 điểm trung tâm.
Một cách tiếp cận khác có tên gọi là Ward cũng gần như xem một nhóm được đại diện bởi điểm trung tâm.
Tuy nhiêm, phương pháp Ward tính độ tương đồng giữa hai nhóm dựa vào điều kiện gia tăng trong SSE~\cite{SSE} từ kết quả của quá trình gom nhóm.
%Tương tự như K-means, phương pháp Ward cũng cố gắng giảm tổng bình phương của khoảng cách từ điểm đến điểm trung tâm của nhóm.

Sau đây là phần dữ liệu mẫu được thể hiện qua biểu đồ \ref{fig:pic25}, bảng \ref{tab:2_1} và bảng \ref{tab:2_2} với mục đích dùng để mô tả các kỹ thuật liên kết khác nhau trong gom nhóm phân cấp kết hợp.
\begin{figure}[htp]
\psscalebox{1.0 1.0} % Change this value to rescale the drawing.
{
\begin{pspicture}(0,-3.9618518)(8.025,3.9618518)
\psline[linecolor=black, linewidth=0.04](0.7038889,3.7418518)(0.7038889,-3.3692594)(0.7038889,-3.3692594)(7.815,-3.3692594)
\psline[linecolor=black, linewidth=0.04, linestyle=dotted, dotsep=0.10583334cm](1.8890741,3.7418518)(1.8890741,-3.3692594)
\psline[linecolor=black, linewidth=0.04, linestyle=dotted, dotsep=0.10583334cm](3.0742593,-3.3692594)(3.0742593,3.7418518)
\psline[linecolor=black, linewidth=0.04, linestyle=dotted, dotsep=0.10583334cm](4.259444,3.7418518)(4.259444,-3.3692594)
\psline[linecolor=black, linewidth=0.04, linestyle=dotted, dotsep=0.10583334cm](5.4446297,3.7418518)(5.4446297,-3.3692594)
\psline[linecolor=black, linewidth=0.04, linestyle=dotted, dotsep=0.10583334cm](6.6298146,-3.3692594)(6.6298146,3.7418518)(6.6298146,3.7418518)
\psline[linecolor=black, linewidth=0.04, linestyle=dotted, dotsep=0.10583334cm](7.815,3.7418518)(7.815,-3.3692594)
\psline[linecolor=black, linewidth=0.04, linestyle=dotted, dotsep=0.10583334cm](0.7038889,-2.1840742)(7.815,-2.1840742)
\psline[linecolor=black, linewidth=0.04, linestyle=dotted, dotsep=0.10583334cm](0.7038889,-0.9988889)(7.815,-0.9988889)
\psline[linecolor=black, linewidth=0.04, linestyle=dotted, dotsep=0.10583334cm](0.7038889,0.1862963)(7.815,0.1862963)
\psline[linecolor=black, linewidth=0.04, linestyle=dotted, dotsep=0.10583334cm](0.7038889,1.3714815)(7.815,1.3714815)
\psline[linecolor=black, linewidth=0.04, linestyle=dotted, dotsep=0.10583334cm](0.7038889,2.5566666)(7.815,2.5566666)
\psline[linecolor=black, linewidth=0.04, linestyle=dotted, dotsep=0.10583334cm](0.7038889,3.7418518)(7.815,3.7418518)
\rput[b](0.4075926,-3.6655555){0}
\rput[b](1.8890741,-3.9618518){0.1}
\rput[b](3.0742593,-3.9618518){0.2}
\rput[b](4.259444,-3.9618518){0.3}
\rput[b](5.4446297,-3.9618518){0.4}
\rput[b](6.6298146,-3.9618518){0.5}
\rput[b](7.815,-3.9618518){0.6}
\rput[b](0.215,-2.1840742){0.1}
\rput[b](0.215,-0.9988889){0.2}
\rput[b](0.215,0.1862963){0.3}
\rput[b](0.215,1.3714815){0.4}
\rput[b](0.215,2.5566666){0.5}
\rput[b](0.215,3.7418518){0.6}
\psdots[linecolor=black, dotsize=0.2](1.5927777,1.6677778)
\psdots[linecolor=black, dotsize=0.2](3.3705556,1.0751852)
\psdots[linecolor=black, dotsize=0.2](5.4446297,2.852963)
\psdots[linecolor=black, dotsize=0.2](4.852037,0.48259258)
\psdots[linecolor=black, dotsize=0.2](6.0372224,0.1862963)
\psdots[linecolor=black, dotsize=0.2](3.6668518,-1.2951852)
\rput[b](1.8890741,1.6677778){5}
\rput[b](3.6668518,1.0751852){2}
\rput[b](5.740926,2.852963){1}
\rput[b](5.1483335,0.48259258){3}
\rput[b](6.3335185,0.1862963){6}
\rput[b](3.963148,-1.5914814){4}
\end{pspicture}
}
\caption{Đồ thị thể hiện tọa độ giữa các điểm}
\label{fig:pic25}
\end{figure}

\begin{table}[h!]
\centering
\caption{Bảng thể hiện tọa độ của các điểm}
\label{tab:2_1}
\begin{tabular}{|c|c|c|}
\hline
Điểm & x & y \\ \hline
p1    & 0.40         & 0.53         \\ \hline
p2    & 0.22         & 0.38         \\ \hline
p3    & 0.35         & 0.32         \\ \hline
p4    & 0.26         & 0.19         \\ \hline
p5    & 0.08         & 0.41         \\ \hline
p6    & 0.45         & 0.30         \\ \hline
\end{tabular}
\end{table}

\begin{table}[h!]
\centering
\caption{Bảng thể hiện độ đo khoảng cách Euclean giữa các điểm}
\label{tab:2_2}
\begin{tabular}{|c|c|c|c|c|c|c|}
\hline
   & p1   & p2   & p3   & p4   & p5   & p6   \\ \hline
p1 & 0.00 & 0.24 & 0.22 & 0.37 & 0.34 & 0.23 \\ \hline
p2 & 0.24 & 0.00 & 0.15 & 0.2  & 0.14 & 0.25 \\ \hline
p3 & 0.22 & 0.15 & 0.00 & 0.15 & 0.28 & 0.11 \\ \hline
p4 & 0.37 & 0.2  & 0.15 & 0.00 & 0.29 & 0.22 \\ \hline
p5 & 0.34 & 0.14 & 0.28 & 0.29 & 0.00 & 0.39 \\ \hline
p6 & 0.23 & 0.25 & 0.11 & 0.22 & 0.39 & 0.00 \\ \hline
\end{tabular}%
\end{table}

\subsection{Liên kết đơn}	
\label{sec:lkd}	
%Giói thiệu
Liên kết đơn hay kỹ thuật nhỏ nhất trong gom nhóm phân cấp kết hợp là phương pháp xem độ tương đồng của hai nhóm là khoảng cách gần nhất của hai điểm bất kì trong mỗi nhóm~\cite{Vipin-Kumar, HAC, AHC, hierarchical-clustering, single-complete, explain-HAC}. 
Kỹ thuật này bắt đầu với tất cả các điểm đơn lẻ được xem như là một nhóm.
Sau đó, ta sẽ lần lượt tính khoảng cách giữa các điểm, chọn khoảng cách ngắn nhất giữa hai điểm thuộc hai nhóm khác nhau rồi gom 2 nhóm này lại thành một nhóm.
Thuật toán sẽ thực thi cho đến khi chỉ còn lại một nhóm duy nhất.
Liên kết đơn là kỹ thuật mạnh trong xử lí dữ liệu có hình dạng phi ellip nhưng có nhược điểm là nhạy cảm với độ nhiễu và những điểm tách biệt.

%Chi tiết
Nhìn vào hình \ref{fig:pic26} (dựa vào hình 8.16~\cite{Vipin-Kumar}), ta thấy được kết quả của việc ứng dụng kỹ thuật liên kết đơn trong gom nhóm phân cấp kết hợp đối với dữ liệu ở \ref{fig:pic25}.
Phần bên trái của hình \ref{fig:pic26} là một chuỗi các vòng ellip lồng vào nhau.
Ở mỗi vòng ellip, một con số thứ tự xuất hiện để thể hiện các nhóm được gom ở bước tương ứng.
Phần bên phải của hình \ref{fig:pic26} có thông tin tương tự như phần bên trái nhưng được biểu diễn dưới dạng đồ thị.
Chiều cao mà hai nhóm gom lại vào nhau trong đồ thị thể hiện khoảng cách của hai nhóm.
Ví dụ, từ bảng \ref{tab:2_2}, ta có thể thấy khoảng cách giữa 2 điểm 3 và 6 là 0.11 và đây cũng là khoảng cách ngắn nhất giữa các điểm để hai nhóm sát nhập vào nhau thành một. 
Một ví dụ khác, khoảng cách giữa các nhóm \{3, 6\} và \{2, 5\} được tính  bằng cách
\begin{equation}
\begin{aligned}
dist(\{3, 6\}, \{2, 5\})
&= min(dist(3,2),dist(6, 2), dist(3, 5), dist(6, 5)) \\
&= min (0.148, 0.254, 0.284, 0.392) = 0.148. \\
\end{aligned}
\end{equation}


\begin{figure}[htp]
\psscalebox{1.0 1.0} % Change this value to rescale the drawing.
{
\begin{pspicture}(0,-3.6)(16.92,3.6)
\pscircle[linecolor=black, linewidth=0.04, dimen=outer](3.6,0.0){3.6}
\psellipse[linecolor=black, linewidth=0.04, dimen=outer](3.6,0.0)(3.2,2.8)
\psellipse[linecolor=black, linewidth=0.04, dimen=outer](3.6,0.4)(2.8,2.0)
\psellipse[linecolor=black, linewidth=0.04, dimen=outer](2.6,1.0)(1.4,0.6)
\psellipse[linecolor=black, linewidth=0.04, dimen=outer](4.2,-0.6)(1.4,0.6)
\rput[bl](3.6,-3.2){4}
\rput[bl](6.8,2.4){5}
\rput[bl](2.8,2.0){2}
\rput[bl](4.4,1.6){3}
\rput[bl](4.8,0.4){1}
\psdots[linecolor=black, dotsize=0.2](3.2,3.2)
\psdots[linecolor=black, dotsize=0.2](1.6,1.2)
\psdots[linecolor=black, dotsize=0.2](2.8,0.8)
\psdots[linecolor=black, dotsize=0.2](3.2,-0.4)
\psdots[linecolor=black, dotsize=0.2](4.4,-0.8)
\psdots[linecolor=black, dotsize=0.2](3.2,-2.4)
\rput[bl](3.6,3.2){1}
\rput[bl](3.2,0.8){2}
\rput[bl](3.6,-0.4){3}
\rput[bl](4.8,-0.8){6}
\rput[bl](3.6,-2.4){4}
\rput[bl](2.0,1.2){5}
\psline[linecolor=black, linewidth=0.04](9.6,1.6)(9.6,-3.2)(16.8,-3.2)
\psline[linecolor=black, linewidth=0.04](9.6,-2.0)(10.0,-2.0)(10.0,-2.0)
\psline[linecolor=black, linewidth=0.04](9.6,-0.8)(10.0,-0.8)
\psline[linecolor=black, linewidth=0.04](9.6,0.4)(10.0,0.4)
\psline[linecolor=black, linewidth=0.04](9.6,1.6)(10.0,1.6)
\rput[bl](8.4,-2.0){0.05}
\rput[bl](8.4,-0.8){0.1}
\rput[bl](8.4,0.4){0.15}
\rput[bl](8.4,1.6){0.2}
\rput[bl](10.8,-3.6){3}
\rput[bl](12.0,-3.6){6}
\rput[bl](13.2,-3.6){2}
\rput[bl](14.4,-3.6){5}
\rput[bl](15.6,-3.6){4}
\rput[bl](16.8,-3.6){1}
\psline[linecolor=black, linewidth=0.02](10.8,-3.2)(10.8,-0.4)(12.0,-0.4)(12.0,-3.2)
\psline[linecolor=black, linewidth=0.02](13.2,-3.2)(13.2,0.0)(14.4,0.0)(14.4,-3.2)
\psline[linecolor=black, linewidth=0.02](11.2,-0.4)(11.2,0.4)(14.0,0.4)(14.0,0.0)
\psline[linecolor=black, linewidth=0.02](15.6,-3.2)(15.6,0.8)(12.4,0.8)(12.4,0.4)
\psline[linecolor=black, linewidth=0.02](14.0,0.8)(14.0,2.0)(16.8,2.0)(16.8,-3.2)
\psline[linecolor=black, linewidth=0.04](9.6,1.6)(9.6,3.2)(9.6,3.2)
\end{pspicture}
}
\caption{Kết quả khi ứng dụng liên kết đơn}
\label{fig:pic26}
\end{figure}

%Nhược điểm

\subsection{Liên kết toàn vẹn}		
%Giới thiệu
Liên kết toàn vẹn còn được gọi là kỹ thuật lớn nhất hoặc là CLIQUE~\cite{Vipin-Kumar, HAC, AHC, hierarchical-clustering, single-complete}.
Liên kết toàn vẹn tính độ tương đồng của hai nhóm dựa vào khoảng cách lớn nhất (độ tương đồng nhỏ nhất) của hai điểm bất kì của hai nhóm khác nhau.
Kỹ thuật này bắt đầu với tất cả các điểm đơn lẻ được xem như là một nhóm.
Sau đó, ta sẽ lần lượt tính khoảng cách giữa các điểm thuộc các nhóm khác nhau và chọn những khoảng cách xa nhất giữa các nhóm.
Trong số những khoảng cách xa nhất, ta sẽ chọn khoảng cách có giá trị nhỏ nhất và gom 2 nhóm có khoảng cách này thành một.
Thuật toán sẽ thực thi cho đến khi các nhóm liên kết hoàn toàn.
Liên kết toàn vẹn thường không nhạy cảm với độ nhiễu và điểm tách biệt, nhưng lại có thể chia cắt nhóm lớn và có xu hướng tạo thành nhóm toàn cục.

%Chi tiết
Hình \ref{fig:pic27} (dựa vào hình 8.17~\cite{Vipin-Kumar}) cho ta thấy được kết quả của việc áp dụng kỹ thuật liên kết toàn vẹn trong gom nhóm phân cấp kết hợp.
Một lần nữa, điểm 3 và 6 được gom vào thành một nhóm trước.
Tuy nhiên, \{3, 6\} lại được gom với \{4\} thay vì gom với \{2, 5\}.
Điều này xảy ra vì:

\begin{equation}
\begin{aligned}
dist(\{3, 6\}, \{4\})
&= max(dist(3, 4), dist(6, 4)) 		\\
&= max(0.15, 0.22) 					\\
&= 0.22.							\\
dist(\{3, 6\}, \{2, 5\})
&= max(dist(3, 2), dist(6, 2), dist(3, 5), dist(6, 5))			\\
&= max(0.15, 0.25, 0.28, 0.39)									\\
&= 0.392.										\\					
dist(\{3, 6\}, \{1\})
&= max(dist(3, 1), dist(6, 1)) =	\\ 
&= max(0.22, 0.23) = 0.23.		\\
\end{aligned}
\end{equation}

\begin{figure}[htp]
\psscalebox{1.0 1.0} % Change this value to rescale the drawing.
{
\begin{pspicture}(0,-3.6)(16.95,3.6)
\pscircle[linecolor=black, linewidth=0.04, dimen=outer](3.6,0.0){3.6}
\psellipse[linecolor=black, linewidth=0.04, dimen=outer](2.6,1.0)(1.4,0.6)
\psellipse[linecolor=black, linewidth=0.04, dimen=outer](5.0,-0.6)(1.4,0.6)
\rput[bl](2.4,2.8){4}
\rput[bl](6.8,2.4){5}
\rput[bl](1.6,1.6){2}
\rput[bl](2.0,-1.2){3}
\rput[bl](4.8,-1.6){1}
\psdots[linecolor=black, dotsize=0.2](3.6,2.4)
\psdots[linecolor=black, dotsize=0.2](1.6,1.2)
\psdots[linecolor=black, dotsize=0.2](2.8,0.8)
\psdots[linecolor=black, dotsize=0.2](4.0,-0.4)
\psdots[linecolor=black, dotsize=0.2](5.6,-0.8)
\psdots[linecolor=black, dotsize=0.2](3.2,-2.4)
\rput[bl](4.0,2.4){1}
\rput[bl](3.2,0.8){2}
\rput[bl](4.4,-0.4){3}
\rput[bl](5.2,-0.8){6}
\rput[bl](3.6,-2.4){4}
\rput[bl](2.0,1.2){5}
\psline[linecolor=black, linewidth=0.04](9.6,1.6)(9.6,-3.2)(16.8,-3.2)
\psline[linecolor=black, linewidth=0.04](9.6,-2.0)(10.0,-2.0)(10.0,-2.0)
\psline[linecolor=black, linewidth=0.04](9.6,-0.8)(10.0,-0.8)
\psline[linecolor=black, linewidth=0.04](9.6,0.4)(10.0,0.4)
\psline[linecolor=black, linewidth=0.04](9.6,1.6)(10.0,1.6)
\rput[bl](8.4,-2.0){0.1}
\rput[bl](8.4,-0.8){0.2}
\rput[bl](8.4,0.4){0.3}
\rput[bl](8.4,1.6){0.4}
\rput[bl](10.8,-3.6){3}
\rput[bl](12.0,-3.6){6}
\rput[bl](13.2,-3.6){4}
\rput[bl](14.4,-3.6){1}
\rput[bl](15.6,-3.6){2}
\rput[bl](16.8,-3.6){5}
\psline[linecolor=black, linewidth=0.04](9.6,1.6)(9.6,3.2)(9.6,3.2)
\psellipse[linecolor=black, linewidth=0.04, dimen=outer](2.8,1.6)(2.0,1.6)
\rput{36.308563}(0.025317831,-2.8772123){\psellipse[linecolor=black, linewidth=0.04, dimen=outer](4.4,-1.4)(2.8,1.4)}
\psline[linecolor=black, linewidth=0.04](10.8,-3.2)(10.8,-2.0)(12.0,-2.0)(12.0,-3.2)
\psline[linecolor=black, linewidth=0.04](11.2,-2.0)(11.2,-0.4)(13.2,-0.4)(13.2,-3.2)
\psline[linecolor=black, linewidth=0.04](15.6,-3.2)(15.6,-1.2)(16.8,-1.2)(16.8,-3.2)
\psline[linecolor=black, linewidth=0.04](14.4,-3.2)(14.4,0.8)(16.0,0.8)(16.0,-1.2)
\psline[linecolor=black, linewidth=0.04](12.0,-0.4)(12.0,1.6)(15.2,1.6)(15.2,0.8)
\end{pspicture}
}
\caption{Kết quả khi ứng dụng liên kết toàn vẹn}
\label{fig:pic27}
\end{figure}

\subsection{Nhóm trung bình}
\label{sec:ntb}
Trong kỹ thuật nhóm trung bình của gom nhóm phân cấp kết hợp, độ tương đồng của hai nhóm được định nghĩa là độ trung bình của các cặp điểm ở 2 nhóm khác nhau~\cite{Vipin-Kumar, HAC, AHC, hierarchical-clustering, average}.
Đây là hướng tiếp cận trung gian giữa kỹ thuật nhỏ nhất và kỹ thuật lớn nhất.
Độ tương đồng của nhóm trung bình $proximity(C_i, C_j)$ của nhóm $C_i$ và $C_j$, với độ lớn (số lượng điểm) là $m_i$ và $m_j$ được biểu diễn thành công thức:
\begin{equation}
proximity(C_i, C_j) = \sum_{\substack{x \in C_i \\ y \in C_j}} \frac{proximity(x, y)}{m_i * m_j}
\end{equation}

Hình \ref{fig:pic28} (dựa vào hình 8.18~\cite{Vipin-Kumar}) thể hiện kết quả của gom nhóm phân cấp kết hợp theo kỹ thuật nhóm trung bình.
Để thấy được cách hoạt động của nhóm trung bình, ta tính khoảng cách trung bình giữa các nhóm:
\begin{equation}
\begin{aligned}
dist(\{3, 6, 4\}, \{1\})
&= (0.221 + 0.368 + 0.234) / (3 * 1)	\\ 
&= 0.28									\\
dist(\{2, 5\}, \{1\})
&= (0.235 + 0.342) / (2 * 1)			\\
&= 0.2889								\\
dist(\{3, 6, 4\}, \{2, 5\})				
&= (0.148 + 0.284 + 0.254 + 0.392 + 0.204 + 0.2932) /  (6 * 2)	\\
&= 0.26									\\
\end{aligned}
\end{equation}

Do $dist(\{3, 6, 4\}, \{2, 5\})$ nhỏ hơn $dist(\{3, 6, 4\}, \{1\})$ và $dist(\{2, 5\}, \{1\})$, nên $\{3, 6, 4\}$ và $\{2, 5\}$ sẽ được gom lại thành một nhóm mới ở bước thứ tư.

\begin{figure}[htp]
\psscalebox{1.0 1.0} % Change this value to rescale the drawing.
{
\begin{pspicture}(0,-3.6)(16.92,3.6)
\pscircle[linecolor=black, linewidth=0.04, dimen=outer](3.6,0.0){3.6}
\psellipse[linecolor=black, linewidth=0.04, dimen=outer](2.6,1.4)(1.4,0.6)
\psellipse[linecolor=black, linewidth=0.04, dimen=outer](4.6,0.2)(1.4,0.6)
\rput[bl](4.8,-2.8){4}
\rput[bl](6.8,2.4){5}
\rput[bl](2.8,2.0){2}
\rput[bl](1.6,-0.4){3}
\rput[bl](4.4,-0.8){1}
\psdots[linecolor=black, dotsize=0.2](3.6,3.2)
\psdots[linecolor=black, dotsize=0.2](1.6,1.6)
\psdots[linecolor=black, dotsize=0.2](2.8,1.2)
\psdots[linecolor=black, dotsize=0.2](3.6,0.4)
\psdots[linecolor=black, dotsize=0.2](5.2,0.0)
\psdots[linecolor=black, dotsize=0.2](2.8,-1.6)
\rput[bl](4.0,3.2){1}
\rput[bl](3.2,1.2){2}
\rput[bl](4.0,0.4){3}
\rput[bl](4.8,0.0){6}
\rput[bl](3.2,-1.6){4}
\rput[bl](2.0,1.6){5}
\psline[linecolor=black, linewidth=0.04](9.6,1.6)(9.6,-3.2)(16.8,-3.2)
\psline[linecolor=black, linewidth=0.04](9.6,-2.0)(10.0,-2.0)(10.0,-2.0)
\psline[linecolor=black, linewidth=0.04](9.6,-0.8)(10.0,-0.8)
\psline[linecolor=black, linewidth=0.04](9.6,0.4)(10.0,0.4)
\psline[linecolor=black, linewidth=0.04](9.6,1.6)(10.0,1.6)
\rput[bl](8.4,-2.0){0.05}
\rput[bl](8.4,-0.8){0.1}
\rput[bl](8.4,0.4){0.15}
\rput[bl](8.4,1.6){0.2}
\rput[bl](10.8,-3.6){3}
\rput[bl](12.0,-3.6){6}
\rput[bl](13.2,-3.6){4}
\rput[bl](14.4,-3.6){2}
\rput[bl](15.6,-3.6){5}
\rput[bl](16.8,-3.6){1}
\psline[linecolor=black, linewidth=0.04](9.6,1.6)(9.6,3.2)(9.6,3.2)
\rput{36.308563}(0.42136064,-2.4850307){\psellipse[linecolor=black, linewidth=0.04, dimen=outer](4.0,-0.6)(2.8,1.4)}
\psellipse[linecolor=black, linewidth=0.04, dimen=outer](3.6,-0.2)(3.2,3.0)
\psline[linecolor=black, linewidth=0.04](9.6,2.8)(10.0,2.8)
\rput[bl](8.4,2.8){0.25}
\psline[linecolor=black, linewidth=0.02](10.8,-3.2)(10.8,-0.8)(12.0,-0.8)(12.0,-3.2)
\psline[linecolor=black, linewidth=0.02](13.2,-3.2)(13.2,1.2)(11.6,1.2)(11.6,-0.8)
\psline[linecolor=black, linewidth=0.02](14.4,-3.2)(14.4,-0.4)(15.6,-0.4)(15.6,-3.2)
\psline[linecolor=black, linewidth=0.02](12.4,1.2)(12.4,2.8)(14.8,2.8)(14.8,-0.4)
\psline[linecolor=black, linewidth=0.02](16.8,-3.2)(16.8,3.2)(13.6,3.2)(13.6,2.8)
\end{pspicture}
}
\caption{Kết quả khi ứng dụng nhóm trung bình}
\label{fig:pic28}
\end{figure}

\subsection{Phương pháp Ward và điểm trung tâm}
Phương pháp Ward tính độ tương đồng của 2 nhóm bằng sự gia tăng SSE~\cite{SSE} trong quá trình gom 2 nhóm với nhau~\cite{Vipin-Kumar, AHC, hierarchical-clustering}.
Vì vậy, phương pháp này có hàm mục tiêu giống như phương pháp gom nhóm K-means.
So với các phương pháp đã thảo luận ở trên thì phương pháp này có kỹ thuật hơi khác biệt.
Mặc dù vậy, công thức tính độ tương đồng của phương pháp Ward có chút tương đồng với phương pháp nhóm trung bình.
Điểm khác biệt của phương pháp Ward so với nhóm trung bình là khoảng cách giữa 2 điểm sẽ được bình phương lên.
Hình \ref{fig:pic29} (dựa vào hình 8.19~\cite{Vipin-Kumar}) sau đây thể hiện kết quả của việc áp dụng phương pháp Ward. Kết quả của quá trình gom nhóm hơi khác so với các kỹ thuật nhỏ nhất, lớn nhất và nhóm trung bình.

\begin{figure}[htp]
\psscalebox{1.0 1.0} % Change this value to rescale the drawing.
{
\begin{pspicture}(0,-3.6)(16.95,3.6)
\pscircle[linecolor=black, linewidth=0.04, dimen=outer](3.6,0.0){3.6}
\psellipse[linecolor=black, linewidth=0.04, dimen=outer](1.8,1.4)(1.4,0.6)
\psellipse[linecolor=black, linewidth=0.04, dimen=outer](4.6,-0.6)(1.4,0.6)
\rput[bl](1.6,-1.6){4}
\rput[bl](0.8,2.8){5}
\rput{-12.733253}(-0.38180095,0.578176){\rput[bl](2.4,2.0){2}}
\rput[bl](3.2,-2.4){3}
\rput[bl](5.2,-1.6){1}
\psdots[linecolor=black, dotsize=0.2](4.8,2.0)
\psdots[linecolor=black, dotsize=0.2](0.8,1.6)
\psdots[linecolor=black, dotsize=0.2](2.0,1.2)
\psdots[linecolor=black, dotsize=0.2](3.6,-0.4)
\psdots[linecolor=black, dotsize=0.2](5.2,-0.8)
\psdots[linecolor=black, dotsize=0.2](4.0,-2.0)
\rput[bl](5.2,2.0){1}
\rput[bl](2.4,1.2){2}
\rput[bl](4.0,-0.4){3}
\rput[bl](4.8,-0.8){6}
\rput[bl](4.4,-2.0){4}
\rput[bl](1.2,1.6){5}
\psline[linecolor=black, linewidth=0.04](9.6,1.6)(9.6,-3.2)(16.8,-3.2)
\psline[linecolor=black, linewidth=0.04](9.6,-2.0)(10.0,-2.0)(10.0,-2.0)
\psline[linecolor=black, linewidth=0.04](9.6,-0.8)(10.0,-0.8)
\psline[linecolor=black, linewidth=0.04](9.6,0.4)(10.0,0.4)
\psline[linecolor=black, linewidth=0.04](9.6,1.6)(10.0,1.6)
\rput[bl](8.4,-2.0){0.05}
\rput[bl](8.4,-0.8){0.1}
\rput[bl](8.4,0.4){0.15}
\rput[bl](8.4,1.6){0.2}
\rput[bl](10.8,-3.6){3}
\rput[bl](12.0,-3.6){6}
\rput[bl](13.2,-3.6){4}
\rput[bl](14.4,-3.6){1}
\rput[bl](15.6,-3.6){2}
\rput[bl](16.8,-3.6){5}
\psline[linecolor=black, linewidth=0.04](9.6,1.6)(9.6,3.2)(9.6,3.2)
\psline[linecolor=black, linewidth=0.04](9.6,2.8)(10.0,2.8)
\rput[bl](8.4,2.8){0.25}
\rput{-2.7438383}(0.04357028,0.2192881){\psellipse[linecolor=black, linewidth=0.02, dimen=outer](4.6,-0.8)(1.4,1.6)}
\rput{-36.899612}(1.1215355,2.5617006){\psellipse[linecolor=black, linewidth=0.02, dimen=outer](4.4,-0.4)(2.0,3.2)}
\psline[linecolor=black, linewidth=0.02](10.8,-3.2)(10.8,-1.2)(12.0,-1.2)(12.0,-3.2)
\psline[linecolor=black, linewidth=0.02](11.2,-1.2)(11.2,0.4)(13.2,0.4)(13.2,-3.2)
\psline[linecolor=black, linewidth=0.02](15.6,-3.2)(15.6,-0.8)(16.8,-0.8)(16.8,-3.2)(16.8,-3.2)
\psline[linecolor=black, linewidth=0.02](14.4,-3.2)(14.4,2.0)(12.0,2.0)(12.0,0.4)(12.0,0.4)
\psline[linecolor=black, linewidth=0.02](13.2,2.0)(13.2,3.2)(16.0,3.2)(16.0,-0.8)
\end{pspicture}
}
\caption{Kết quả khi ứng dụng phương pháp Ward}
\label{fig:pic29}
\end{figure}

Gom nhóm phân cấp kết hợp theo điểm trung tâm tính độ tương đồng của 2 nhóm bằng cách tính độ tương đồng của 2 điểm trung tâm trong mỗi nhóm~\cite{Vipin-Kumar, AHC, hierarchical-clustering, centroid}.
%Kỹ thuật này dường như tương đồng với K-means, nhưng cần lưu ý rằng phương pháp Ward mới gần tương đồng với kỹ thuật này.
Ngoài ra, kỹ thuật này còn có đặc tính khác là sự khả nghịch mà những kỹ thuật khác không thể hiện được.
Đối với các kỹ thuật gom nhóm phân cấp kết hợp khác đã được thảo luận, độ tương đồng của quá trình gom nhóm càng về sau thì càng giảm.
Tuy nhiên, sự khả nghịch của phương pháp gom nhóm phân cấp kết hợp theo điểm trung tâm có thể làm đảo chiều quá trình này, nghĩa là độ tương đồng về sau có thể tăng chứ không phải giảm.
Điều này thường được xem là không tốt vì gây khó khăn cho việc phân tích cây phân cấp.

\section{Công thức Lance-William cho độ tương đồng trong gom nhóm phân cấp kết hợp}
%~\cite{Vipin-Kumar}
Các cách tính độ tương đồng của 2 nhóm mà được thảo luận ở phần trước \ref{sec:cpplk} có thể được tham số hóa như trong công thức \ref{eq:2_Lance_William} với các hệ số từ bảng \ref{tab:2_3}.
Chẳng hạn, ta có 2 nhóm $Q$ và $R$ với $R$ được tạo thành từ quá trình gom 2 nhóm khác là $A$ và $B$.
Ta sẽ sử dụng công thức \ref{eq:2_Lance_William} để tính độ tương đồng của 2 nhóm $Q$ và $R$ với $p(.,.)$ là hàm tương đồng của 2 nhóm, $m_A, \, m_B$ và $m_Q$ là số lượng phần tử tương ứng trong nhóm $A, \, B$ và $Q$.
Độ tương đồng của 2 nhóm $Q$ và $R$ là hàm tuyến tính của độ tương đồng giữa $Q$ với 2 nhóm gốc $A$ và $B$ (2 nhóm tạo nên $R$).
%Sau khi gom 2 nhóm $A$ và $B$ thành nhóm $R$, độ tương đồng của nhóm mới $R$ đối với nhóm hiện có $Q$ là hàm tuyến tính của độ tương đồng của nhóm $Q$ đối với 2 nhóm ban đầu $A$ và $B$.
Bảng \ref{tab:2_3} chỉ ra những giá trị của hệ số cho những kỹ thuật mà đã được thảo luận.

\begin{equation}
\label{eq:2_Lance_William}
p(R,Q)=\alpha_Ap(A,Q) + \alpha_Bp(B,Q) + \beta p(A,B) + \gamma |p(A, Q) - p(B, Q)|
\end{equation}

Bất kỳ kỹ thuật gom nhóm phân cấp nào cũng có thể biểu diễn dưới dạng công thức của Lance-William.
Ngoài ra, gom nhóm phân cấp kết hợp còn có thể loại bỏ dữ liệu gốc ban đầu bằng việc xây dựng ma trận tương đồng như bảng \ref{tab:2_2}.
Khi có nhóm mới xuất hiện, ma trận tương đồng sẽ được cập nhật giá trị giữa nhóm mới với các nhóm khác theo kỹ thuật đã được thảo luận ở phần \ref{sec:cpplk}.
Nhìn chung, công thức Lance-William giúp cho ta có cái nhìn tổng quát cũng như là dễ thực thi trong lập trình.

\begin{table}[h!]
\centering
\caption{Hệ số của Lance-William dành cho các cách gom nhóm phân cấp}
\label{tab:2_3}
\resizebox{\textwidth}{!}{%
\begin{tabular}{|l|r|r|r|r|}
\hline
Cách thức gom nhóm & $\alpha_A$ & $\alpha_B$ & $\beta$ & $\gamma$ \\ \hline \hline
Liên kết đơn       & 1/2 & 1/2 & 0 & -1/2 \\ \hline
Liên kết toàn vẹn  & 1/2 & 1/2 & 0 & 1/2 \\ \hline
Nhóm trung bình    & $\frac{m_A}{m_A + m_B}$ & $\frac{m_B}{m_A + m_B}$ & 0 & 0 \\ \hline
Trung điểm         & $\frac{m_A}{m_A + m_B}$ & $\frac{m_B}{m_A + m_B}$ & $\frac{-m_A m_B}{(m_A + m_B) ^ 2}$ & 0 \\ \hline
Phương pháp Ward   & $\frac{m_A + m_Q}{m_A + m_B + m_Q}$ & $\frac{m_B + m_Q}{m_A + m_B + m_Q}$ & $\frac{-m_Q}{m_A + m_B + m_Q}$ & 0 \\ \hline
\end{tabular}
}
\end{table}

%\begin{table}[h!]
%\centering
%\caption{Hệ số của Lance-William dành cho các cách gom nhóm phân cấp}
%\label{tab:2_3}
%\resizebox{\textwidth}{!}{%
%\begin{tabular}{|l|r|r|r|r|}
%\hline
%Cách thức gom nhóm & $\alpha_A$ & $\alpha_B$ & $\beta$ & $\gamma$ \\ \hline \hline
%Liên kết đơn       & 1/2 & 1/2 & 0 & -1/2 \\ \hline
%Liên kết hoàn toàn & 1/2 & 1/2 & 0 & 1/2 \\ \hline
%Nhóm trung bình    & $\frac{m_A}{m_A + m_B}$ & $\frac{m_B}{m_A + m_B}$ & 0 & 0 \\ \hline
%Trung điểm         & $\frac{m_A}{m_A + m_B}$ & $\frac{m_B}{m_A + m_B}$ & $\frac{-m_A m_B}{(m_A + m_B) ^ 2}$ & 0 \\ \hline
%Phương pháp Ward   & $\frac{m_A + m_Q}{m_A + m_B + m_Q}$ & $\frac{m_B + m_Q}{m_A + m_B + m_Q}$ & $\frac{-m_Q}{m_A + m_B + m_Q}$ & 0 \\ \hline
%\end{tabular}%
%}
%\end{table}

\section{Những vấn đề trong gom nhóm phân cấp kết hợp}
\subsection{Thiếu hàm mục tiêu toàn cục}
%~\cite{Vipin-Kumar}
Gom nhóm phân cấp kết hợp không có hàm mục tiêu toàn cục để có thể tối ưu.
Thay vào đó, thuật toán sẽ sử dụng những kỹ thuật liên kết khác nhau \ref{sec:cpplk} để quyết định cục bộ ở mỗi bước thì hai nhóm nào sẽ được gom vào nhau.
Cách tiếp cận này giúp cho quá trình gom nhóm tránh được khó khăn khi giải quyết vấn đề tối ưu tổ hợp.
Nhìn chung, vấn đề của gom nhóm phân cấp là không thể có được hàm mục tiêu để có thể tối ưu như tối thiểu hóa SSE~\cite{SSE}.
Tuy nhiên, gom nhóm phân cấp kết hợp lại không gặp phải vấn đề về lựa chọn điểm để khởi tạo nhóm.

\subsection{Khả năng xử lý kích thước khác nhau của nhóm}
Một khía cạnh khác của gom nhóm phân cấp kết hợp mà vẫn chưa được thảo luận là làm sao để xử lý kích thước (số lượng điểm) của cặp nhóm vừa mới được gom vào thành một (vấn đề này chỉ được xem xét khi sử dụng độ tương đồng nhóm liên quan đến tổng như trung điểm, phương pháp Ward và nhóm trung bình).
Đối với vấn đề này, ta có 2 hướng tiếp cận khác nhau.
Hướng tiếp cận đầu tiên là xem mỗi điểm trong nhóm là khác nhau nên mỗi điểm sẽ có trọng số kèm theo.
Hướng tiếp còn lại là không có trọng số, nghĩa là xem tất cả các điểm trong nhóm là như nhau.
Nhìn chung, cách tiếp cận không có trọng số được sử dụng nhiều trừ khi có lý do để tin tưởng rằng mỗi một điểm nên có trọng số riêng.

Sau đây, ta sử dụng kỹ thuật gom nhóm trung bình \ref{sec:ntb} với phiên bản sử dụng không có trọng số.
Ta sẽ chuyển kỹ thuật này thành tham số của công thức \ref{eq:2_Lance_William} như bảng \ref{tab:2_3}.
Hệ số trong trường hợp này liên quan đến kích thước của mỗi nhóm khi được gom là:
\begin{equation}
\begin{aligned}
\alpha_A &= \frac{m_A}{m_A + m_B}	\\
\alpha_B &= \frac{m_B}{m_A + m_B}	\\
\beta &= 0		\\
\gamma &= 0
\end{aligned}
\end{equation}

Trong trường hợp ta sử dụng gom nhóm trung bình có trọng số thì hệ số sẽ là:
\begin{equation}
\begin{aligned}
\alpha_A &= \frac{1}{2} \\
\alpha_B &= \frac{1}{2} \\
\beta &= 0		\\
\gamma &= 0		\\
\end{aligned}
\end{equation}

\subsection{Quyết định gom nhóm không thể đảo ngược}
Gom nhóm phân cấp kết hợp có xu hướng chọn quyết định cục bộ tốt về kết hợp giữa 2 nhóm khi mà cùng sử dụng thông tin về độ tương đồng của tất cả các cặp điểm của nhau.
Tuy nhiên, một khi quyết định gom 2 nhóm lại với nhau được thực hiện thì không có cách nào có thể làm ngược lại.
Cách tiếp cận này ngăn chặn tối ưu toàn cục vì không còn có thể thay đổi được ở cục bộ sau khi 2 nhóm được gom lại.
Như đã giới thiệu, phương pháp Ward quyết định gom 2 nhóm dựa vào tiêu chuẩn tối tiểu hóa lỗi bình phương từ K-means.
Tuy nhiên, các nhóm tại mỗi cấp thì lại không thể hiện cục bộ tối tiểu tương ứng theo tổng SSE~\cite{SSE}.
Phương pháp Ward thường được xem như là cách hiệu quả trong việc khởi tạo nhóm cho K-means.

Một vài kỹ thuật được đề xuất đề cố gắng vượt quá giới hạn của quyết định gom nhóm nhưng lại không thể đảo ngược.
Một hưóng tiếp cận để giải quyết vấn đề này là di chuyển những nhánh cây ra xung quanh để cải thiện hàm mục tiêu toàn cục.
Hướng tiếp cận khác là sử dụng kỹ thuật gom nhóm phân chia như K-means để tạo ra nhiều nhóm nhỏ, rồi sử dụng gom nhóm phân cấp với những nhóm nhỏ này như là điểm khởi đầu.

\subsection{Xác định vết cắt tại đồ thị cây phân cấp}
Như đã nói, đôi lúc ta muốn xác định số lượng nhóm tồn tại thực sự trong dữ liệu như gom nhóm phân chia.
Để làm được điều này, ta cần cắt tại một mức độ nào đó trên cây phân cấp để xác định số lượng nhóm.
Tuy nhiên, vết cắt này cần được xác định rõ ràng và phải hợp lý.
Một vài cách thức để thực hiện vết cắt như sau:
\begin{enumerate}
\item[•]Ta có thể cắt tại một ngưỡng cho trước.
Ví dụ, ta có thể cắt đồ thị cây phân cấp tại giá trị bất kì cho trước.
Giá trị này cho ta biết được độ tương đồng tối thiểu để gom các nhóm lại.
\item[•]Khi khoảng cách để gom các nhóm lại là lớn nhất (nghĩa là có độ tương đồng nhỏ nhất), ta có thể cắt tại độ thị ở mức mà trước khi điều này xảy ra.
Khi thuật toán gom nhóm ở khoảng cách lớn nhất sẽ gây ảnh hưởng đến chất lượng các nhóm được gom.
Vì vậy, việc vết cắt được thực hiện ở thời điểm trước khi điều này xảy ra sẽ giúp ta tránh được kết quả gom nhóm xấu.
\end{enumerate}
%\subsection{Các cách để thực hiện vết cắt trên HAC}
%Giải thích

%Cách để thực hiện vết cắt của gom nhóm phân cấp
%Sau đây, một vài cách để cắt tạo thành các nhóm thường được sử dụng trong HAC

%Các công thức tính khoảng cách
%Trong gom nhóm văn bản, các công thức tính khoảng cách được sử dụng để đo lường giữa hai văn bản. Sau đây, một số công thức tính khoảng cách thường được sử dụng:
%\begin{enumerate}
%\item[•]Khoảng cách Euclidean : $\parallel a \,- \, b \parallel_2 \, = \, \sqrt{\underset{i}{\sum}(a_i \, - \, b_i)^2} $
%\item[•]Khoảng cách Euclidean vuông : $\parallel \, a \, - \, b \, \parallel^2_2 \, = \, \underset{i}{\sum} (a_i - b_i)^2$
%\item[•]Khoảng cách Manhatthan : $\parallel \, a \, - \, b \, \parallel_1 \, = \, \underset{i}{\sum} \mid a_i \, - \, b_i\mid$
%\item[•]Khoảng cách cực đại : $\parallel \, a \, - \, b\, \parallel_\infty \, = \, \underset{i}{max} \mid a_i \, - \, b_i \mid$
%\item[•]Khoảng cách Mahalanobis : $\sqrt{(a \, - \, b)^{\top} \, S^{-1} \, (a \, - \, b)}$ với $S$ là ma trận covariance
%\end{enumerate}


%Các phương pháp liên kết là tiêu chuẩn dùng đề tác động khoảng cách được sử dụng giữa các tập đặc trưng.
%Thuật toán sẽ hợp các cặp phân nhóm với nhau với việc tối thiểu hóa các tiêu chuẩn này.
%Gom nhóm phân cấp có 3 phương pháp liên kết: liên kết đơn, liên kết hoàn toàn và liên kết trung bình.
%\begin{enumerate}
%\item[•]Liên kết đơn tối thiểu hóa độ biến thiên giữa các phân nhóm được ghép vào nhau.
%\item[•]Liên kết hoàn toàn tối thiểu hóa khoảng cách tối đa giữa các đặc trưng của hai tập.
%\item[•]Liên kết trung bình tối thiểu hóa cách trung bình của mỗi đặc trưng của hai tập.
%\end{enumerate}
 
%Đồng gom nhóm phân cấp
%\section{Đồng gom nhóm phân cấp}
%
%	%Giới hạn của gom nhóm
%	\subsection{Giới hạn của gom nhóm}
%	
%	%Giới thiệu đồng gom nhóm
%	\subsection{Giới thiệu đồng gom nhóm}
%	Đồng gom nhóm là phương pháp gom nhóm cả hai chiều của dữ liệu(bao gồm gom nhóm văn bản và gom nhóm đặc trưng).
%	Đây là phương pháp hiệu quả vì khai thác được độ tương đồng của các phân nhóm trong chiều này của dữ liệu để gom nhóm trong chiều khác.
%	Điều này có nghĩa các phân nhóm của văn bản được đánh giá bằng các phân nhóm của đặc trưng và ngược lại.
%	Bằng cách này, các phân nhóm của văn bản có thể được tiến hành dựa trên các phân nhóm đặc trưng để giúp làm giảm số chiều của dữ liệu.
%	Như vậy, đồng gom nhóm là phương pháp hữu hiệu để giúp ta gom nhóm văn bản đồng thời làm giảm số chiều của dữ liệu.
%	%Ví sao chọn đồng gom nhóm
%	%Lợi ích của đồng gom nhóm
%	%Sự kết hợp của đồng gom nhóm với gom nhóm phân cấp
%	%Công dụng của đồng gom nhóm phân cấp
%	%Lợi ích của đồng gom nhóm phân cấp
%	%Giải thích việc tương tác hai chiều dữ liệu
%	%Kết luận về đồng gom nhóm phân cấp
%
%%%14
%
%%Độ tương đồng Goodman kruskal
%\section{Độ tương đồng Goodman kruskal}
%
%	\subsection{Contingency table}
%	%Giới thiệu về contingency table
%	%Ví dụ về contingency table
%	
%	\subsection{Độ tương đồng Goodman Kruskal}
%	%Giới thiệu về độ tương đồng Goodman Kruskal
%	%Vì sao chọn độ tương đồng Goodman Kruskal
%	%Sử dụng Goodman kruskal trong đồng gom nhóm phân cấp
%	%Luật sử dụng cho độ tương đồng Goodman kruskal
%		%Giới thiệu
%		%Luật thứ nhất
%		%Luật thứ hai
%	%Ví dụ về cách tính Goodman kruskal
%	%Cách để gom nhóm những hàng với nhau
%	%Cách để gom nhóm những cột với nhau
%%%11
%
%%Thuật toán đồng gom nhóm phân cấp
%\section{Thuật toán đồng gom nhóm phân cấp}
%	%Các ý cần giải thích
%	\subsection{Các ý cần giải thích}
%		%Dòng, cột, ma trận trong thuật toán(2 đoạn)
%		%Mục tiêu của đồng gom nhóm phân cấp
%		%Điều kiện của các nhóm trong hàng, cột
%		%Cách để gom lại các nhóm trong ma trận
%		%Cách để tạo ra phân nhóm đầu tiên(2 đoạn)
%	%%7
%	
%	\subsection{Áp dụng Goodman kruskal trong đồng gom nhóm phân cấp}
%	%Sử dụng Goodman kruskal trong đồng gom nhóm phân cấp
%	
%		%Giới thiệu
%		\subsubsection{Giới thiệu}		
%		
%		%Công thức
%		\subsubsection{Công thức}
%		
%		%Ý tưởng
%		\subsubsection{Ý tưởng}
%			%Định nghĩa một
%			%Định nghĩa hai
%		%Giải thích
%		%Cách áp dụng để tính nhóm thích hợp
%		%Giải thích cách tính	
%		%Sự tối ưu hóa của cách tính
%		%Sự lựa chọn nhóm được gom
%		%kết quả cuối cùng khi áp dụng
%		%kết luận
%	%%11
%	%Thuật toán
%	\subsection{Thuật toán}
%		%Giải thuật 1
%		%Giải thuật 2
%		%Giải thuật 3
%		%Giải thuật 4
%		%Giải thích toàn bộ giải thuật của thuật toán(15 đoạn)
%	%%19
%	
%	%Những lưu ý về đồng gom nhóm phân cấp
%	\subsection{Những lưu ý về đồng gom nhóm phân cấp}
%		%Các tác động đến đồng góm nhóm phân cấp
%		%Sự ảnh hưởng của từng tác động này(6 đoạn)
%	%%7
%%%44
%		
%%Đồng gom nhóm phân cấp theo hướng tăng trưởng
%\section{Đồng gom nhóm phân cấp theo hướng tăng trưởng}
%	%Điểm hạn chế hiện thời của đồng gom nhóm phân cấp
%	%Phân tích tình huống của thuật toán hiện thời
%	%Phát triển thuật toán cho dữ liệu tăng trưởng theo thời gian thực
%	%hướng cần phải giải quyết
%	%Các giải thuật
%		%Giải thuật 5
%		%Giải thuật 6
%	%Giải thích các giải thuật theo hướng tổng quát(15 đoạn)
%%%21
%%%90	
%
%
%%%%%%%%%%%%%%%%%%%%%%%%%%%%%%%%%%%%%%%%%%%%%%%%%%%%%%%%%%%%%%%%%%%55
%%\section{Các phương pháp gom nhóm}
%%%kể ra một vài phương pháp gom nhóm: kmeans, DBScan
%%
%%\section{Phương pháp gom nhóm phân cấp}
%%\subsection{Các hướng tiếp cận}
%%\subsubsection{Hướng tiếp cận agglomerative}
%%%Giới thiệu hướng tiếp cận agglomerative
%%
%%%Ví dụ về agglomerative
%%
%%\subsubsection{Hướng tiếp cận divisive}
%%%Giới thiệu hướng tiếp cận divisive
%%
%%%Ví dụ về divisive
%%
%%\section{Đồng gom nhóm phân cấp}
%%\subsection{Giới thiệu}
%%%Điểm hạn chế về gom nhóm
%%
%%%Giới thiệu về đồng gom nhóm
%%
%%%Sự kết hợp của đồng gom nhóm phân cấp
%%
%%%Công dụng của đồng gom nhóm phân cấp
%%
%%%Giải thích việc tương tác hai chiều của dữ liệu
%%
%%%Kết luận về đồng gom nhóm phân cấp
%%
%%\subsection{Độ tương đồng Goodman Kruskal $\tau$}
%%%Vì sao chọn Goodman Kruskal $\tau$
%%
%%%Giới thiệu Goodman Kruskal $\tau$
%%
%%%Luật sử dụng Goodman Kruskal $\tau$
%%
%%%Ví dụ về Goodman Kruskal
%%
%%\subsection{Thuật toán}
%%%Giới thiệu thuật toán
%%
%%%Các giải thuật
%%
