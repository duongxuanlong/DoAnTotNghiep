\chapter*{Tóm tắt}
\label{tomtat}

\section*{Tổng quan đề tài}
Ngày nay, internet đang phát triển ở tốc độ vũ bão dẫn đến lượng thông tin được phổ cập đến người dùng diễn ra một cách nhanh chóng.
Việc có quá nhiều tin tức cung cấp cho người dùng khiến cho họ gặp nhiều khó khăn khi xử lý lượng thông tin khổng lồ này.
Vì vậy, ta cần tổ chức lại thông tin cho người dùng để có thể giúp họ truy cập một cách dễ dàng hơn.
Một trong số những cách này chính là gom nhóm văn bản.
Gom nhóm văn bản giúp cho các văn bản có cùng chủ đề thành những nhóm riêng biệt.
Qua đó, người dùng cho thể truy vấn dựa vào các nhóm chủ đề khác nhau và giúp cho họ có thể truy cập thông tin mong muốn một cách nhanh gọn, chính xác hơn.

\section*{Thực trạng và mục tiêu}
Hiện nay, gom nhóm văn bản có nhiều cách khác nhau.
Tuy nhiên, một vấn đề quan trọng cần phải được giải quyết là thể hiện văn bản.
Gom nhóm văn bản sẽ thực hiện dựa vào vector thể hiện cho văn bản.
Vector này được hình thành dựa vào nhiều phương pháp khác nhau như: tần số, lựa chọn đặc trưng, \ldots
Những cách này vẫn chưa thể giải quyết được vấn đề triệt để là chiều của văn bản quá cao, độ thưa thớt trong vector của văn bản hoặc bảo toàn ngữ nghĩa trong vector thể hiện văn bản.

Vì vậy, mục tiêu của đồ án là giảm số chiều của vector nhưng vẫn đảm bảo kết quả cho gom nhóm văn bản.
Đồ án đề xuất dùng doc2vec để biểu diễn thể hiện cho văn bản.
Trong phần thực nghiệm, đồ án sẽ so sánh kết quả gom nhóm văn bản giữa việc sử dụng 2 cách biểu diễn văn bản khác nhau là doc2vec và TFIDF.
Kết quả cho thấy là sử dụng doc2vec cho hiệu quả tốt hơn.

\section*{Cấu trúc của đồ án}
Đồ án được chia thành 5 chương với nội dung chính như sau:
\begin{enumerate}
\item[•]Chương 1: giới thiệu về gom nhóm văn bản, những ích lợi đồng thời các phương pháp gom nhóm văn bản.
\item[•]Chương 2: giới thiệu gom nhóm phân cấp, các hướng tiếp cận cũng như là những vấn đề trong phương pháp gom nhóm này.
\item[•]Chương 3: đề xuất sử dụng doc2vec dùng để thể hiện văn bản cho phương pháp gom nhóm phân cấp.
\item[•]Chương 4: trình bày kết quả thực nghiệm theo phương pháp đề xuất ở chương 3 cho dữ liệu ở tiếng Anh lẫn tiếng Việt.
\item[•]Chương 5: kết quả đạt được, nêu ra hướng phát triển trong tương lai.
\end{enumerate}