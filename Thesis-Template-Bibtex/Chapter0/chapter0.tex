\chapter*{Tổng quan}
\label{Chapter0}
\addcontentsline{toc}{chapter}{Tổng quan đề tài}

\hspace{10mm}Ngày nay, internet đang phát triển ở một tốc độ chóng mặt dẫn đến thông tin được phổ cập đến người dùng diễn ra một cách nhanh chóng, khác hoàn toàn với thời trước khi mà chúng ta cần đến báo chí, đài phát thanh để biết tin tức. Ngày nay, mọi thông tin đều hiện diện trên internet. Có thể nói, internet chứa nguồn tài nguyên vô hạn không khác gì một kho tàng bất tận cho những ai biết tận dụng nó. \\
\hspace*{10mm}Trong nguồn tài nguyên vô hạn đó thì tin tức là một tài nguyên mà được nhiều người tìm đến nhất. Có lẽ ngày nay việc chúng ta hay đọc báo trên mạng không không còn là chuyện mới mẻ gì nữa. Và cũng ngày càng có nhiều người lựa chọn hình thức đọc báo trên mạng. Điều đó góp phần thúc đẩy báo mạng phát triển và ngày càng có nhiều trang báo mạng được ra đời như nấm mọc sau mưa. \\
\hspace*{10mm}Chính tốc độ phát triển như vũ bão của báo mạng đang đe dọa đến sự tồn tại của ngành công nghiệp báo chí truyền thống. Chính vì thế mà chúng ta thấy được một cuộc cách mạng khi mà các tờ báo giấy đang chuyển mình thành các trang báo mạng. Và không chỉ có sự tham gia của các tờ báo truyền thống, các đại gia trong ngành công nghệ thông tin cũng gia nhập trào lưu này như Facebook, Google và Apple khiến cho cuộc đua tranh ở lĩnh vực này đang ngày một khốc liệt. Facebook thì tích hợp hệ thống đọc tin tức Feed ngay trên trang mạng xã hội của mình. Google thì cho ra mắt hệ thống đọc tin tức Google news. Còn Apple thì mới tích hợp Apple news ngay trong iOS\textunderscore{}9. \\
\hspace*{10mm}Việc các đại gia trong ngành công nghệ thông tin đua nhau vào mảng nội dung, tin tức trên mạng cho thấy sắp tới đây sẽ là lĩnh vực đầy tìm năng và hứa hẹn. Tuy nhiên, việc chỉ có tin tức hay nội dung không thì vẫn là chưa đủ, người dùng ngoài việc đọc báo ra còn muốn theo dõi các tin tức cũng như là các nội dung liên quan có cùng chủ đề. Tin tức trên mạng thì tràn lan đại khải, một số người thì chỉ quan tâm đến chủ đề này, còn một số khác thì quan tâm đến chủ đề khác. Cho nên việc cho người dùng lựa chọn các bài báo mà có chủ đề mà mình yêu thích là nhu cầu thiết thực và hợp lí. Đề làm được như vậy, chúng ta cần phân loại văn bản hay bài báo để giúp cho người dùng dễ theo dõi các tin tức mà mình mong muốn.\\
\hspace*{10mm}Vì thế mà em đã quyết định chọn bài toán phân loại văn bản để thực hiện đồ án tốt nghiệp của mình.\\
\hspace*{10mm}Phân loại là một tiến trình tổ chức dữ liệu thành những nhóm mà mỗi nhóm có các đối tượng có độ tương đồng cao. Đây là công việc chính trong khai thác dữ liệu, và là kỹ thuật chung cho trong phân tích dữ liệu thống kê được sử dụng trong nhiều lĩnh vực khác nhau như là máy học, nhận dạng mẫu, phân tích ảnh và truy vấn thông tin \ldots\\
\hspace*{10mm}Phân loại cũng đồng thời giúp các nhà nghiên cứu thị trường phát hiện ra được những nhóm khách hàng riêng biệt thông qua quá trình mua hàng và từ đó giúp cho các nhà nghiên cứu thị trường có thể đặc trưng hóa nhóm khách hàng này. Ngoài ra, trong lĩnh vực sinh học, phân loại cũng có thể được sử dụng để dẫn xuất gen của cây cối, động vật thành những nhóm có cùng chức năng để có thể hiểu được hoạt động tổng quan của chúng. Mục đích của việc phân loại tổ chức dữ liệu thành nhóm để thể hiện cấu trúc bên trong của dữ liệu và đôi khi việc chia cắt dữ liệu cũng là mục đích chính. Ngoài ra thì phân loại cũng là bước chuẩn bị cho kỹ thuật AI(tóm tắt văn bản).\\
\hspace*{10mm}Vấn đề của phân loại là tìm được những đối tượng tương đồng, nhưng làm cách nào để có thể tìm được những đối tượng đó? Khi nói đến đây chúng ta sẽ liên tưởng đến việc tính khoảng cách của các đối tượng. Việc tính toán đó sẽ giúp chúng ta tìm được những đối tượng gần nhau để từ đó có thể dễ dàng xếp thành những nhóm có đặc trưng gần giống nhau. Chúng ta thấy là phân loại được áp dụng cho nhiều lĩnh vực khác nhau nên khái niệm khoảng cách cũng được hiểu theo tùy ngữ cảnh. Và chính vì có quá nhiều lĩnh vực áp dụng phân loại nên em sẽ không thể nào nghiên cứu hết được. Ở đây em chọn lĩnh vực là phân loại văn bản cho đồ án tốt nghiệp của mình.\\
\hspace*{10mm}Sau đây, em giới thiệu sơ bộ chác chương trong đồ án của mình: \\
\begin{enumerate}
\vspace{-10mm}
\addtolength{\itemindent}{5mm}
\item[] \hyperref[Chapter1]{\textcolor{blue}{Chương 1}} - Giới thiệu
\item[] \hyperref[Chapter2]{\textcolor{blue}{Chương 2}} - Các phương pháp phân loại
\item[] \hyperref[Chapter3]{\textcolor{blue}{Chương 3}} - Phương pháp đề xuất
\item[] \hyperref[Chapter4]{\textcolor{blue}{Chương 4}} - Thực nghiệm và kết quả
\item[] \hyperref[Chapter5]{\textcolor{blue}{Chương 5}} - Kết luận và hướng phát triển
\end{enumerate}

